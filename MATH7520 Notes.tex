%% LyX 2.2.1 created this file.  For more info, see http://www.lyx.org/.
%% Do not edit unless you really know what you are doing.
\documentclass[11pt,english]{article}
\usepackage[LGR,T1]{fontenc}
\usepackage[latin9]{inputenc}
\usepackage{geometry}
\geometry{verbose,tmargin=1in,bmargin=1in,lmargin=1in,rmargin=1in}
\usepackage{fancyhdr}
\pagestyle{fancy}
\setcounter{secnumdepth}{2}
\setcounter{tocdepth}{2}
\synctex=-1
\usepackage{mathrsfs}
\usepackage{mathtools}
\usepackage{amsmath}
\usepackage{amsthm}
\usepackage{amssymb}
\usepackage{stmaryrd}
\usepackage{setspace}
\usepackage[all]{xy}
\doublespacing

\makeatletter

%%%%%%%%%%%%%%%%%%%%%%%%%%%%%% LyX specific LaTeX commands.
\DeclareRobustCommand{\greektext}{%
  \fontencoding{LGR}\selectfont\def\encodingdefault{LGR}}
\DeclareRobustCommand{\textgreek}[1]{\leavevmode{\greektext #1}}
\ProvideTextCommand{\~}{LGR}[1]{\char126#1}


%%%%%%%%%%%%%%%%%%%%%%%%%%%%%% Textclass specific LaTeX commands.
  \theoremstyle{definition}
  \newtheorem{defn}{\protect\definitionname}[section]
  \theoremstyle{definition}
  \newtheorem{example}{\protect\examplename}[section]
  \theoremstyle{plain}
  \newtheorem{prop}{\protect\propositionname}[section]
  \theoremstyle{plain}
  \newtheorem{cor}{\protect\corollaryname}[section]
  \theoremstyle{remark}
  \newtheorem{rem}{\protect\remarkname}[section]
  \theoremstyle{plain}
  \newtheorem{lem}{\protect\lemmaname}[section]
  \theoremstyle{plain}
  \newtheorem{thm}{\protect\theoremname}[section]
  \theoremstyle{plain}
  \newtheorem{fact}{\protect\factname}[section]
  \theoremstyle{remark}
  \newtheorem{claim}{\protect\claimname}[section]

%%%%%%%%%%%%%%%%%%%%%%%%%%%%%% User specified LaTeX commands.

\renewenvironment{bmatrix}{\begin{ytableau}}{\end{ytableau}}

\usepackage{tikz}
\usetikzlibrary{matrix,arrows}

\makeatother

\usepackage{babel}
  \providecommand{\claimname}{Claim}
  \providecommand{\definitionname}{Definition}
  \providecommand{\examplename}{Example}
  \providecommand{\factname}{Fact}
  \providecommand{\lemmaname}{Lemma}
  \providecommand{\propositionname}{Proposition}
  \providecommand{\remarkname}{Remark}
\providecommand{\corollaryname}{Corollary}
\providecommand{\theoremname}{Theorem}

\begin{document}

\title{MATH7520 Principal G-Bundles and Clasifying Spaces}

\author{Spring 2019\\
Taught by Brian Hwang\\
Notes by Yun Liu}

\maketitle
\pagebreak{}\tableofcontents{}

\pagebreak{}

\part{Principal G-bundles}

\section{Fiber Bundle}

\subsection{Fiber Bundle}
\begin{defn}
A \emph{fiber bundle} is a structure $\left(E,B,\pi,F\right)$, where
$E,B$, and $F$ are topological spaces and $\pi:E\to B$ is a continuous
surjection satisfying a local triviality condition: for every $x\in E$,
there is an open neighborhood $U\subseteq B$ of $\pi\left(x\right)$
such that there is a homeomorphism $\varphi:\pi^{-1}\left(U\right)\to U\times F$
in such a way that $\pi$ agrees with the projection onto the first
factor. That is, the following diagram should commute: 
\[
\xymatrix{\pi^{-1}\left(U\right)\ar[rr]\ar[dr]_{\pi} &  & U\times F\ar[dl]^{\text{pr}_{1}}\\
 & U
}
\]
The set of all $\left\{ \left(U_{i},\varphi_{i}\right)\right\} $
is called a \emph{local trivialization} of the bundle.Thus for any
$p\in B$ in B, the preimage $\pi^{-1}\left(p\right)$ is homeomorphic
to $F$ and is called the \emph{fiber} over $p$. Every fiber bundle
$\pi:E\to B$ is an open map, since projections of products are open
maps. Therefore $B$ carries the quotient topology determined by the
map $\pi$. The space $B$ is called the \emph{base space} of the
bundle, $E$ the total space, and $F$ the fiber. The map $\pi$ is
called the projection map (or bundle projection). A fiber bundle $\left(E,B,\pi,F\right)$
is often denoted 
\[
\xymatrix{F\ar@{^{(}->}[dr]\\
 & E\ar[d]^{\pi}\\
 & B
}
\]
\end{defn}
%
\begin{example}
(Trivial bundle) Let $E=B\times F$ and let $\pi:E\to B$ be the projection
onto the first factor. Then $E$ is a fiber bundle (of $F$) over
$B$. Here $E$ is not just locally a product but globally one. Any
such fiber bundle is called a \emph{trivial bundle}. Any fiber bundle
over a contractible CW-complex is trivial.
\end{example}
%
\begin{example}
(Covering space) A covering space is a fiber bundle such that the
bundle projection is a local homeomorphism. It follows that the fiber
is a discrete space.
\end{example}
\begin{defn}
A \emph{fibration} (or \emph{Hurewicz fibration} or \emph{Hurewicz
fiber space}) is a continuous mapping ${\displaystyle p\colon E\to B}$
satisfying the homotopy lifting property with respect to any space,
i.e. for any space $X$ and for any homotopy ${\displaystyle f\colon X\times[0,1]\to B}$,
and for any map ${\displaystyle \tilde{f}_{0}\colon X\to E}$ lifting
${\displaystyle f_{0}=f|_{X\times\{0\}}}$ (i.e., so that ${\displaystyle f_{0}=\pi\circ\tilde{f}_{0}}$),
there exists a homotopy ${\displaystyle \tilde{f}\colon X\times[0,1]\to E}$
lifting $f$, (i.e., so that ${\displaystyle f=\pi\circ\tilde{f}}$,)
which also satisfies ${\displaystyle \tilde{f}_{0}=\tilde{f}|_{X\times\{0\}}}$.
In other words, we have the following diagram
\[
\xymatrix{X\ar[r]^{\tilde{f}_{0}}\ar[d]_{X\times\left\{ 0\right\} } & E\ar[d]^{p}\\
X\times I\ar[r]^{f}\ar@{..>}[ur]^{\tilde{f}} & B
}
\]
\end{defn}
%
\begin{example}
Fibre bundles are important examples of fibrations. 
\end{example}
\begin{defn}
A \emph{vector bundle} is a fibre bundle with whose fibers are vector
spaces. 
\end{defn}
\begin{example}
\emph{Tangent} and \emph{cotangent bundles} of smooth manifolds are
vector bundles. 
\end{example}
%
\begin{example}
A \emph{sphere bundle} is a fiber bundle whose fiber is an $n$-sphere.
Given a vector bundle $E$ with a metric (such as the tangent bundle
to a Riemannian manifold) one can construct the associated unit sphere
bundle, for which the fiber over a point $x$ is the set of all unit
vectors in $E_{x}$. When the vector bundle in question is the tangent
bundle $TM$, the unit sphere bundle is known as the unit tangent
bundle, and is denoted $UTM$. 
\end{example}
%
Fiber bundles often come with a group of symmetries which describe
the matching conditions between overlapping local trivialization charts. 

Specifically, let $G$ be a topological group which acts continuously
on the fiber space $F$ on the right. We lose nothing if we require
$G$ to act faithfully on $F$ so that it may be thought of as a group
of homeomorphisms of $F$. A $G$-atlas for the bundle $\left(E,B,\pi,F\right)$
is a local trivialization such that for any two local trivializations
for the overlapping charts $\left(U_{i},\varphi_{i}\right)$ and $\left(U_{j},\varphi_{j}\right)$
the function 

\[
{\displaystyle \varphi_{i}\varphi_{j}^{-1}\colon\left(U_{i}\cap U_{j}\right)\times F\to\left(U_{i}\cap U_{j}\right)\times F}
\]
is given by

\[
{\displaystyle \varphi_{i}\varphi_{j}^{-1}\left(x,\xi\right)=\left(x,t_{ij}\left(x\right)\xi\right)}
\]
where $t_{ij}:U_{i}\cap U_{j}\to G$ is a continuous map called a
\emph{transition function}. Two $G$-atlases are equivalent if their
union is also a $G$-atlas. A $G$-bundle is a fiber bundle with an
equivalence class of $G$-atlases. The group $G$ is called the \emph{structure
group} of the bundle. 

\begin{defn}
A \emph{bundle map} (or \emph{bundle morphism}) consists of a pair
of continuous functions ${\displaystyle \varphi\colon E\to F,f\colon M\to N}$
such that ${\displaystyle \pi_{F}\circ\varphi=f\circ\pi_{E}}$. That
is, the following diagram commutes: 
\[
\xymatrix{E\ar[r]^{\varphi}\ar[d]_{\pi_{E}} & F\ar[d]^{\pi_{F}}\\
M\ar[r]^{f} & N
}
\]
\end{defn}
\pagebreak{}

\section{Principal G-bundles}
\begin{defn}
A \emph{principal $G$-bundle} is a fiber bundle $\pi:E\to B$ with
a free and transitive (right) action by $G$ on the fibers, such that
the local trivializations intertwine the right $G$-action with right
translation (i.e. if $\varphi:\pi^{-1}\left(U\right)\to U\times G$
and $h\left(p\right)=\left(u,g\right)$, then $h\left(p\cdot g'\right)=\left(u,gg'\right)$). 
\end{defn}
Equivalently, $E$ is equipped with a free right $G$-action such
that $\pi:E\to E/G\cong B$ and it satisfies some appropriate local
trivialization condistion.
\begin{description}
\item [{Warning.}] Principal $G$-bundles are not just $G$-fibre bundles.
There is no canonical identity on fibre $F$. More specifically, fibres
of principal $G$-bundles are $G$-tosors, i.e. principal homogeneous
spaces. 
\end{description}
\begin{example}
(unit tangent vectors on $S^{2}$) $E=\left\{ \left(x,v\right)\in TS^{2}|\left|\left|v\right|\right|=1\right\} \to S^{2}$.
This is a nontrivial principal $S^{1}$-bundle because it contains
no global sections. 
\end{example}
%
\begin{example}
(Hopf fibration) $S^{3}\to S^{2}$ is a principal $S^{1}$-bundle. 
\end{example}
\begin{description}
\item [{Question.}] How to tell the difference between the two? 
\end{description}
\begin{quote}
Principal $G$-bundles are much more rigid than fibre bundles. 
\end{quote}
\begin{prop}
\label{prop-id-lift} If $E\to B,E'\to B'$ are principal $G$-bundles
and $\phi:E\to E'$ is a principal bundle morphism such that the underlying
map is identity $\text{Id}_{B}:B\to B$, then $\phi$ is an isomorphism. 
\end{prop}
\begin{proof}
Consider the commutative diagram
\[
\xymatrix{ & E'\ar@{-->}[dd]\ar@{-->}[dr]\\
E\ar@{..>}[ur]^{\phi}\ar[dd]\ar[rr] &  & EG\ar[dd]\\
 & B\ar@{-->}[dr]\\
B\ar[rr]\ar@{=}[ur]_{Id} &  & BG
}
\]
Since $\text{Bun}_{G}\left(B\right)\cong\left[B,BG\right]$, homotopy
classes of maps $B\to BG$ corresponds to isomorphism classes of principal
$G$-bundles on $B$. 
\end{proof}

\subsection{Sections of Principal $G$-bundles}

Recall a section of $E\xrightarrow{\pi}B$ over $U\subseteq B$ is
a map $s:U\to E$ such that $\pi\circ s=\text{Id}_{U}:U\to U$ and
a trvialization over $U$ is a (continuous) map $h:\pi^{-1}\left(U\right)\to U\times G$
which intertwines the right $G$-action with the right translation. 
\begin{prop}
There is a bijection 
\[
\left\{ \text{sections }s\in\Gamma\left(U,E\right)\right\} \leftrightarrow\left\{ \text{trivialization }h:\pi^{-1}\left(U\right)\to U\times G\right\} .
\]
\end{prop}
\begin{proof}
Given a section $s\in\Gamma\left(U,E\right)$, we can define a trivialization
as follows. For any $p\in\pi^{-1}\left(U\right)$ , let $b=\pi\left(p\right)\in U$
and $g=\tau\left(s\left(b\right),p\right)\in G$ is the element which
sends $s\left(b\right)$ to $p$, then we have a map 
\[
\begin{array}{cccc}
h: & \pi^{-1}\left(U\right) & \to & U\times G\\
 & p & \mapsto & \left(\pi\left(p\right),g\right)
\end{array}
\]
which intertwines the $G$-action $\tau\left(s\left(b\right),p\cdot g'\right)=\tau\left(s\left(b\right),p\right)\cdot g'$,
and the inverse map is given by $\left(b,g\right)\mapsto s\left(b\right)\cdot g$. 

Conversely, given a trivialization $h:\pi^{-1}\left(U\right)\to U\times G$
we have a canonical section 
\[
\begin{array}{cccc}
s: & U & \to & E\\
 & b & \mapsto & h^{-1}\left(b,e\right)
\end{array}
\]
which is an element in $\Gamma\left(U,E\right)$. 
\end{proof}
\begin{cor}
A principal $G$-bundle is trivial if and only if it admits global
sections. 
\end{cor}
\begin{rem}
The key point is, sections $s\in\Gamma\left(U,E\right)$ gives a choice
of identity on fibre over $b\in U$. 
\end{rem}

\subsection{Associated Bundles}

Given a fibre bundle, we can construct a pricipal $G$-bundle and
vice versa. 

Given a principal $G$-bundle, it comes with a group homomorphism
$\rho:G\to\text{Aut}\left(F\right)$ which is a left $G$-action on
$F$, then we have a canonical (right) $G$-action on $E\times F$
\[
\begin{array}{ccc}
G & \to & \text{Aut}\left(E\times F\right)\\
g & \mapsto & \left(p,f\right)\cdot g=\left(p\cdot g,\rho\left(g^{-1}\right)\cdot f\right)
\end{array}
\]
and we will show that the quotient $\left(E\times F\right)/G$ is
a fibre bundle. 
\begin{prop}
$\left(E\times F\right)/G=E\times^{G}F$ is a fibre bundle over $B$
with fibre $F$, 
\[
\begin{array}{cccc}
\pi^{G}: & E\times^{G}F & \to & B\\
 & \left[p,f\right] & \mapsto & \pi\left(p\right)
\end{array}
\]
this fibre bundle $E\times^{G}F$ is called the associated fibre bundle
to $\pi:E\to B$ via the $G$-action on $F$. 
\end{prop}
\begin{proof}
First, we want to show that the projection map is well-defined. Let
$\left(p',f'\right)\in\left[p,f\right]$, then $p'=p\cdot g,f'=g^{-1}f\cdot f$
for some $g\in G$, then $\pi\left(p\cdot g\right)=\pi\left(p\right)$
because $\pi$ is a principal $G$-bundle. 

Second, we need to show that fibres are homeomophic to $F$. Fix $b\in B$
and choose $p_{o}\in\pi^{-1}\left(b\right)$, we have a continuous
map 
\[
\begin{array}{ccc}
F & \to & \left(\pi^{G}\right)^{-1}\left(b\right)\\
f & \mapsto & \left[p_{0},f\right]
\end{array}
\]
with inverse $\left[p,f\right]\mapsto\tau\left(p_{0},p\right)\cdot f$.
Then 
\[
\begin{array}{ccc}
\pi^{-1}\left(b\right)\times F & \to & F\\
\left(p,f\right) & \mapsto & \tau\left(p_{0},p\right)\cdot f
\end{array}
\]
is invariant with respect to the $G$-action, 
\[
\left(p,f\right)\cdot g=\left(p\cdot g,g^{-1}f\right)=\tau\left(p_{0},p\cdot g\right)\cdot g^{-1}\cdot f=\tau\left(p_{0},p\right)\cdot g\cdot g^{-1}\cdot f=\tau\left(p_{0},p\right)f
\]
thus we get the induced map on quotient $\left(\pi^{-1}\left(b\right)\times F\right)/G=\left(\pi^{G}\right)^{-1}\left(b\right)\xrightarrow{\cong}F$. 

Last, we need to show local triviality. Assume without lose of generality,
$E=B\times G$ by local trvialization-section correspondence, $\left(E\times F\right)/G=\left(B\times G\times F\right)/G=\left\{ \left[b,g,f\right],b\in B,g\in G,f\in F\right\} $.
We have isomoprhisms of bundles 
\[
\begin{array}{ccc}
\left(B\times G\times F\right)/G & \to & B\times F\\
\left[b,g,f\right] & \mapsto & \left(b,g\cdot f\right)
\end{array}
\]
 with inverse map 
\[
\begin{array}{ccc}
B\times F & \to & \left(B\times G\times F\right)/G\\
\left(b,f\right) & \mapsto & \left[b,e,f\right]
\end{array}
\]
\end{proof}
\begin{prop}
Given a fibre bundle $F\hookrightarrow E\xrightarrow{\pi}B$ with
fibre $F$ and choice of structure group $G=\text{Aut}\left(F\right)$,
there is a principal $\text{Aut}\left(F\right)$-bundle $P$ such
that $E=P\times^{G}F$. 
\end{prop}
\begin{proof}
For $b\in B$, let $P_{b}=$ set of $G$-isomorphisms $\phi:F\to\pi^{-1}\left(b\right)$,
then $P_{b}$ has a right $G$-action since for any $g\in G=\text{Aut}\left(F\right)$,
$\phi\circ g\in P_{b}$. This is a free and transitive action because
if $\phi,\phi':F\to\pi^{-1}\left(b\right)$ are two $G$-isomorphisms,
then $\phi^{-1}\circ\phi'\in\text{Aut}\left(F\right)=G$. The bundle
is defined as $P={\displaystyle \bigsqcup_{b\in B}}P_{b}\to B$. 
\begin{description}
\item [{Case1:}] If $E=B\times F$, $\pi^{-1}\left(b\right)=\left\{ b\right\} \times F$,
so we have a canonical isomorphism $P_{b}\cong G$, then $P={\displaystyle \bigsqcup_{b\in B}}P_{b}={\displaystyle \bigsqcup_{b\in B}}\left\{ b\right\} \times G=B\times G$. 
\item [{Case2:}] do the above construction for local trivialization. 
\end{description}
Then it remains to show that $P\times^{G}F\cong E$. Note that any
element in $P\times^{G}F$ is an equivalence class $\left[b,\phi,f\right]$
where $b\in B,\phi:F\to\pi^{-1}\left(b\right),f\in F$, then we have
a map
\[
\begin{array}{ccc}
P\times^{G}F & \to & E\\
\left[b,\phi,f\right] & \mapsto & \phi\left(f\right)
\end{array}
\]
this is well-defined because $\left[b,\phi\cdot g,g^{-1}f\right]\mapsto\phi\left(gg^{-1}f\right)=\phi\left(f\right)$.
And since $\phi$ is an isomorphism, this map is a piecewise isomorphism,
with inverse given by $p\mapsto\left[\pi\left(p\right),\phi_{0},\tau\left(p,\phi_{0}\left(f_{0}\right)\right)f_{0}\right]$
where $\phi_{0}:F\to\pi^{-1}\left(b\right)$ and $f_{0}\in F$ are
fixed. 
\end{proof}
\begin{rem}
We've shown that for any principal $G$-bundle $\pi:E\to B$, there
is an associated fibre bundle $\pi^{G}:E\times^{G}F\to B$, and for
each fibre bundle $F\hookrightarrow E\xrightarrow{\pi}B$ with fibre
$F$ and choice of structure group $G=\text{Aut}\left(F\right)$,
there is a principal $G$-bundle whose associated fibre bundle is
isomorphic to the one given. In particular, trivial principal $G$-bundles
corresponds to trivial fibre bundles. 
\end{rem}
\begin{example}
(Vector bundles as principal bundles) If $F=V$ is a vector space
and $\rho:G\to GL\left(V\right)$ is a linear $G$-action (e.g. representation)
on $V$, the $P\times^{G}V$ is the associated fibre bundle. Conversely,
each real/complex vector bundle $\pi:E\to B$ is associated to a principal
$GL\left(n,\mathbf{R}\right)/GL\left(n,\mathbf{C}\right)$-bundle.
By chossing an inner product on $F$, we can take a principal $O\left(n\right)/U\left(n\right)$
bundle. 
\end{example}
%
\begin{example}
(Associated self-principal bundle) Let $\rho:G\to\text{Aut}\left(G\right)$
be an adjoint representation of $G$ on itself, this does not preserves
the group structure of $G$ but it commutes with right multiplication
and preserves the $G$-torsor structure. Given a principal $G$-bundle
$G\hookrightarrow P\xrightarrow{\pi}B$, what is $P\times_{\rho}^{G}G$?
This is also a principal $G$-bundle $\left[p,g\right]g'=\left[p,gg'\right]$
and is isomorphic to $P$. Consider the map 
\[
\begin{array}{ccc}
P\times_{\rho}^{G}G & \to & P\\
\left[p,g\right] & \mapsto & p\cdot g
\end{array}
\]
 This map is well-defined $\left[p\cdot h,h^{-1}g\right]\mapsto p\cdot hh^{-1}g=p\cdot g$,
with inverse $p\mapsto\left[p.e\right]$. 

(Variant) Let $\phi\in\text{Aut}\left(G\right)$ be an automorphism,
consider the left action $g\cdot h=\phi\left(g\right)h$, what is
$P\times_{\rho}^{G}G$? This is already a principal $G$-bundle. We
have two cases. 
\begin{enumerate}
\item If $\phi\left(g\right)=\gamma g\gamma^{-1}$ for some $\gamma\in G$
is an inner automorphism, then we have 
\[
P\times_{\phi}^{G}G\xrightarrow{\cong}P,\left[p,g\right]\mapsto p\gamma g.
\]
This is well-defined $\left[p\cdot h,\gamma h^{-1}\gamma^{-1}g\right]\mapsto ph\gamma\gamma^{-1}h^{-1}\gamma g=p\gamma g$,
with inverse 
\[
P\times_{\phi}^{G}G\xleftarrow{\cong}P,\left[p\cdot\gamma^{-1},e\right]\mapsfrom p.
\]
\item If $\phi$ is not an inner product, $P\times_{\rho}^{G}G\not\cong P$. 
\end{enumerate}
%
(General) Given a group homomorphism $\phi:G\to H$, we have a left
$G$-action on $H$, $g\cdot h=\phi\left(g\right)h$, so we can form
a principal $H$-bundle $P\times_{\phi}^{G}H$. 
\end{example}
%

\subsection{Sections of Associated Bundles}
\begin{description}
\item [{Key.}] relate to equivariant functions on $P$. 
\end{description}
If $F$ has a left $G$-action, we say $f:P\to F$ is \emph{equivariant}
if $f\left(p\cdot g\right)=g^{-1}f\left(p\right),g\in G$, denote
the set of all equivariant maps $P\to F$ by $\mathbf{Map}\left(P,F\right)^{G}$
.
\begin{prop}
Let $\pi:P\to B$ be a principal $G$-bundle, and $F$ has a left
$G$-action. Let $E=P\times^{G}F$. There is a bijection $\Gamma\left(U,E\right)=\mathbf{Map}\left(\pi^{-1}\left(U\right),F\right)^{G}$. 
\end{prop}
\begin{proof}
We use the local section/trivialization correspondence to identify
$\pi^{-1}\left(U\right)\cong U\times F$. 

Given an equivariant map $\tilde{s}:\pi^{-1}\left(U\right)\to F$,
define a section $s:U\to E$ by 
\[
b\mapsto\left[p,\tilde{s}\left(p\right)\right],p\in\pi^{-1}\left(b\right).
\]
By equivariance, $\left[p\cdot g,\tilde{s}\left(p\cdot g\right)\right]=\left[p\cdot g,g^{-1}\tilde{s}\left(p\right)\right]=\left[p,\tilde{s}\left(p\right)\right]$,
so we obtain a local section. 

Conversely, given $s\in\Gamma\left(U,E\right)$, suppose $s\left(\pi\left(p\right)\right)=\left[p,f\right]$,
then we can define 
\[
\begin{array}{cccc}
\tilde{s}: & \pi^{-1}\left(U\right) & \to & F\\
 & p & \mapsto & f
\end{array}
\]
note $\tilde{s}\left(p\cdot g\right)=g^{-1}f$ because $s\left(\pi\left(p\cdot g\right)\right)=s\left(\pi\left(p\right)\right)=\left[p,f\right]=\left[p\cdot g,g^{-1}f\right]$. 
\end{proof}
\begin{prop}
\label{prop-hom-sec} Fix $G$, let $\pi:P\to B$ be a principal $G$-bundle.
There is a bijection morphisms $\left(\phi:P\to Q,\bar{\phi}:B\to B'\right)$
of principal bundles and sections of $P\times^{G}Q$ where $Q$ is
a left $G$-space with action $g\cdot q=qg^{-1}$. 
\end{prop}
\begin{proof}
Any principal bundle morphism $\phi:P\to Q$ is equivalent to a $G$-equivariant
map $\phi:P\to Q$, so we have 
\[
\mathbf{Mor}\left(P,Q\right)=\mathbf{Map}\left(P,Q\right)^{G}\cong\Gamma\left(B,P\times^{G}Q\right).
\]
\end{proof}

\subsection{Homotopy Classes of Principal $G$-bundles}

Recall we have $\mathbf{Prin}_{G}\left(X\right)=\left[X,BG\right]$
obtained by pulling back the universal bundle 
\[
\xymatrix{P\cong f^{*}\left(EG\right)\ar[r]\ar[d]_{\pi} & EG\ar[d]\\
X\ar[r]^{f} & BG
}
\]

\begin{prop}
If $\pi:P\to B'$ is a principal $G$-bundle and if $f_{0},f_{1}:B\to B'$
are homotopic, then $f_{0}^{*}\left(P\right)=f_{1}^{*}\left(P\right)$
as $G$-bundles over $B$. 
\end{prop}
\begin{proof}
Consider the pullback by homotopy $f_{t}:B\times I\to B'$, it suffices
to show that for any principal $G$-bundle $Q\to B\times I$, the
restrictions $Q_{0}\coloneqq Q|_{B\times\left\{ 0\right\} }\to B\times\left\{ 0\right\} \cong B$
and $Q_{1}:Q|_{B\times\left\{ 1\right\} }\to B\times\left\{ 1\right\} \cong B$
are isomorphisms and so we can produce an isomorphism $Q\cong Q_{0}\times I$
of bundles over $B\times I,$then restricts to $B\times\left\{ 1\right\} $
to get an isomorphism $Q_{1}\cong Q_{0}$. 

We want to show that $Q$ and $Q_{0}\times I$ are isomorphic as bundles
over $B\times I$. Assuming that $Q\to B\times I$ is a principal
$G$-bundle, by Proposition \ref{prop-id-lift} it's enough to give
a morphsim $Q\to Q_{0}\times I$ lying over $\text{Id}:B\times I\to B\times I$.
Equivalently, we need to find a section of $Q\times^{G}\left(Q_{0}\times I\right)\to B\times I$.
Note $Q\times^{G}\left(Q_{0}\times I\right)$ has a section over $B\times\left\{ 0\right\} $
because $Q|_{B\times\left\{ 0\right\} }$ and $Q_{0}\times I|_{B\times\left\{ 0\right\} }\cong Q_{0}$
are isomorphic.  To extend this section, we use the homotopy lifting
property. Since $Q\times^{G}\left(Q_{0}\times I\right)\to B\times I$
is a fibration, we have a lift in the following commuting diagram.
\[
\xymatrix{B\times\left\{ 0\right\} \ar[r]\ar[d] & Q\times^{G}\left(Q_{0}\times I\right)\ar[d]\\
B\times I\ar[r]^{\text{Id}}\ar@{..>}[ur]^{\exists} & B\times I
}
\]
\end{proof}
%

\subsection{Classifying Space $BG$}

Note that we have a functor
\[
\begin{array}{ccc}
\mathbf{Space} & \longrightarrow & \mathbf{Set}\\
B & \longmapsto & \mathbf{Prin}_{G}\left(B\right)=\mathbf{Bun}_{G}\left(G\right)\\
\left(A\xrightarrow{f}B\right) & \longmapsto & \left(\begin{array}{ccc}
\mathbf{Prin}_{G}\left(B\right) & \to & \mathbf{Prin}_{G}\left(A\right)\\
P & \mapsto & f^{*}P
\end{array}\right)
\end{array}
\]
 Since homotopic maps gives isomorphic principal $G$-bundles, we
can look at $\mathbf{Prin}_{G}:\mathbf{hTop}\to\mathbf{Set}$ where
in $\mathbf{hTop}$, morphisms are homotopy classes of maps in $\mathbf{Space}$.
We wang to show this contravariant functor is representable. 
\begin{defn}
A principal $G$-bundle $EG\to BG$ is \emph{universal} if the total
space $EG$ is contractible. 
\end{defn}
There are many ways to realize the classifying space. 
\begin{itemize}
\item Moduli space. 
\[
\left[X,BG\right]=\mathbf{Prin}_{G}\left(X\right)=\mathbf{Bun}_{G}\left(X\right).
\]
If the moduli functor is representable, it is represneted by $BG$. 
\item Infinitely dimensional realization. For example, line bundles are
represented by $\mathbb{C}P^{\infty}$ , vector bundles are represented
by Grassmannians. 
\item Simplicial realization. 
\end{itemize}
\begin{lem}
If $\left(B,A\right)$ is a CW pair and $F$ is a space such that
$\pi_{k}\left(F\right)=0$ if $B\backslash A$ has cells of dimension
$k+1$, then every map $f:A\to F$ extends to a map $\tilde{f}:B\to F$
such that $\tilde{f}|_{A}=f$. 
\end{lem}
\begin{proof}
Induction on $k$. We can assume that without lose of generality that
$f$ has been extended to the $k$-skeleton $B^{k}$ of the pair $\left(B,A\right)$.
For each $\left(k+1\right)$-cell $e^{k+1}\in B$ with attaching map
$\phi:\partial I^{k+1}\to B^{k}$, by assumption, the composition
$f\circ\phi:\partial I^{k+1}\to B^{k}\to F$ is null-homotopic, hence
we can extend to $B^{k}\cup e^{k+1}$. Do the same for each $\left(k+1\right)$-cell. 
\end{proof}
\begin{cor}
\label{cor-ext}If $\left(B,A\right)$ is a CW pair and $F\hookrightarrow E\xrightarrow{\pi}B$
is a fibre bundle, and if $\pi_{k}\left(F\right)=0$ for all $k$
where $B\backslash A$ has $\left(k+1\right)$-cells, then every section
over $A$ can be extended to global section. In particular, if $\left(B,A\right)=\left(B,\phi\right)$,
then $E\to B$ admits a global section if and only if $F$ is $\dim\left(B\right)$-connected. 
\end{cor}
\begin{proof}
If $E=B\times F$ is trivial, then a section is equivalent to a map
$B\to F$, then we are done by previou lemma. 

In general, refine CW structure on $B$ to open coverings corresponds
to local trvializations. Namely, we proceed by induction on $k$,
assuming that a section $s$ has been extended to $k$-skeleton, so
$s\in\Gamma\left(B^{k},E\right)$. Let $e^{k+1}\subseteq B$ be a
$\left(k+1\right)$-cell. It may not be in the domain of one trivialization,
but then subdivide $e^{k+1}\cong I^{k+1}$ into small cubes, then
we can reduce to $e^{k+1}\subseteq U_{\alpha}$ where $\left\{ U_{\alpha}\right\} $
is an open covering of $B$ and $\pi^{-1}\left(U_{\alpha}\right)\cong U_{\alpha}\times F$
. 
\end{proof}
\begin{thm}
Let $EG\to BG$ be a universal bundle, i.e. $EG$ is weakly contractible.
Then for any CW complex $B$, we have a bijection $\left[B,BG\right]\longleftrightarrow\mathbf{Prin}_{G}\left(B\right)$
where 
\[
\begin{array}{ccc}
\left[-,BG\right] & \to & \mathbf{Prin}_{G}\left(-\right)\\
\left[f\right] & \mapsto & f^{*}\left(EG\right)
\end{array}
\]
 is an equivalence of functors $\mathbf{hTop}\to\mathbf{Set}$. 
\end{thm}
\begin{proof}
Surjectivity. Let $Q\to B$ be a principal $G$-bundle, then by Corollary
\ref{cor-ext} the associated fibre bundle $Q\times^{G}EG$ has a
global section over $B$ because $EG$ is contractible, thus by Proposition
\ref{prop-hom-sec} we have a morphism of $G$-bundles 
\[
\xymatrix{Q\ar[r]\ar[d] & EG\ar[d]\\
B\ar[r]^{f} & BG
}
\]
Such a moprhism is equivalent to a morphism 
\[
\xymatrix{Q\ar[r]^{\cong}\ar[d] & f^{*}\left(EG\right)\ar[d]\\
B\ar@{=}[r]^{\text{Id}} & B
}
\]
which is an isomorphism. 

Injectivity. Suppose $f_{0},f_{1}:B\to BG$ gives $f_{0}^{*}\left(EG\right)\cong f_{1}^{*}\left(EG\right)$,
we want to show that $f_{0}\sim f_{1}$. Consider the principal $G$-bundle
$P=f_{0}^{*}\left(EG\right)\times I\to B\times I$ such that $p|_{B\times\left\{ 0\right\} }\cong f_{0}^{*}\left(EG\right)\cong p|_{B\times\left\{ 1\right\} }$.
Consider the $G$-bundle morphism 
\[
\xymatrix{P|_{B\times\left\{ 0\right\} }\ar[r]^{\cong}\ar[dr] & f_{0}^{*}\left(EG\right)\ar[r]\ar[d] & EG\ar[d]\\
 & B\ar[r]^{f_{0}} & BG
}
\]
it corresponds to a section $s_{0}\in\Gamma\left(B\times\left\{ 0\right\} ,P\times^{G}EG\right)$.
Similarly, if we look over $B\times\left\{ 1\right\} $, we get a
local section $s_{1}\in\Gamma\left(B\times\left\{ 1\right\} ,P\times^{G}EG\right)$.
So we obtain a section $s_{0}\cup s_{1}\in\Gamma\left(B\times\left\{ 0,1\right\} ,P\times^{G}EG\right)$.
Since $EG$ is contractible, this extends to a global section $s\in\Gamma\left(B\times I,P\times^{G}EG\right)$,
which corresponds to a principal bundle morphism 
\[
\xymatrix{P\ar[r]\ar[d] & EG\ar[d]\\
B\times I\ar[r]^{h} & BG
}
\]
and induces a map $h:B\times I\to BG$ which is a homotopy from $f_{0}$
to $f_{1}$. 
\end{proof}
\begin{fact}
\cite{key-1} Given a topological group $G$, there exists a universal
principal bundle $EG\to BG$. 
\end{fact}
\begin{cor}
We can take $BG$ to have the structure of a CW complex when $BG$
admits a simplicial realization and such a $BG$ is unique up to homotopy. 
\end{cor}
\begin{rem}
We really should think of the equivalence as equivalences between
bifunctors $\left[-,B\left(\bullet\right)\right]\simeq\mathbf{Prin}_{\bullet}\left(-\right)$. 
\end{rem}

\subsection{Classifying Space Functor $B$ }

We want to show that the classifying space construction 
\[
\begin{array}{cccc}
B: & \mathbf{Gp} & \longrightarrow & \mathbf{hCW}\\
 & G & \longmapsto & BG
\end{array}
\]
is functorial (up to homotopy). Note there is no canonical construction
of $BG$. 
\begin{prop}
For each homomorphism $\phi\in\text{Hom}_{\mathbf{Gr}}\left(G,H\right)$,
there exists a homotopy class $B\phi\in\left[BG,BH\right]$ such that
$B\left(\phi\circ\psi\right)=B\left(\phi\right)\circ B\left(\psi\right)$,
$B\left(\text{Id}\right)=\text{Id}$, and $B\left(G\times H\right)=BG\times BH$,
i.e. $B$ respects products. 
\end{prop}
\begin{proof}
We have a universal bundle $EG\to BG$. Consider the associated bundle
$EG\times_{\phi}^{G}H$ which is a principal $H$-bundle over $BG$,
thus it correpsonds to a map $B\phi\in\left[BG,BH\right]$. We get
the composition property because 
\[
\left(EG\times_{\phi}^{G}H\right)\times_{\psi}^{H}K\cong EG\times_{\psi\phi}^{H}K.
\]
Furthermore, $EG\times^{G}G\cong EG$, so $B\left(\text{Id}\right)=\text{Id}$. 

For products, note that $EG\times EH$ is a contractible space with
a $G\times H$-action, 
\[
\left(EG\times EH\right)/G\times H\cong BG\times BH.
\]
\end{proof}
\begin{prop}
Let $i:H\hookrightarrow G$ be a subgroup such that $G\to G/H$ is
a principal $H$-bundle (e.g. $H$ is a closed subgroup such that
$G/H$ is a homogeneous space). Then we can view $Bi:BH\to BG$ as
a fibre bundle with fibre $G/H$. 
\end{prop}
\begin{proof}
Under our hypothesis, $EG$ is a contractible space with a free (right)
$H$-action, so $EG/H$ is a classifying space of $H$. Furthermore,
$EG$ is universal bundle of $H$. 

We have $EG/H\cong\left(EG\times^{G}G\right)/H\cong EG\times^{G}\left(G/H\right)$,
thus we have a morphism of principal $G$-bundles 
\[
\xymatrix{EG\ar[r]^{\text{Id}}\ar[d] & EG\ar[d]\\
BH=EG/H\ar[r] & BG=EG/G
}
\]
so the induced map can be identified with $BH\cong EG\times^{G}\left(G/H\right)\to BG$,
this is a $G/H$-fibre bundle. 
\end{proof}
\begin{prop}
If $\phi:G\to G$ is an inner automorphism, then $B\phi:BG\to BG$
is homotopic to $\text{Id}$. 
\end{prop}
\begin{proof}
It follows from the fact that $EG\times^{G}G=EG$. 
\end{proof}

\subsection{Grassmannians and Classifying Spaces}

Let $V$ be a finite-dimensional vector space over a field $F$, say
$\mathbf{R}$, or $\mathbf{C}$. Let $0\leq k\leq\dim V$. 
\begin{defn}
The Grassmannians are defiend as $\text{Gr}_{k}\left(V\right)=\left\{ W\subseteq V|\dim W=k\right\} ,\text{Gr}^{k}\left(V\right)=\left\{ W\subseteq V|\dim W+k=\dim V\right\} $. 
\end{defn}
%
Grassmannians can be viewed as smooth manifolds, or algebraic varieties. 

Over $\text{Gr}_{k}\left(V\right)$ we have a tautological exact sequence
$0\to S\to V\to Q\to0$ of vector bundles. 
\begin{itemize}
\item fibre of universal subbundle $S_{W}\cong W,W\in\text{Gr}_{k}\left(V\right)$.
\item fibre of universal quotient bundle $Q_{W}\cong V/W$ .
\item $V_{W}\cong V$. 
\end{itemize}
\begin{fact}
Any vector bundle of rank $k$ $\pi:E\to M$ for $M$ compact can
be expressed as pullback of the universal quotient bundle/ subbundle
over $\text{Gr}^{k}\left(V\right)$/$\text{Gr}_{k}\left(V\right)$,
i.e. there exists finite dimensional vector space $V$ and smooth
maps 
\[
\xymatrix{E\ar[r]\ar[d] & Q\ar[d] & \ar@{-}[d] & E\ar[r]\ar[d] & Q\ar[d]\\
M\ar[r]^{f} & \text{Gr}^{k}\left(V\right) &  & M\ar[r]^{f} & \text{Gr}_{k}\left(V\right)
}
\]
\end{fact}
\begin{proof}
Since $E\to M$ is locally trivial and $M$ is compact, there exists
finite open cover $\left\{ U_{\alpha}\right\} _{\alpha\in A}$ of
$M$ and basis $s_{1}^{\alpha},\cdots,s_{k}^{\alpha}\in\Gamma\left(U_{\alpha},E\right)$
of local trivializations. Let $\left\{ \rho^{\alpha}\right\} $ be
a partition of unity subordinate to $\left\{ U_{\alpha}\right\} $,
then $\tilde{s}_{i}^{\alpha}=\rho^{\alpha}s_{i}^{\alpha}\in\Gamma\left(M,E\right)$
which vanishes outside $U_{\alpha}$. Define a finite dimensional
vector space $V=\text{Span}_{F}\left\{ \tilde{s}_{i}^{\alpha}\right\} _{\alpha\in A,1\leq i\leq k}$,
the for any $p\in M$, there is a linear map 
\[
\begin{array}{cccc}
\text{ev}_{p}: & V & \to & E_{p}\\
 & \tilde{s}_{i}^{\alpha} & \mapsto & \tilde{s}_{i}^{\alpha}\left(p\right)
\end{array}
\]
which is surjective. Thus $V/\text{Ker}\left(\text{ev}_{p}\right)\cong E_{p}$.
By taking the inverse of this isomorphism, we obtain a map $\tilde{f}:E\to Q$
with 
\[
\xymatrix{E\ar[r]^{\tilde{f}}\ar[d] & Q\ar[d]\\
M\ar[r]^{f} & \text{Gr}^{k}\left(V\right) & p\ar@{|->}[r] & \text{Ker}\left(\text{ev}_{p}\right)
}
\]
\end{proof}
There are natural questions as follows. 
\begin{description}
\item [{Quetsion.}] What if $M$ is not compact? What about all vector
bundles instead of just $k$-dimensional vector bundles? 
\end{description}
It motivates us to study infinite Grassmannians. 

\subsubsection{Infinite Grassmannians}

Given an integer $k\geq1$, consider the sequence of clsed inclusions
\[
\begin{array}{ccccccc}
\mathbf{R}^{q} & \hookrightarrow & \mathbf{R}^{q+1} & \hookrightarrow & \mathbf{R}^{q+2} & \hookrightarrow & \cdots\\
\left(x_{1},x_{2},\cdots\right) & \mapsto & \left(0,x_{1},x_{2},\cdots\right) & \mapsto & \left(0,0,x_{1},x_{2},\cdots\right)
\end{array}
\]
 which gives corresponding closed inclusions 
\[
\begin{array}{ccccccc}
\text{Gr}_{k}\left(\mathbf{R}^{q}\right) & \hookrightarrow & \text{Gr}_{k}\left(\mathbf{R}^{q+1}\right) & \hookrightarrow & \text{Gr}_{k}\left(\mathbf{R}^{q+2}\right) & \hookrightarrow & \cdots\\
W & \mapsto & 0\oplus W & \mapsto & 0\oplus0\oplus W
\end{array}
\]
and 
\[
\begin{array}{ccccccc}
\text{Gr}^{k}\left(\mathbf{R}^{q}\right) & \hookrightarrow & \text{Gr}^{k}\left(\mathbf{R}^{q+1}\right) & \hookrightarrow & \text{Gr}^{k}\left(\mathbf{R}^{q+2}\right) & \hookrightarrow & \cdots\\
W & \mapsto & \mathbf{R}\oplus W & \mapsto & \mathbf{R}^{2}\oplus W
\end{array}
\]

We can get a corresponding sequence of pullback maps of universal
quotient bundle 
\[
\xymatrix{Q_{q}\ar[r]\ar[d] & Q_{q+1}\ar[r]\ar[d] & Q_{q+2}\ar[r]\ar[d] & \cdots\\
\text{Gr}_{k}\left(\mathbf{R}^{q}\right)\ar[r] & \text{Gr}_{k}\left(\mathbf{R}^{q+1}\right)\ar[r] & \text{Gr}_{k}\left(\mathbf{R}^{q+2}\right)\ar[r] & \cdots
}
\]
By taking colimit of this diagram we get a vector bundle $\pi:Q^{\text{univ}}\to B_{k}$.
$B_{k}$ is a topological space, but not a manifold. 
\begin{description}
\item [{Question.}] If we want to make $\pi:Q^{\text{univ}}\to B_{k}$
continuous, what is the best topology to impose here? 
\end{description}
\begin{fact}
If $X$ is metrizable and $\pi:E\to X$ is a vector bundle over $X$,
then there exists a classfying diagram 
\[
\xymatrix{E\ar[r]^{\tilde{f}}\ar[d]_{\pi} & Q^{\text{univ}}\ar[d]\\
X\ar[r]^{f} & B_{k}
}
\]
\end{fact}
\begin{description}
\item [{Question.}] What does it mean for $Q^{\text{univ}}$ to be contractible? 
\end{description}
%
One model for $B_{k}$ is Hilbert manifold, a manifold modeled on
Hilbert spaces. Thus it is a separable Hausdorff space in which each
point has a neighbourhood homeomorphic to an infinite dimensional
Hilbert space. 

Let $\mathcal{H}$ be a seperable Hilbert space over $\mathbf{R}$
or $\mathbf{C}$. The topological on $\mathcal{H}$ is induced by
inner product $\left\langle -,-\right\rangle $ on $\mathcal{H}$.
Given an integer $k\geq1$, $\text{Gr}_{k}\left(\mathcal{H}\right)=\left\{ W\subseteq H|\dim W=k\right\} $.
Usually $\text{Gr}^{k}\left(\mathcal{H}\right)=\left\{ W\subseteq H|\dim H/W=k\right\} $
is called a Hilbert manifold, or Grassmannian. Since $\mathcal{H}$
is separable, we can pick an orthogonal basis $\left\{ e_{1},\cdots,e_{n},\cdots\right\} $
of $\mathcal{H}$. Define $\mathbf{R}^{q}\subseteq\mathcal{H}$ by
$\mathbf{R}^{q}\cong\text{Span}_{\mathbf{R}}\left\{ e_{1},\cdots,e_{q}\right\} $,
then we have a commutative diagram 
\[
\xymatrix{\cdots\ar[r] & \text{Gr}^{k}\left(\mathbf{R}^{q}\right)\ar[rr]\ar[dr] &  & \text{Gr}^{k}\left(\mathbf{R}^{q+1}\right)\ar[r]\ar[dl] & \cdots\\
 &  & \text{Gr}^{k}\left(\mathcal{H}\right)
}
\]
taking colimit we can get a map $i:B_{k}\to\text{Gr}^{k}\left(\mathcal{H}\right)$. 
\begin{prop}
The map $i:B_{k}\to\text{Gr}^{k}\left(\mathcal{H}\right)$ is a homotopy
equivalence. 
\end{prop}
\begin{fact}
Let $F\hookrightarrow E\xrightarrow{p}B$ be a fibre bundle such that
$F$ is contractible and metrizable topological manifold. If $B$
is metrizable, then $\pi$ admits a section. 
\end{fact}
\begin{cor}
Given the conditions above, and suppose that $F,E,B$ are all homotopic
to CW complexes, then $\pi$ is a homotopy equivalence. 
\end{cor}
\begin{proof}
As $F$ is contractible, the long exact sequence in homotopy groups
shows that $p_{n}:\pi_{n}\left(E\right)\to\pi_{n}\left(B\right)$
is always an isomorphism. The Whitehead theorem (that a weak homotopy
equivalence of CW-complexes is an homotopy equivalence) now shows
that $p$ is a homotopy equivalence. 
\end{proof}
%
Let $\mathcal{H}$ be a Hilbert space, we want to understand how to
get a universal $GL_{k}\left(\mathbf{R}\right)$-bundle over its classifying
space. 

Consider the \emph{Stiefel manifold} $St_{k}\left(\mathcal{H}\right)=\left\{ \left(e_{1},\cdots,e_{k}\right)\in\mathcal{H}^{k}|\left\langle e_{i},e_{j}\right\rangle =\delta_{ij},\forall1\leq i,j\leq k\right\} $
which is the set of all orthonormal $k$-frames in $\mathcal{H}$,
this is an infinite dimensional vector space which is a Hilbert manifold.
We have a projection map 
\[
\begin{array}{cccc}
\pi: & St_{k}\left(\mathcal{H}\right) & \to & Gr_{k}\left(\mathcal{H}\right)\\
 & \left(e_{1},\cdots,e_{n}\right) & \mapsto & \text{Span}\left\{ e_{1},\cdots,e_{n}\right\} 
\end{array}
\]
which is smooth (involves work) and is a principal $GL_{k}\left(\mathbf{R}\right)$-bundle. 
\begin{thm}
\label{Thm-Stiefel} $St_{k}\left(\mathcal{H}\right)$ is contractible. 
\end{thm}
\begin{cor}
$\pi:St_{k}\left(\mathcal{H}\right)\to Gr_{k}\left(\mathcal{H}\right)$
is a universal $GL_{k}\left(\mathbf{R}\right)$-bundle. 
\end{cor}
\begin{rem}
Usually theorem can be proved like this for complete, metrizable,
locally convex vector spaces. Alternatively, we can prove it via an
induction argument for $St_{k}\left(\mathbf{R}^{\infty}\right)$ with
a colimit topology. 
\end{rem}
\pagebreak{}

\part{Classifying Spaces}

\section{Construction of Universal Bundles}

In previous lectures, we have shown that for given topological group
$G$ and topological space $X$, the homotopy classes of principal
$G$-bundles over $X$ can be uniquely determined by a map $f:X\to BG$
from $X$ to the classifying space of $G$, i.e. $\mathbf{Prin}_{G}\left(X\right)=\left[X,BG\right]$
can be obtained by pulling back the universal bundle 
\[
\xymatrix{P\cong f^{*}\left(EG\right)\ar[r]\ar[d]_{\pi} & EG\ar[d]\\
X\ar[r]^{f} & BG
}
\]
We ssen that classifying space characterizes the principal $G$-bundles
on $X$, so it is appealing to know how to construct $BG$ and whether
such construction is canonical. 
\begin{description}
\item [{Question1.}] How to construct a classifying space of $G$, and
in particular, when will this construction be a CW complex? Is this
construction unique (up tp homotopy)?
\item [{Question2.}] For any CW complex $X$, when will it be the classifying
space for some group $G$? Is such group unique? 
\end{description}

\subsection{Mail Result}
\begin{defn}
A topological group $G$ is called a \emph{countable CW-group} if
$G$ is a countable CW complex such that the inverse map $G\to G,g\mapsto g^{-1}$
and the product map $G\times G\to G,\left(f,g\right)\mapsto fg$ are
both cellular (i.e. carry the $k$-skeleton into the $k$-skeleton). 
\end{defn}
\begin{thm}
$ $
\begin{enumerate}
\item Any countable CW-group $G$ has a countable CW complex $X$ as classifying
space. If there is another CW complex $X'$ which is a classifying
space for $G$, then $X,X'$ are of the same homotopy type. 
\item Any countable connected CW complex $X$ is the classifying space for
some countable CW-group $G$. If there is another countable CW-group
$G'$ with the same classifying space $X$, then $G'=G$. 
\end{enumerate}
\end{thm}

\subsection{Topology of Joins}
\begin{defn}
The \emph{join} $A_{1}\circ\cdots\circ A_{n}$ of $n$ topological
spaces is defined with the following data: 
\begin{itemize}
\item $n$ real numbers $t_{i},1\leq i\leq n$ satisfying $t_{i}\geq0,\sum t_{i}=1$,
and 
\item a point $a_{i}\in A_{i}$ for each $i$ if $t_{i}\neq0$. 
\end{itemize}
Such a point in $A_{1}\circ\cdots\circ A_{n}$ will be denoted by
$t_{1}a_{1}\oplus\cdots\oplus t_{n}a_{n}$ where the element $a_{i}$
may be chosen randomly or omitted whenever the corresponding $t_{i}=0$. 
\end{defn}
%
\begin{defn}
The \emph{strong topology} in $A_{1}\circ\cdots\circ A_{n}$ is the
initial topology with respect to the coordinate functions 
\[
\begin{array}{ccc}
t_{i}:A_{1}\circ\cdots\circ A_{n}\to\left[0,1\right] & \text{and} & a_{i}:t_{i}^{-1}(0,1]\to a_{i}\end{array},
\]
i.e. any $f:X\to A_{1}\circ\cdots\circ A_{n}$ is continuous if and
only if the compositions $\left\{ t_{i}f:X\to\left[0,1\right],a_{i}f:X_{i}\to A_{i}|X_{i}=f^{-1}\left(t_{i}^{-1}(0,1]\right)\right\} $
are continuous. 
\end{defn}
\begin{rem}
The join of infinitely many topological spaces in the strong topology
can be defined in the same manner, with the restriction that all but
a finite number of $t_{i}$ should vanish. 
\end{rem}
%
This operation of finite/infinite joins is associative and commutative
(up to isomorphism). 
\begin{prop}
Let $1\leq i\leq n$. There is a canonical homeomorphism 
\[
\alpha:\left(A_{1}\circ\cdots\circ A_{r}\right)\circ\left(A_{r+1}\circ\cdots\circ A_{n}\right)\to A_{1}\circ\cdots\circ A_{n}.
\]
\end{prop}
\begin{proof}
The map is given by 
\[
\begin{array}{cccc}
\alpha: & \left(A_{1}\circ\cdots\circ A_{r}\right)\circ\left(A_{r+1}\circ\cdots\circ A_{n}\right) & \to & A_{1}\circ\cdots\circ A_{n}\\
 & s\left(t_{1}a_{1}\oplus\cdots\oplus t_{r}a_{r}\right)\oplus\left(1-s\right)\left(t_{r+1}a_{r+1}\right) & \mapsto & st_{1}a_{1}\oplus\cdots st_{r}a_{r}\oplus\left(1-s\right)t_{r+1}\oplus\cdots\oplus\left(1-s\right)t_{n}
\end{array}
\]
with inverse 
\[
\begin{array}{cccc}
\beta: & A_{1}\circ\cdots\circ A_{n} & \to & \left(A_{1}\circ\cdots\circ A_{r}\right)\circ\left(A_{r+1}\circ\cdots\circ A_{n}\right)\\
 & t_{1}a_{1}\oplus\cdots\oplus t_{n}a_{n} & \mapsto & \left({\displaystyle \sum_{i=1}^{r}t_{i}}\right)\left(\frac{t_{1}}{{\displaystyle \sum_{i=1}^{r}t_{i}}}a_{1}\oplus\cdots\oplus\frac{t_{r}}{{\displaystyle \sum_{i=1}^{r}t_{i}}}a_{r}\right)\oplus{\displaystyle \sum_{i=r}^{n}}t_{i}\left(\frac{t_{r+1}}{{\displaystyle \sum_{i=1}^{r}t_{i}}}a_{r+1}\oplus\cdots\oplus\frac{t_{n}}{{\displaystyle \sum_{i=r}^{n}}t_{i}}a_{n}\right)
\end{array}
\]
which is well-defined because when ${\displaystyle \sum_{i=1}^{r}}t_{i}=0$,
the image is of the form $0\oplus\cdots\oplus0\oplus t_{r+1}a_{r+1}\oplus\cdots\oplus t_{n}a_{n}$,
and ${\displaystyle \sum_{i=1}^{r}}t_{i}=1$, the image is of the
form $t_{1}a_{1}\oplus\cdots\oplus t_{r}a_{r}\oplus0\oplus\cdots\oplus0$. 
\end{proof}
A subbasis of $A_{1}\circ\cdots\circ A_{n}$ is given by two types
of sets: 
\begin{enumerate}
\item $\left\{ t_{1}a_{1}\oplus\cdots\oplus t_{n}a_{n}|\alpha<t_{i}<\beta\right\} $,
the set of all $t_{1}a_{1}\oplus\cdots\oplus t_{n}a_{n}$ such that
$\alpha<t_{i}<\beta$. 
\item $\left\{ t_{1}a_{1}\oplus\cdots\oplus t_{n}a_{n}|t_{i}\neq0,a_{i}\in U\right\} $,
the set of all $t_{1}a_{1}\oplus\cdots\oplus t_{n}a_{n}$ suhc that
$t_{i}\neq0$ and $a_{i}\in U$ where $U$ is an open set in $A_{i}$. 
\end{enumerate}
\begin{lem}
\label{Lem1} The reduced singular homology groups of the join $A\circ B$
with coefficient in a principal ideal domain are given by 
\[
\tilde{H}_{r+1}\left(A\circ B\right)\cong\sum_{i+j=r}\tilde{H}_{i}\left(A\right)\otimes\tilde{H}_{j}\left(B\right)+\sum_{i+j=r-1}\text{Tor}\left(\tilde{H}_{i}\left(A\right),\tilde{H}_{j}\left(B\right)\right).
\]
\end{lem}
\begin{proof}
We have 
\[
A\circ B\cong\left(C\left(A\right)\times B\right)\bigcup_{A\times B}\left(A\times C\left(B\right)\right)
\]
so the triad $\left(A\circ B,C\left(A\right)\times B,A\times C\left(B\right)\right)$
gives us the reduced Mayer-Vietoris sequence 
\[
{\displaystyle \cdots\to\tilde{H}_{n+1}\left(A\circ B\right)\xrightarrow{\partial_{*}}\tilde{H}_{n}\left(A\times B\right)\xrightarrow{(i_{*},j_{*})}\tilde{H}_{n}\left(A\right)\oplus\tilde{H}_{n}\left(B\right)\xrightarrow{k_{*}-l_{*}}\tilde{H}_{n}\left(A\circ B\right)\xrightarrow{\partial_{*}}\tilde{H}_{n-1}\left(A\times B\right)\to\cdots\to\tilde{H}_{0}\left(A\right)\oplus\tilde{H}_{0}\left(B\right)\xrightarrow{k_{*}-l_{*}}\tilde{H}_{0}\left(A\circ B\right)\to0.}
\]
since the inclusion $k:A\times C\left(B\right)\to A\circ B$ and $l:C\left(A\right)\times B\to A\circ B$
are null-homotopic, this long exact sequence splits into short exact
sequences 
\[
0\to\tilde{H}_{n+1}\left(A\circ B\right)\xrightarrow{\partial_{*}}\tilde{H}_{n}\left(A\times B\right)\xrightarrow{(i_{*},j_{*})}\tilde{H}_{n}\left(A\right)\oplus\tilde{H}_{n}\left(B\right)\to0.
\]
Since the singular complex $S_{*}\left(A\times B\right)$ and $S_{*}\left(A\right)\otimes S_{*}\left(B\right)$
has the same homology group (theorem of Eilenberg and Zilber), and
the later is given by the K�nneth theorem (over PID),

\[
{\displaystyle 0\to\bigoplus_{i+j=k}H_{i}\left(S_{*}\left(A\right)\right)\otimes_{R}H_{j}\left(S_{*}\left(B\right)\right)\to H_{k}\left(S_{*}\left(A\right)\otimes S_{*}\left(B\right)\right)\to\bigoplus_{i+j=k-1}\mathrm{Tor}_{1}^{R}\left(H_{i}\left(S_{*}\left(A\right)\right),H_{j}\left(S_{*}\left(B\right)\right)\right)\to0.}
\]
By computing $\ker\left(i_{*},j_{*}\right)\cong\tilde{H}_{n+1}\left(A\circ B\right)$
we get our desired result. 
\end{proof}
\begin{lem}
\label{Lem2}If $B$ is path connected and $A$ is nonempty, then
$A\circ B$ is simply connected. 
\end{lem}
\begin{proof}
Note any $\left[f\right]\in\pi_{1}\left(A\circ B\right)$ is represented
by a map 
\[
\begin{array}{cccc}
f: & S^{1} & \to & A\circ B\\
 & s & \mapsto & t\left(s\right)a\left(s\right)\oplus\left(1-t\left(s\right)\right)b\left(s\right)
\end{array}
\]
where $t:S^{1}\to\left[0,1\right]$, and $a\left(s\right)$ is defined
when $t\left(s\right)\neq0$ and $b\left(s\right)$ is defined when
$t\left(s\right)\neq1$. We will show that $f$ is null-homotopic. 

Since $B$ is path-connected, we can define a map $p:S^{1}\to B$
such that $p\left(s\right)=b\left(s\right)$ whenever $t\left(s\right)\leq\frac{1}{2}$.
Fix $a_{0}\in A$. Define a map 
\[
\begin{array}{cccc}
t: & S^{1}\times\left[0,2\right] & \to & \left[0,1\right]\\
 & \left(s,t\right) & \mapsto & \begin{cases}
\min\left\{ 1,\left(1+u\right)t\left(s\right)\right\}  & 0\leq u\leq1\\
\left(2-u\right)t\left(s,1\right) & 1\leq u\leq2
\end{cases}
\end{array}
\]
and our homotopy is given by 
\[
\begin{array}{cccc}
F: & S^{1}\times\left[0,3\right] & \to & A\circ B\\
 & \left(s,t\right) & \mapsto & \begin{cases}
t\left(s,u\right)a\left(s\right)\oplus\left(1-t\left(s,u\right)\right)b\left(s\right) & 0\leq u\leq1\\
t\left(s,u\right)a\left(s\right)\oplus\left(1-t\left(s,u\right)\right)p\left(s\right) & 1\leq u\leq2\\
\left(u-2\right)a_{0}\oplus\left(3-u\right)p\left(s\right) & 2\leq u\leq3
\end{cases}
\end{array}
\]
where $F\left(s,0\right)=f\left(s\right)$ and $F\left(s,3\right)=\left(a_{0}\oplus0\right)$. 
\end{proof}
We call a nonempty space $\left(-1\right)$-connected. 
\begin{lem}
\label{Lem3} The join of $n+1$ nonempty spaces is always $\left(n-1\right)$-connected.
In fact, if $A_{i}$ is $\left(c_{i}-1\right)$-connected, then $A_{0}\circ\cdots\circ A_{n}$
is $\left(c_{0}+\cdots+c_{n}+n-1\right)$-connected. 
\end{lem}
\begin{proof}
Induction on $n$ by applying Lemma \ref{Lem2}. It suffices to prove
for $n=1$. 

When $c_{0}=c_{1}=0$, there is a path between any point $t_{0}a_{0}\oplus\cdots\oplus t_{n}a_{n}$
with $t_{i}\neq0$ and $x_{i}=0\oplus\cdots\oplus0\oplus a_{i}\oplus0\oplus\cdots\oplus0$
given by 
\[
\begin{array}{cccc}
p: & I & \to & A_{0}\circ\cdots\circ A_{n}\\
 & t & \mapsto & \left(1-t\right)t_{0}a_{0}\oplus\cdots\oplus\left(t_{i}+t-tt_{i}\right)a_{i}\oplus\cdots\oplus\left(1-t\right)t_{n}a_{n}
\end{array}
\]
and each $x_{i}$ is connected to $x_{j}$ for $i\neq j$ via 
\[
\begin{array}{cccc}
p: & I & \to & A_{0}\circ\cdots\circ A_{n}\\
 & t & \mapsto & 0\oplus\cdots\oplus ta_{i}\oplus\cdots\oplus\left(1-t\right)a_{j}\oplus\cdots\oplus0
\end{array}
\]
hence the join of any two nonempty spaces is path-connected, so the
statement holds. 

When $c_{0}>0$ or $c_{1}>0$, Lemma \ref{Lem2} implies $A_{0}\circ A_{1}$
is connected, and Lemma \ref{Lem1} implies $H_{r}\left(A_{0}\circ A_{1}\right)=0,\forall r\leq c_{0}+c_{1}$,
so $A_{0}\circ A_{1}$ is $\left(c_{0}+c_{1}\right)$-connected. 
\end{proof}
%
\begin{cor}
\label{Inf-Join} The join of infinitely many nonempty spaces is always
$\infty$-connected. 
\end{cor}
\begin{proof}
Let $A=A_{0}\circ A_{1}\circ\cdots$ be a join of infinitely many
nonempty spaces. We will prove by induction that $\pi_{n}\left(A\right)=0,\forall n\geq0$.

When $n=0$, $A$ is path-connected because $A=A_{0}\circ\left(A_{1}\circ\cdots\right)$
and by Lemma \ref{Lem3}, $A$ is path-connected. 

Assume that $A$ is $\left(n-1\right)$-connected, then $A=\left(A_{0}\circ A_{1}\circ\cdots\circ A_{n+2}\right)\circ A'$
where $A'$ is a join of infinitely many nonempty spaces, so by induction
hypothesis, $A'$ is $n-1$-connected. Note $A_{0}\circ\cdots\circ A_{n+2}$
is $\left(n+1\right)$-connected, so Lemma \ref{Lem3} implies $A$
is $\left(2n\right)$-connected, and thus $A$ is $n$-connected. 
\end{proof}

\subsection{Construction of $BG$ }
\begin{defn}
An \emph{$n$-universal bundle} is a principal fibre bundle such that
the bundle space is $\left(n-1\right)$-connected. 
\end{defn}
Given any topological group $G$, let $E_{n}=G\circ\cdots\circ G$
be the join of $n+1$ copies of $G$ in the strong topology. Define
the right translation 
\[
\begin{array}{cccc}
R: & E_{n}\times G & \to & E_{n}\\
 & \left(t_{0}g_{0}\oplus\cdots\oplus t_{n}g_{n},g\right) & \mapsto & t_{0}\left(g_{0}g\right)\oplus\cdots\oplus t_{n}\left(g_{n}g\right)
\end{array}
\]
Let $X_{n}=E_{n}/G$ and $p:E_{n}\to X_{n}$ be tha canonical projection. 
\begin{thm}
$G$ is the group of the $n$-universal bundle $p:E_{n}\to X_{n}$. 
\end{thm}
\begin{proof}
$E_{n}$ is $\left(n-1\right)$-connected by \ref{Lem3}. The bundle
structure is defined as follows. 

Let $V_{j}=\left\{ p\left(t_{0}g_{0}\oplus\cdots\oplus t_{n}g_{n}\right)\in X_{n}|t_{j}\neq0\right\} $
be an open cover $X_{n}$. Define a map 
\[
\begin{array}{cccc}
\phi_{j}: & V_{j}\times G & \to & p^{-1}\left(V_{j}\right)\\
 & \left(p\left(t_{0}g_{0}\oplus\cdots\oplus t_{n}g_{n}\right),g\right) & \mapsto & t_{0}\left(g_{0}g_{j}^{-1}g\right)\oplus\cdots\oplus t_{n}\left(g_{n}g_{j}^{-1}g\right)
\end{array}
\]
this is well-defined, 
\[
\left(p\left(t_{0}\left(g_{0}h\right)\oplus\cdots\oplus t_{n}\left(g_{n}h\right)\right),g\right)\mapsto t_{0}\left(g_{0}h\right)\left(g_{j}h\right)^{-1}g\oplus\cdots\oplus t_{n}\left(g_{n}h\right)\left(g_{j}h\right)^{-1}g=t_{0}\left(g_{0}g_{j}^{-1}g\right)\oplus\cdots\oplus t_{n}\left(g_{n}g_{j}^{-1}g\right)
\]
with inverse 
\[
\begin{array}{cccc}
\varphi_{j}=\left(p,p_{j}\right): & p^{-1}\left(V_{j}\right) & \to & V_{j}\times G\\
 & t_{0}g_{0}\oplus\cdots\oplus t_{n}g_{n} & \mapsto & \left(p\left(t_{0}g_{0}\oplus\cdots\oplus t_{n}g_{n}\right),g_{j}\right)
\end{array}
\]

The definition of the strong topology in the join shows that $R$
and $\varphi_{j}$ are continuous. Let $e\in G$ be identity, then
\[
\phi_{j}\left(p\left(y\right),e\right)=R\left(e,p_{j}\left(y\right)^{-1}\right)
\]
shows that $\phi_{j}\left(p\left(y\right),e\right)$ is continuous
in $y$, so $\phi_{j}\left(x,e\right)$ is continuous in $x$, and
hence $\phi_{j}\left(x,g\right)=R\left(\phi_{j}\left(x,e\right),g\right)$
is continuous. 

Last, note that the right $G$-action commutes with the coordinate
maps. The $G$-action on $E_{n}$ is free and transitive. 
\end{proof}
\begin{rem}
An $\infty$-universal bundle $p:E_{\infty}\to X_{\infty}$ can be
constructed in the same way.
\end{rem}
%
There is a natural embedding of $E_{n}\hookrightarrow E_{n+1}$ given
by $x\mapsto x\oplus0$, which gives a map betweem principal $G$-bundles
and thus we have the following diagram 
\[
\xymatrix{\cdots\ar[r] & E_{n-1}\ar[r]\ar[d] & E_{n}\ar[r]\ar[d] & E_{n+1}\ar[r]\ar[d] & \cdots\\
\cdots\ar[r] & X_{n-1}\ar[r] & X_{n}\ar[r] & X_{n+1}\ar[r] & \cdots
}
\]
and taking the colimit of this diagram we get $p:E_{\infty}\to X_{\infty}$.
The topology on $E_{\infty}$ is the colimit topology, i.e. the terminal
topology with respect to the upper sequence of inclusions, equivalently,
any $f:E_{\infty}\to Y$ is continuous if and only if the compositions
$f\circ\varphi_{n}:E_{n}\to E_{\infty}\to Y$ is continuous. The topology
on $X_{\infty}$ is the quotient topology. 
\begin{claim}
The space $E_{\infty}$ is weakly contractible. 
\end{claim}
\begin{proof}
This is an immediate result of Corollary \ref{Inf-Join}. 
\end{proof}

\subsection{Countable CW-groups}
\begin{thm}
Every countable CW-group $G$ is the group of an $\infty$-universal
bundle for which the base space $X_{\infty}$ is a countable CW complex. 
\end{thm}
\begin{rem}
The continuity of $R:E_{n}\times G\to E_{n}$ requires the fact that
$E_{n}$ is a countable CW complex.
\end{rem}
\begin{proof}
Let $E_{n}$ be the join of $\left(n+1\right)$ copies of $G$ in
the weak topology, then we can construct $p:E_{n}\to X_{n}$ in the
same way. 

CW structure for $E_{n}$: induction on $n$. $E_{0}=G$ same as $G$.
For $E_{n}=\left(E_{n-1}\circ G\right)$, cells are of the form 
\[
\tau\circ\emptyset,\emptyset\circ\sigma,\left(\tau\circ e\right)\sigma=\left\{ R\left(tx\oplus\left(1-t\right)e,g\right)|x\in\tau,g\in\sigma,t\in\left(0,1\right)\right\} 
\]
 where $\tau$ is a generic cell in $E_{n}$ and $\sigma$ is a generic
cell in $G$, and $\emptyset$ is the emptyset. 

CW structure for $X_{n}$: for $X_{n}=p\left(E_{n-1}\circ G\right)$,
cells are
\begin{enumerate}
\item cells in $X_{n-1}$, and
\item $p\left(\emptyset\circ e\right)$, and 
\item $p\left(\tau\circ e\right)$ where $\tau$ is a cell in $E_{n-1}$. 
\end{enumerate}
Then $X_{n}$ is a countable CW complex with respect to this subdivision. 

The union $E_{\infty},X_{\infty}$ of $\left\{ E_{n},X_{n}\right\} $
can be topologized as CW complexes and the map $p_{\infty}:E_{\infty}\to X_{\infty}$
is the projection map of an $\infty$-universal bundle. 
\end{proof}

\subsection{Universal Bundle of Base Space $X$}
\begin{thm}
For any countable, connected simplicial complex $X$ in the weak topology,
there exists an $\infty$-universal bundle with base space $X$, and
the bundle space and group being CW complexes. 
\end{thm}
The standard construction of contractible fibre space over a space
is based on $X^{I}$ of paths in $X$. Our construction here is based
on a similar space $\tilde{S}$ of simplicial paths in $X$. 
\begin{lem}
{[}HA{]} The product of two countable CW complexes is a CW complex. 
\end{lem}
%
Let $S_{n}=\left\{ \left(x_{n},x_{n-1},\cdots,x_{0}\right)\in X^{n+1}||x_{i},x_{i-1}\text{ lies in a common simplex of }X\right\} \subseteq X^{n+1}$. 

Let $S={\displaystyle \bigsqcup_{n\geq1}}S_{n}$. 

Let $\tilde{S}=S/\sim$ where $\left(x_{n},\cdots,x_{i},\cdots,x_{0}\right)\sim\left(x_{n},\cdots,\hat{x}_{i},\cdots,x_{0}\right)$
if either $x_{i}=x_{i-1}$ or $x_{i+1}=x_{i-1}$. 

Fix $v_{0}\in X$. Take $\tilde{E}=\left\{ \left[x_{n},\cdots,x_{0}\right]\in\tilde{S}|x_{0}=v_{0}\right\} $
and 
\[
\begin{array}{cccc}
p: & \tilde{E} & \to & X\\
 & \left[x_{n},\cdots,x_{1},v_{0}\right] & \mapsto & x_{n}
\end{array}
\]

Take the fibre $\tilde{G}=p^{-1}\left(v_{0}\right)\left\{ \left[x_{n},\cdots,x_{0}\right]\in\tilde{S}|x_{n}=x_{0}=v_{0}\right\} \subseteq\tilde{E}$. 

A product on elements of $\tilde{S}$ is defined as follows: if $\left[x_{n},\cdots,x_{0}\right]$
and $\left[y_{m},\cdots,y_{0}\right]$ satisfy $x_{0}=y_{m}$, then
\[
\left[x_{n},\cdots,x_{0}\right]\cdot\left[y_{m},\cdots,y_{0}\right]=\left[x_{n},\cdots,x_{0},y_{m},\cdots,y_{0}\right]
\]
this multiplication is well-defined and associative. 
\begin{lem}
The space $\tilde{S}$ can be given a structure of a CW-complex, with
subcomplexes $\tilde{E},\tilde{G}$. 
\end{lem}
Let $D\subseteq S$ be the subset of all degenerate $\left(x_{n},\cdots,x_{0}\right)$
in the sense that either $x_{i}=x_{i-1}$ or $x_{i-1}=x_{i+1}$. 
\begin{fact}
Every element of $S$ is equivalent to a unique nondegenerate element
in $S\backslash D$. 
\end{fact}
\begin{proof}
Define a map $\mu:S\to S$ by 
\[
\mu\left(x_{n},\cdots,x_{0}\right)=\begin{cases}
\left(x_{n},\cdots,x_{0}\right) & \text{nondegenerate}\\
\left(x_{n},\cdots,\hat{x}_{i},\cdots,x_{0}\right) & i=\max\left\{ x_{i}=x_{-1}\vee x_{i+1}=x_{i-1}\right\} 
\end{cases}
\]
Let $\nu:S\to S\backslash D$ be the map by applying $\mu$ until
a nondegenerate point is obtained. Then $\nu\left(s\right)=\nu\left(s'\right)$
whenever $s\sim s'$. 
\end{proof}
\begin{fact}
$D$ is a subcomplex of $S$. In other words, if $x\in\sigma$ a simplex
in $S$ is a nondegenerate point, then $\forall y\in\sigma^{o}$,
$y$ is nondegenerate. 
\end{fact}
%
\begin{fact}
If $\sigma$ is a simplex in $S$, then there is a unique map $\sigma\mapsto\sigma'$
where $\sigma'$ is nondegenerate which maps points onto equivalent
points. 
\end{fact}
\begin{lem}
Let $A$ be a simplicial complex and $\Delta_{ij}=\left\{ \left(a_{1},\cdots,a_{n}\right)\in A^{n}|a_{i}=a_{j}\right\} $.
Then $\Delta_{ij}$ is a subcomplex of the first derived complex of
$A^{n}$.  
\end{lem}
\begin{prop}
Given a (not necessarily Hausdorff) space $A$ and a collection of
maps $f:\sigma^{n}\to A$ where each $\sigma^{n}$ is a closed $n$-cell,
and let $e^{n}=f\left(\sigma^{n}\right)^{0}$, $A^{n}=\bigcup_{i\leq n}e^{i}$.
If the following conditions are satisfied, then $A$, together with
cells $\left\{ e^{n}\right\} $ form a CW complex. 
\begin{enumerate}
\item $\left(\sigma^{n}\right)^{0}$ maps 1-1 onto correpsonding $e^{n}$,
every point of $A$ belongs to exactly one $e^{n}$. 
\item $f\left(\partial\sigma^{n}\right)\subseteq A^{n-1}$. 
\item A subset of $A^{n},0\leq n\leq\infty$ is closed if and only if its
inverse image in each cell $\sigma^{i}$ is closed. 
\end{enumerate}
\end{prop}
%
Now we want to show that $\tilde{S}$ has a CW structure. The cells
in $\tilde{S}$ are defined as the image of the interiors of the nondegenerate
simplexes in $S$. 

By applying the proposition, we see that condition (1) and (2) are
satisfied, and for (3), consider any subset $C$ of the $n$-skeleton
of $\tilde{S}$ such that the inverse image in each nondegenerate
simplex of $\dim\leq n$ is closed, we need to show that its inverse
image in any simplex is closed. This is true for any $j$-simplex
$\sigma^{j},j\leq n$, and for $j>n$, the inverse image contains
no interior point. So we only need to consider the boundary and apply
indcution hypothesis. Since $S$ has the weak topology, the inverse
image of $C$ in $S$ is a closed subset. Since $\tilde{S}$ is the
quotient space, $C$ is closed. 
\begin{lem}
The projection map $p:\tilde{E}\to X$ is continuous. 
\end{lem}
\begin{proof}
Let $\eta:S\to\tilde{S}$ be the quotient map and $\bar{p}:\tilde{S}\to X,\left[x_{n},\cdots,x_{0}\right]\mapsto x_{n}$,
since the compostion $\bar{p}\eta:S\to X$ is continuous, $p$ is
continuous. 
\end{proof}
\begin{lem}
The product operation is continuous. 
\end{lem}
\begin{proof}
Let $\Delta$ be the subcomplex of $\tilde{S}\times\tilde{S}$ consisting
of $\left[x_{n},\cdots,x_{0}\right],\left[y_{m},\cdots,y_{0}\right]$
such that $x_{0},y_{m}\in\sigma$ lies in the same simplex. Need to
show $\Delta\to\tilde{S}$ is continuous, it suffices to verify on
each closed cell of $\Delta$. A cell in $\Delta$ is of the form
$\eta\left(\sigma\right)\times\eta\left(\tau\right)$ with $\sigma,\tau$
nongenenerate, the composition $\sigma\times\tau\to\Delta\to\tilde{S}$
is the same as $\sigma\times\tau\to S\to\tilde{S}$ which is continuous. 
\end{proof}
\begin{rem}
This follows from the fact that any CW complex is locally compact,
and therefore the operations of taking quotients and products commute.
\end{rem}
Every element $\left[x_{n},\cdots,x_{0}\right]$ has an inverse $\left[x_{n},\cdots,x_{0}\right]^{-1}=\left[x_{0},\cdots,x_{n}\right]$
such that $\left[x_{n},\cdots,x_{0}\right]^{-1}\cdot\left[x_{n},\cdots,x_{0}\right]=\left[x_{0},x_{0}\right]$. 
\begin{prop}
$\tilde{G}$ is a topological group with identity $\left[v_{0},v_{0}\right]$
with the above multiplication. 
\end{prop}
The bundle structure is defined as follows. Let $V_{j}$ be the star
neighborhood of the $j$-th vertex $v_{j}\in X$. Fix $e_{j}=\left[v_{j},x_{n-1},\cdots,x_{1},v_{0}\right]\in p^{-1}\left(v_{j}\right)$,
we have coordinate map 
\[
\begin{array}{cccc}
\phi_{j}: & V_{j}\times\tilde{G} & \to & p^{-1}\left(V_{j}\right)\\
 & \left(x,g\right) & \mapsto & \left[x,v_{j}\right]\cdot e_{j}\cdot g
\end{array}
\]
which is continuous with inverse 
\[
\begin{array}{cccc}
\varphi_{j}=\left(p,p_{j}\right): & p^{-1}\left(V_{j}\right) & \to & V_{j}\times\tilde{G}\\
 & e & \mapsto & \left(p\left(e\right),e_{j}^{-1}\cdot\left[v_{j},p\left(e\right)\right]e\right)
\end{array}
\]
which is also continuous. The transition map is 
\[
\begin{array}{cccc}
g_{ij}: & V_{i}\cap V_{j} & \to & \tilde{G}\\
 & x & \mapsto & e_{i}^{-1}\cdot\left[v_{i},x,v_{j}\right]\cdot e_{j}
\end{array}
\]
which satisfies $\varphi_{i}\phi_{j}\left(x,g\right)=g_{ij}\left(x\right)\cdot g$. 

Note that the right $G$-action commutes with the coordinate maps.
Furthermore, the $G$-action on $E_{n}$ is free and transitive. Thus
$\left(\tilde{G}\hookrightarrow\tilde{E}\xrightarrow{p}X,\tilde{G},\left\{ V_{j}\right\} \right)$
is a principal fibre bundle. 
\begin{lem}
$\tilde{E}$ is contractible. 
\end{lem}
\begin{proof}
Let $E_{n}\subset S_{n}$ be the set of sequences $\left(x_{n},\cdots,x_{1},v_{0}\right)$
with endpoint $v_{0}$. $\tilde{E}_{n}$ is the image of $E_{n}$
in $\tilde{S}_{n}$. Let $R_{n}=E_{n}\cap D$ the degenerate sequences. 

We want to show that $E_{n},R_{n}$ are contractible. Then it follows
from the below proposition that $R_{n}$ is a strong deformation retract
of $E_{n}$. 

The map $\eta:E_{n}\to\tilde{E}_{n}$ takes $R_{n}$ onto $\tilde{E}_{n-1}$
and carries $E_{n}\backslash R_{n}$ 1-1 onto $\tilde{E}_{n}\backslash\tilde{E}_{n-1}$
and maps simplexes of $E_{n}$ onto cells of $\tilde{E}_{n}$, then
$\tilde{E}_{n-1}$ is a strong deformation retract of $\tilde{E}_{n}$.
(Let $r:E_{n}\times I\to E_{n}$ be the strong deformation retract
of $E_{n}$ to $R_{n}$, and define $\tilde{\eta}:E_{n}\times I\to\tilde{E}_{n}\times I,\left(e,t\right)\mapsto\left(\eta\left(e\right),t\right)$
then $\tilde{\eta}r\tilde{\eta}^{-1}$ strong deformation retract
of $\tilde{E}_{n}$. ) Thus we have $\tilde{E}_{0}\subset\tilde{E}_{1}\subset\cdots\subset\tilde{E}_{n}\subset\cdots$
where each $\tilde{E}_{n-1}$ is a strong deformation retract of $\tilde{E}_{n}$,
hence $\tilde{E}=\bigcup\tilde{E}_{n}$ is contractible (each contraction
of $\tilde{E}_{n-1}$ can be extended to a contraction of $\tilde{E}_{n}$
by extension theorem). 

To show that $E_{n}$ is contractible, let $T_{n}$ be the linear
graph with vertices $\left\{ \left[0\right],\left[1\right],\cdots,\left[n\right]\right\} $
and edges $\left[0,1\right],\cdots,\left[n-1,n\right]$, then we can
identify $E_{n}$ with the set of maps $\left(T_{n},\left[0\right]\right)\to\left(X,v_{0}\right)$
which carry edges linearly into simplexes. es. A contraction of $T_{n}$
is obtained by first deforming the edge $\left[n-1,n\right]$ into
the vertex $\left[n-1\right]$; then deforming $\left[n-2,n-1\right]$
into $\left[n-2\right]$ etc. This induces a contraction of $E_{n}$.

To show $R_{n}$ is contractible, note that $R_{n}=P_{1}\cup P_{2}\cup\cdots\cup P_{s}\cup Q_{1}\cup\cdots\cup Q_{t}$
where $P_{i}=\left\{ \left(x_{n},\cdots,x_{0}\right)\in E_{n}|x_{i}=x_{i-1}\right\} $
and $Q_{j}=\left\{ \left(x_{n},\cdots,x_{0}\right)\in E_{n}|x_{j+1}=x_{j-1}\right\} \subset E_{n}$
it suffices to show that $R'=P_{1}\cap\cdots\cap Q_{t}$ is contractible.
Let $T'$ be a linear graph obtained by identifying $\left[j_{l}+1,j_{l}\right]$
and $\left[j_{m}-1,j_{m}\right]$ of $T_{n}$ for $1\leq l\leq t$
and identifying all points in the edge $\left[i_{l}-1,i_{l}\right]$
for $1\leq l\leq s$ then $R'$ can be considered as the set of maps
$\left(T',\left[0\right]\right)\to\left(X,v_{0}\right)$. Since $T'$
is a tree and contractible, $R'$ is contractible. Thus $R_{n}$ is
contractible. 
\end{proof}
\begin{prop}
If $A$ is a contractible complex and $B$ is contractible subcomplex,
then $B$ is a strong deformation retract of $A$. 
\end{prop}
%
\begin{thm}
Let $p:\tilde{E}\to X$ be the universal bundle cosntrcuted with group
$\tilde{G}$, then any principal $G$-bundle $E\to X$ is induced
by a continuous map $h:\tilde{G}\to G$. 
\end{thm}
\pagebreak{}

\section{The Bar and Cobar Construction}

\subsection{Simplicial Objects}

\subsubsection{The (Co)simplex Category $\Delta$}
\begin{defn}
$\Delta$ is the simplex category with 
\begin{itemize}
\item objects are the finite totally ordered sets $\left[n\right]=\left\{ 0\leq1\leq2\leq\cdots\leq n\right\} ,n\geq0$,
and 
\item morphisms are order-preserving maps 
\[
\text{Hom}_{\Delta}\left(\left[m\right],\left[n\right]\right)\coloneqq\left\{ f\in\text{Hom}_{\mathbf{Set}}\left(\left[m\right],\left[n\right]\right)|f\left(i\right)\leq f\left(j\right),\forall0\leq i\leq j\leq m\right\} 
\]
\end{itemize}
\end{defn}
The morphisms in $\Delta$ are generated by the following two classes
of morphisms: 
\begin{itemize}
\item the \emph{coface} maps $d^{i}:\left[n-1\right]\hookrightarrow\left[n\right],0\leq i\leq n,n\geq1$
defined by the property that $d^{i}$ is injective and contains no
``$i$'' in its image, i.e. 
\[
d^{i}\left(k\right)=\begin{cases}
k & k<i\\
k-1 & k\geq i
\end{cases}
\]
\item the \emph{codegeneracy} maps $s^{j}:\left[n+1\right]\twoheadrightarrow\left[n\right],0\leq j\leq n,n\geq0$
defined by the property that $s^{j}$ is surjective and takes the
value ``$j$'' twice, i.e. 
\[
s^{j}\left(k\right)=\begin{cases}
k & k\leq i\\
k-1 & k>i
\end{cases}
\]
\end{itemize}
and they satisfies the following cosimplical conditions 
\[
\begin{array}{cclc}
d^{j}d^{i} & = & d^{j+1}d^{i} & i\leq j\\
s^{j}d^{i} & = & d^{i}s^{j+1} & i<j\\
s^{j}d^{i} & = & \text{Id} & i=j,j+1\\
s^{j}d^{i} & = & d^{i-1}s^{j} & i>j+1\\
s^{j}s^{i} & = & s^{i}s^{j+1} & i\leq j
\end{array}
\]

The dual category $\Delta^{op}$ has the presentation $Ob\left(\Delta^{op}\right)=Ob\left(\Delta\right)=\left\{ \left[n\right],n\geq0\right\} $,
and morphisms are generated by the \emph{face} maps $d_{i}:\left[n\right]\to\left[n-1\right]$
and the \emph{degeneracy} maps $s_{j}:\left[n\right]\to\left[n+1\right]$
satisfying simplicial relations. 

\subsubsection{(Co)simplicial Objects}
\begin{defn}
Let $\mathscr{C}$ be a category. 
\begin{enumerate}
\item a\emph{ cosimplicial object} in $\mathcal{C}$ is a functor $X^{*}:\Delta\to\mathscr{C},\left[n\right]\mapsto X^{n}\coloneqq X^{*}\left[n\right]$. 
\item a \emph{simplicial object} in $\mathcal{C}$ is a functor $X_{*}:\Delta^{op}\to\mathscr{C},\left[n\right]\mapsto X_{n}\coloneqq X_{*}\left[n\right]$.
\end{enumerate}
The category of simplicial ans cosimplicial objects in $\mathscr{C}$
are denoted $\mathbf{s}\mathscr{C}=\mathscr{C}^{\Delta^{op}}=\mathbf{Fun}\left(\Delta^{op},\mathscr{C}\right)=\mathscr{C}_{\Delta}$
and $\mathbf{cs}\mathscr{C}=\mathscr{C}^{\Delta}=\mathbf{Fun}\left(\Delta,\mathscr{C}\right)$.
In both cases, morphisms are natural transformations of functors. 
\end{defn}
%
We will discuss about simplical objects mostly, and dualize everything
will give us the construction about cosimplicial objects. 

Let $\mathcal{C}$ be a category. If $X_{*}\in Ob\left(\mathbf{sSet}\right)$,
then $X_{n}\coloneqq X_{*}\left[n\right]$ is called the set of \emph{$n$-simplices}.
An $n$-simplex $x\in X_{n}$ is called \emph{degenerate} if $x\in\text{Im}\left(s_{j}\right)$
for some $j$. 

Any simplicial object $X_{*}$ can be written explicitly in the following
manner.
\[
X_{*}=\left[\xymatrix{X_{0}\ar@{.>}[r]|-{s_{0}} & X_{1}\ar@<-1ex>[l]_{d_{0}}\ar@<1ex>[l]^{d_{1}}\ar@{.>}@<1ex>[r]\ar@{.>}@<-1ex>[r] & \cdots\ar@<-2ex>[l]\ar[l]\ar@<2ex>[l]}
\right]
\]
where $d_{i}\coloneqq X_{*}\left(d^{i}\right),s_{j}\coloneqq X_{*}\left(s^{j}\right)$
satisfies the simplicial conditions
\[
\begin{array}{ccll}
d_{i}d_{j} & = & d_{j-1}d_{i} & i\leq j\\
d_{i}s_{j} & = & s_{j-1}d_{i} & i<j\\
d_{i}s_{j} & = & \text{Id} & i=j.j+1\\
d_{i}s_{j} & = & s_{j}d_{i-1} & i>j+1\\
s_{i}s_{j} & = & s_{j+1}s_{i} & i\leq j
\end{array}
\]

\begin{example}
$\mathscr{C}=\mathbf{Set},\mathbf{s}\mathscr{C}=\mathbf{sSet}$ is
the category of simplicial sets. The \emph{cartesian product} of simplicial
sets is the categorical product in $\mathbf{sSet}$. Explicitly, given
simplicial sets $X$ and $Y$, the product $X\times Y$ is given by
\[
\begin{array}{ccc}
\left(X\times Y\right)_{n} & = & X_{n}\times Y_{n}\end{array}
\]
with face and degenracy maps given by 
\[
\begin{array}{ccc}
d_{i}\left(x,y\right) & = & \left(d_{i}x,d_{i}y\right)\\
s_{j}\left(x,y\right) & = & \left(s_{j}x,s_{j}y\right)
\end{array}
\]
\end{example}
\begin{defn}
The \emph{standard $n$-simplex}, denoted $\Delta\left[n\right]_{*}$,
is a simplicial set defined as the functor $\text{Hom}_{\Delta}\left(-,\left[n\right]\right)$
where $\left[n\right]$ denotes the ordered set $\left\{ 0,1,\cdots n\right\} $. 
\end{defn}
By the Yoneda lemma, the $n$-simplices of a simplicial set $X$ stand
in 1-1 correspondence with the natural transformations from $\Delta\left[n\right]_{*}$
to $X$, i.e. 
\[
{\displaystyle X_{n}=X([n])\cong\text{Hom}_{\textbf{sSet}}\left(\Delta\left[n\right]*,X\right)}
\]

\begin{rem}
Here we abuse the notation $s_{j}$ by taking it as $X_{*}\left(s_{j}\right)$
in the category $\mathcal{C}$. The proper elements should be considered
when used. 
\end{rem}

\subsection{Geometric Realization}

\subsubsection{Geometric $n$-simplex $\Delta^{n}$ }

The \emph{geometric $n$-simplex} $\Delta^{n}$ is the topological
space defined by 
\[
\Delta^{n}=\left\{ \left(x_{0},\cdots,x_{n}\right)\in\mathbb{R}^{n+1}|\sum_{i=0}^{n}x_{i}=1,x_{i}\geq0,\forall0\leq i\leq n\right\} ,
\]
 i.e. it is the convex hull of unit vectors $\left\{ e_{i}=\left(0,\cdots,1,\cdots,0\right)\right\} _{i=0}^{n}$
in $\mathbb{R}^{n+1}$. 

Given a morphism $f:\left[n\right]\to\left[m\right]$ in $\Delta$,
we define 
\[
\begin{array}{cccc}
\Delta^{*}\left(f\right): & \Delta^{n}\subseteq\mathbb{R}^{n+1} & \longrightarrow & \Delta^{m}\subseteq\mathbb{R}^{m+1}\\
 & e_{i} & \longmapsto & e_{f\left(i\right)}
\end{array}
\]
which defines a functor 
\[
\begin{array}{cccc}
\Delta^{*}: & \Delta & \longrightarrow & \mathbf{Top}\\
 & \left[n\right] & \longmapsto & \Delta^{n}
\end{array}
\]
i.e. a cosimplicial space. Recall that 
\[
d^{i}\left(e_{k}\right)=\begin{cases}
e_{k}, & k<i\\
e_{k+1}, & k\geq i
\end{cases}
\]
so it extends to ${\displaystyle \sum_{k=0}^{n-1}x_{k}e_{k}}\mapsto{\displaystyle \sum_{k=0}^{i-1}x_{k}e_{k}}+{\displaystyle \sum_{k=i}^{n-1}x_{k}e_{k+1}}$,
and we have $d^{i}:\Delta^{n-1}\to\Delta^{n},\left(x_{0},\cdots,x_{n-1}\right)\mapsto\left(x_{0},\cdots,x_{i-1},0,x_{i},\cdots,x_{n}\right)$.
Geometrically, in $\Delta^{n}$ we define $i$-th $\left(n-1\right)$-dimensional
face to be the one opposite to $e_{i}$. Then $d^{i}:\Delta^{n-1}\hookrightarrow\Delta^{n}$
is the inclusion of $i$-th face into $\Delta^{n}$. 

Dually, $s^{j}:\Delta^{n+1}\to\Delta^{n}$ is given by $\left(x_{0},\cdots,x_{n+1}\right)\mapsto\left(x_{0},\cdots,x_{j-1},x_{j}+x_{j+1},\cdots,x_{n+1}\right)$.
Geometrically, $s^{j}$ collapse the $j$-th and $\left(j+1\right)$-th
vertices in $\Delta^{n+1}$ to a point. 
\begin{defn}
The (classical) \emph{geometric realization} of a simplicial set $X$
is 
\[
\left|X\right|\coloneqq\bigcup_{\sigma\in X}\Delta_{\sigma}=\coprod_{n\geq0}X_{n}\times\Delta^{n}/\sim
\]
 where $\left(d_{i}x,\sigma\right)\sim\left(x,d_{i}\sigma\right)$
and $\left(s_{j}x,\sigma\right)\sim\left(x,s_{j}\sigma\right)$. 
\end{defn}
\begin{thm}
The natural map $\left|X\times Y\right|\to\left|X\right|\times\left|Y\right|$
is a bijection, and is a homeomorphism if the product is formed in
the category of compactly generated spaces. 
\end{thm}

\subsubsection{Geometric Realization As Coends}

Assume $\mathcal{D}$ is a cocomplete category with arbitrary coproducts
and $\mathcal{C}$ is a small category. 
\begin{defn}
Given a bifunctor $S:\mathcal{C}^{op}\times\mathcal{C}\rightarrow\mathcal{D}$,
define the coend of $S$ by
\[
\int^{c\in\text{Ob}\left(\mathcal{C}\right)}S\left(c,c\right):=\text{Coeq}\left\{ \begin{array}{c}
\xymatrix{\underset{\underset{f\in\text{Mor}\left(\mathcal{C}\right)}{f:c\rightarrow d}}{\coprod}S\left(d,c\right)\ar@<1ex>[r]^{f^{*}}\ar@<-1ex>[r]_{f_{*}} & \underset{c\in\text{Ob}\left(\mathcal{C}\right)}{\coprod}S\left(c,c\right)}
\end{array}\right\} 
\]
 where $f^{*}=S\left(f,\text{Id}\right):S\left(d,c\right)\rightarrow S\left(c,c\right)$
and $f_{*}=S\left(\text{Id},f\right):S\left(d,c\right)\rightarrow S\left(d,d\right)$.
By UMP for colimits, a coend $X:=\int^{c\in\text{Ob}\left(\mathcal{C}\right)}S\left(c,c\right)$
comes together with a family of morphisms $\left\{ \varphi_{c}:S\left(c,c\right)\rightarrow X\right\} _{c\in\text{Ob}\left(\mathcal{C}\right)}$
making the diagram 
\[
\xymatrix{S\left(d,c\right)\ar[r]^{f^{*}}\ar[d]_{f_{*}} & S\left(c,c\right)\ar[d]^{\varphi_{c}}\\
S\left(d,d\right)\ar[r]^{\varphi_{d}} & X
}
\]
 commute and is initial among all such pairs. 
\end{defn}
We can extend some natural constructions as coends of some bifunctors.

\subsubsection{Functor Tensor Products}
\begin{defn}
A category $\mathcal{S}$ is called \emph{symmetric monoidal} if 
\begin{enumerate}
\item there is a bifunctor $\otimes:\mathcal{S}\times\mathcal{S}\to\mathcal{S}$
called \emph{tensor product}, and
\item there exists an object $\mathbf{1}\in Ob\left(\mathcal{S}\right)$
called the \emph{unit object} such that for any $a,b,c\in Ob\left(\mathcal{S}\right)$,
there are isomorphisms 
\[
\begin{array}{l}
a\otimes b\cong b\otimes a\\
\left(a\otimes b\right)\otimes c\xrightarrow[\cong]{\alpha_{a,b,c}}a\otimes\left(b\otimes c\right)\\
\mathbf{1}\otimes a\xrightarrow[\cong]{\lambda_{a}}a\xrightarrow[\cong]{\rho_{a}}a\otimes\mathbf{1}
\end{array}
\]
which are natural in $a,b,c$ and compatible in the sense that 2 axioms
\emph{triangle} 
\[
\xymatrix{\left(a\otimes\mathbf{1}\right)\otimes b\ar[rr]^{\alpha_{a,1,b}}\ar[dr]_{\rho_{a}^{-1}\otimes\text{Id}} &  & a\otimes\left(\mathbf{1}\otimes b\right)\ar[dl]^{\text{Id}\otimes\lambda_{b}}\\
 & a\otimes b
}
\]
and \emph{pentagon}
\[
\xymatrix{ &  & \left(a\otimes\left(b\otimes c\right)\right)\otimes d\ar[drr]^{\alpha_{a,b\otimes c,d}}\\
\left(\left(a\otimes b\right)\otimes c\right)\otimes d\ar[urr]^{\alpha_{a,b,c}\otimes\text{Id}}\ar[dr]_{\alpha_{a\otimes b,c,d,}} &  &  &  & a\otimes\left(\left(b\otimes c\right)\otimes d\right)\ar[dl]^{\text{Id}\otimes\alpha_{b,c,d}}\\
 & \left(a\otimes b\right)\otimes\left(c\otimes d\right)\ar[rr]^{\alpha_{a,b,c\otimes d}} &  & a\otimes\left(b\otimes\left(c\otimes d\right)\right)
}
\]
 holds.
\end{enumerate}
\end{defn}
%
\begin{defn}
$\mathcal{S}$ is called \emph{closed} if there exists a bifunctor
\[
\mathbf{Hom}_{\mathcal{S}}\left(-,-\right):\mathcal{S}^{op}\times\mathcal{S}\to\mathcal{S}
\]
such that 
\[
\mathbf{Hom}_{\mathcal{S}}\left(a\otimes b,c\right)\cong\mathbf{Hom}_{\mathcal{S}}\left(a,\mathbf{Hom}_{\mathcal{S}}\left(b,c\right)\right),\forall a,b,c\in Ob\left(\mathcal{S}\right).
\]
\end{defn}
\begin{example}
In $\mathcal{S}=\mathbf{sSet}$, the tensor product is given by 
\[
\begin{array}{cccc}
\otimes=\times: & \mathbf{sSet}\times\mathbf{sSet} & \to & \mathbf{sSet}\\
 & \left(X,Y\right) & \mapsto & X\times Y=\left\{ X_{n}\times Y_{n}\right\} _{n\geq0}
\end{array}
\]
and internal hom is given by 
\[
\begin{array}{cccl}
\mathbf{Hom}: & \mathbf{sSet}^{op}\times\mathbf{sSet} & \to & \mathbf{sSet}\\
 & \left(Y,Z\right) & \mapsto & \mathbf{Hom}\left(Y,Z\right)=\left\{ \text{Hom}_{\mathbf{sSet}}\left(Y\times\Delta\left[n\right],Z\right)\right\} _{n\geq0}
\end{array}
\]
we have 
\[
\mathbf{Hom}\left(X\times Y,Z\right)\cong\mathbf{Hom}\left(X,\mathbf{Hom}\left(Y,Z\right)\right).
\]
\end{example}
\begin{defn}
An \emph{$\mathcal{S}$-category} is a category $\mathcal{M}$ enriched
over $\mathcal{S}$, thus 
\begin{description}
\item [{(S1)}] For any objects $X,Y$ in $\mathcal{M}$, there exists an
object $\mathbf{Hom}_{\mathcal{M}}\left(X,Y\right)$ in $\mathcal{S}$
\item [{(S2)}] For any objects $X,Y,Z$ in $\mathcal{M}$, there exists
a morphism 
\[
c_{X,Y,Z}:\mathbf{Hom}_{\mathcal{M}}\left(Y,Z\right)\times\mathbf{Hom}_{\mathcal{M}}\left(X,Y\right)\to\mathbf{Hom}_{\mathcal{M}}\left(X,Z\right)
\]
in $\mathcal{S}$ called\emph{ composition law}. 
\item [{(S3)}] For any object $X$ in $\mathcal{M}$, there is a morphism
$i_{X}:\bullet\to\mathbf{Hom}_{\mathcal{M}}\left(X,X\right)$ called
\emph{unit}. 
\item [{(S4)}] There is an isomorphism 
\[
\mathbf{Hom}_{\mathcal{S}}\left(\bullet,\mathbf{Hom}_{\mathcal{M}}\left(X,Y\right)\right)\cong\text{Hom}_{\mathcal{M}}\left(X,Y\right).
\]
Thus the usual hom can be recovered from internal hom. 
\end{description}
\end{defn}
These data satisfy the compatibility axioms (\emph{triangle} and \emph{pentagon})
similar to the ones in $\mathcal{S}$. 
\begin{defn}
An $\mathcal{S}$-enriched category is called 
\begin{enumerate}
\item \emph{tensored over $\mathcal{S}$} if there exists a bifunctor 
\[
\boxtimes:\mathcal{S}\times\mathcal{M}\to\mathcal{M}
\]
viewed as an action of $\mathcal{S}$ on $\mathcal{M}$, such that
\[
\mathbf{Hom}_{\mathcal{M}}\left(v\boxtimes x,y\right)\cong\mathbf{Hom}_{\mathcal{M}}\left(v,\mathbf{Hom}_{\mathcal{M}}\left(x,y\right)\right).
\]
\item \emph{cotensored over $\mathcal{S}$} if there exists a bifunctor
\[
\left(-\right)^{-}:\mathcal{S}^{op}\times\mathcal{M}\to\mathcal{M}
\]
such that 
\[
\mathbf{Hom}_{\mathcal{M}}\left(v,\mathbf{Hom}_{\mathcal{M}}\left(x,y\right)\right)\cong\mathbf{Hom}_{\mathcal{M}}\left(x,y^{v}\right).
\]
\end{enumerate}
\end{defn}
Let $\mathcal{S}=\left(\mathbf{sSet},\times,\bullet\right)$ and $\mathcal{M}=\mathbf{s}\mathcal{C}=\mathbf{Fun}\left(\Delta^{op},\mathcal{C}\right)$,
then $\mathcal{M}$ has a canonical structure of simplicial category
tensored and cotensored over $\mathcal{S}$. 
\begin{itemize}
\item Tensor
\[
\begin{array}{cccl}
\boxtimes: & \mathbf{sSet}\times\mathbf{s}\mathcal{C} & \to & \mathbf{s}\mathcal{C}\\
 & \left(K,X\right) & \mapsto & K\boxtimes X=\left\{ {\displaystyle \coprod_{K_{n}}X_{n}}\right\} _{n\geq0}
\end{array}
\]
\item Internal hom 
\[
\mathbf{Hom}_{\mathbf{s}\mathcal{C}}\left(X,Y\right)=\left\{ \text{Hom}_{\mathbf{s}\mathcal{C}}\left(\Delta\left[n\right]\boxtimes X,Y\right)\right\} _{n\geq0}
\]
Note that $\left(K\boxtimes L\right)\boxtimes X\cong K\boxtimes\left(L\boxtimes X\right)$
and $\Delta\left[0\right]\boxtimes X\cong X\cong*\boxtimes X$. 
\item Fix $K\in Ob\left(\mathbf{sSet}\right)$, consider $K\boxtimes-:\mathbf{s}\mathcal{C}\to\mathbf{s}\mathcal{C}$.
Since $\mathcal{C}$ is cocomplete, so is $\mathbf{s}\mathcal{C}$,
hence $K\boxtimes-$ has right adjoint defined by left Kan extension
\[
\xymatrix{\mathbf{s}\mathcal{C}\ar[r]^{\text{Id}}\ar[d]_{K\boxtimes-} & \mathbf{s}\mathcal{C}\\
\mathbf{s}\mathcal{C}\ar[ur]_{L_{K\boxtimes-}\left(\text{Id}_{\mathbf{s}\mathcal{C}}\right)}
}
\]
Denote 
\[
Y^{K}\coloneqq L_{K\boxtimes-}\left(\text{Id}_{\mathbf{s}\mathcal{C}}\right)\left(Y\right),\forall Y\in Ob\left(\mathbf{s}\mathcal{C}\right),
\]
then by general properties of Kan extensions, we have 
\[
\text{Hom}_{\mathbf{s}\mathcal{C}}\left(K\boxtimes X,Y\right)\cong\text{Hom}_{\mathbf{s}\mathcal{C}}\left(X,Y^{K}\right).
\]
This implies for any $n\geq0$, 
\[
\begin{array}{ccl}
\mathbf{Hom}_{\mathbf{s}\mathcal{C}}\left(K\boxtimes X,Y\right)_{n} & = & \text{Hom}_{\mathbf{s}\mathcal{C}}\left(\Delta\left[n\right]\boxtimes K\boxtimes X,Y\right)\cong\text{Hom}_{\mathbf{s}\mathcal{C}}\left(K\boxtimes\Delta\left[n\right]\boxtimes X,Y\right)\\
 & \cong & \text{Hom}_{\mathbf{s}\mathcal{C}}\left(\Delta\left[n\right]\boxtimes X,Y^{K}\right)\eqqcolon\mathbf{Hom}_{\mathbf{s}\mathcal{C}}\left(X,Y^{K}\right)_{n}.
\end{array}
\]
\end{itemize}
%

\paragraph{Special cases}
\begin{enumerate}
\item Let $\mathcal{C}=\mathbf{Mod}\left(R\right)$ where $R$ is a unital
(commutative) associative ring. Then $\coprod=\bigoplus$. Let $\mathcal{M}=\mathbf{sMod}\left(R\right)\cong\mathbf{Com}_{\geq0}\left(R\right)$.
The tensor product is given by
\[
\begin{array}{cccl}
\bigoplus: & \mathbf{sSet}\times\mathbf{sMod}\left(R\right) & \to & \mathbf{sMod}\left(R\right)\\
 & \left(K,X\right) & \mapsto & \left\{ {\displaystyle \bigoplus_{K_{n}}X_{n}}\right\} _{n\geq0}=\left\{ R\left[K_{n}\right]\otimes_{R}X_{n}\right\} _{n\geq0}
\end{array}
\]
where $R\left[K_{n}\right]$ is the free bimodule based on $K_{n}$.
We need to check that this agrees with simplicial operations. Internal
hom is defined by 
\[
\mathbf{Hom}_{\mathbf{sMod}\left(R\right)}\left(X,Y\right)\coloneqq\left\{ \text{Hom}_{\mathbf{sMod}\left(R\right)}\left(R\left[\Delta\left[n\right]\right]\otimes_{R}X,Y\right)\right\} _{n\geq0}.
\]
And 
\[
Y^{K}\coloneqq\mathbf{Hom}_{\mathbf{sSet}}\left(X,Y\right)
\]
where the $R$-module structure comes from the target. 
\item Let $\mathcal{C}=\mathbf{CommAlg}_{k}$ with $k$ a commutative ring.
$\mathcal{M}=\mathbf{sCommAlg}_{k}$, then $\coprod=\bigotimes_{k}$.
The tensor product is given by
\[
K\boxtimes A\cong\left\{ {\displaystyle \bigotimes_{K_{n}}}A\right\} _{n\geq0}.
\]
And internal hom is defined as 
\[
\mathbf{Hom}_{\mathbf{sCommAlg}_{k}}\left(A,B\right)=\left\{ \text{Hom}_{\mathbf{sCommAlg}_{k}}\left(\bigotimes_{\Delta\left[n\right]}A,B\right)\right\} _{n\geq0}.
\]
\item Let $\mathcal{M}=\mathbf{Top}$ the category of compactly generated
weak Hausdorff spaces, then we have 
\[
\begin{array}{cccc}
\boxtimes: & \mathbf{sSet}\times\mathbf{Top} & \to & \mathbf{Top}\\
 & \left(K,X\right) & \mapsto & \left|K\right|\times X
\end{array}
\]
and 
\[
\begin{array}{cccl}
\left(-\right)^{-}: & \mathbf{sSet}^{op}\times\mathbf{Top} & \to & \mathbf{Top}\\
 & \left(K,Y\right) & \mapsto & Y^{K}\coloneqq\mathbf{Map}\left(\left|K\right|,Y\right)
\end{array}
\]
\end{enumerate}
%
Let $\mathcal{S}=\left(\mathcal{S},\bigotimes,\mathbf{1}\right)$
be a closed symmetric monoidal category and $\mathcal{C}$ be a small
category. $\mathcal{M}$ is a cocomplete $\mathcal{S}$-category (tensored
over $\mathcal{S}$). 
\[
\boxtimes:\mathcal{S}\times\mathcal{M}\to\mathcal{M}
\]

\begin{defn}
Given two functors $G:\mathcal{C}^{op}\to\mathcal{S}$ and $F:\mathcal{C}\to\mathcal{M}$,
define the \emph{functor tensor product} $G\underset{\mathcal{C}}{\boxtimes}F$
as
\[
G\underset{\mathcal{C}}{\boxtimes}F\coloneqq\int^{c\in Ob\left(\mathcal{C}\right)}G\left(c\right)\boxtimes F\left(c\right)=\text{coeq}\left\{ \xymatrix{{\displaystyle \coprod_{f:c\to c'}}G\left(c'\right)\boxtimes F\left(c\right)\ar@<1ex>[r]^{f^{*}}\ar@<-1ex>[r]_{f_{*}} & {\displaystyle \coprod_{c\in Ob\left(\mathcal{C}\right)}}G\left(c\right)\boxtimes F\left(c\right)}
\right\} 
\]
where 
\[
\begin{array}{cccccc}
f^{*}: & G\left(c'\right)\boxtimes F\left(c\right) & \xrightarrow{G\left(f\right)\boxtimes\text{Id}} & G\left(c\right)\boxtimes F\left(c\right) & \hookrightarrow & {\displaystyle \coprod_{c\in Ob\left(\mathcal{C}\right)}}G\left(c\right)\boxtimes F\left(c\right)\\
f_{*}: & G\left(c'\right)\boxtimes F\left(c\right) & \xrightarrow{\text{Id}\boxtimes F\left(f\right)} & G\left(c'\right)\boxtimes F\left(c'\right) & \hookrightarrow & {\displaystyle \coprod_{c\in Ob\left(\mathcal{C}\right)}}G\left(c\right)\boxtimes F\left(c\right)
\end{array}
\]
\end{defn}
\begin{example}
Let $\mathcal{S}=\left(\mathbf{Ab},\bigotimes_{\mathbb{Z}},\mathbb{Z}\right)$
and $\mathcal{M}=\mathcal{S}$ with $\mathbf{Hom}_{\mathcal{S}}=\text{Hom}_{\mathbf{Ab}}=\text{Hom}_{\mathbb{Z}}$.
$\mathcal{S}$ is enriched and tensored over itself with $\boxtimes=\bigotimes_{\mathbb{Z}}$.

Take a unital associative ring $R$, we can think of $\underline{R}$
as the category with one object $\left\{ *\right\} $ enriched over
$\mathcal{S}$. 

A left module over $R$ is an $\mathcal{S}$-functor 
\[
\begin{array}{cccc}
\underline{F}: & \underline{R} & \to & \mathbf{Ab}\\
 & * & \mapsto & M\\
 & R & \mapsto & \text{End}\left(M\right)
\end{array}
\]

A right module over $R$ is an $\mathcal{S}$-functor 
\[
\begin{array}{cccc}
\underline{G}: & \underline{R}^{op} & \to & \mathbf{Ab}\\
 & * & \mapsto & N\\
 & R^{op} & \mapsto & \text{End}\left(N\right)
\end{array}
\]

Let's think of $R$ as a monoid and take the underlying (unenriched)
functors 
\[
\begin{array}{cccc}
F: & R & \to & \mathbf{Ab}\\
G: & R^{op} & \to & \mathbf{Ab}
\end{array}
\]
Then 
\[
G\underset{R}{\boxtimes}F=\int^{R}N\bigotimes_{\mathbb{Z}}M\cong\frac{N\bigotimes_{\mathbb{Z}}M}{\left\langle nr\otimes m-n\otimes rm\right\rangle }\cong N\bigotimes_{R}M
\]
is the usual tensor product of left and right modules. 
\end{example}
%
\begin{example}
(Kan extension) Given 
\[
\xymatrix{\mathcal{C}\ar[r]^{F}\ar[d]_{G} & \mathcal{D}\\
\mathcal{E}\ar[ur]_{\mathcal{L}_{G}\left(F\right)}^{\Downarrow\eta}
}
\]
the left Kan extension can be interpreted as a tensor product 
\[
\mathcal{L}_{G}\left(F\right)\left(e\right)=\left(G\circ h_{e}\right)\underset{\mathcal{C}}{\boxtimes}F,e\in Ob\left(G\right).
\]
where $G\circ h_{e}=\text{Hom}_{\mathcal{D}}\left(G\left(-\right),e\right):\mathcal{C}^{op}\to\mathbf{Set}$. 
\end{example}
\begin{proof}
See later. We can also prove it by universal properties $\mathcal{L}_{G}\left(F\right)\left(e\right)\cong\text{colim}_{G/e}\left(F\right)$.
\[
\left(G\circ h_{e}\right)\underset{\mathcal{C}}{\boxtimes}F=\text{coeq}\left\{ \xymatrix{{\displaystyle \coprod_{f:c\to c'}}\text{Hom}_{\mathcal{C}}\left(G\left(c'\right),e\right)\boxtimes F\left(c\right)\ar@<1ex>[r]^{f^{*}}\ar@<-1ex>[r]_{f_{*}} & {\displaystyle \coprod_{c\in Ob\left(\mathcal{C}\right)}}\text{Hom}_{\mathcal{C}}\left(G\left(c\right),e\right)\boxtimes F\left(c\right)}
\right\} \cong\text{colim}_{G/e}\left(F\right)
\]
\end{proof}
\begin{example}
(classical geometric realization) Given 
\[
\xymatrix{\Delta\ar[r]^{\Delta^{\bullet}}\ar[d]_{h} & \mathbf{Top}\\
\mathbf{sSet}\ar[ur]_{\left|-\right|=\mathcal{L}_{h}\left(\Delta^{\bullet}\right)}^{\Downarrow\eta}
}
\]
For $X\in Ob\left(\mathbf{sSet}\right)$, $\mathcal{L}_{h}\left(\Delta^{\bullet}\right)\left(X\right)\cong\text{colim}_{\Delta X}\left(\Delta^{\bullet}\right)$.
Then
\[
\mathcal{L}_{h}\left(\Delta^{\bullet}\right)\left(X\right)\cong\text{Hom}_{\mathbf{sSet}}\left(\Delta\left[\cdot\right],X\right)\underset{\Delta}{\boxtimes}\Delta^{\bullet}\cong X\underset{\Delta}{\boxtimes}\Delta^{\bullet}=\coprod_{n\geq0}X_{n}\times\Delta^{n}/\left\langle \left(x,f_{*}\left(u\right)\right)\sim\left(f^{*}\left(x\right),u\right)\right\rangle _{f\in Mor\left(\Delta\right)}
\]
where $X_{n}$ is given the discrete topology. 
\end{example}
In particular, if $X:\Delta^{op}\to\mathbf{Space}$ is a simplicial
space ($\mathbf{Top}$ or $\mathbf{sSet}$) then $\left|-\right|:\mathbf{Space}^{\Delta^{op}}\to\mathbf{Space}$
is given by
\[
\left|X\right|\coloneqq X\underset{\Delta}{\boxtimes}\Delta^{\bullet}=\coprod_{n\geq0}X_{n}\times\Delta^{n}/\sim
\]
 where $X_{n}\times\Delta^{n}$ is given the product topology. 

\subsection{Classifying Space of A Category}

\subsubsection{Nerve of a category }

Recall a category $\mathscr{C}$ is small if its objects form a (proper)
set. Associate to such a category a simplicial set $\mathcal{N}_{*}\mathscr{C}$
(or $\mathcal{B}_{*}\mathscr{C}$) defined by 
\[
\begin{array}{ccl}
\mathcal{N}_{0}\mathscr{C} & = & \left\{ \text{objects in }\mathscr{C}\right\} =Ob\left(\mathscr{C}\right)\\
\mathcal{N}_{1}\mathscr{C} & = & \left\{ \text{morphisms in }\mathscr{C}\right\} =Mor\left(\mathscr{C}\right)\\
\mathcal{N}_{2}\mathscr{C} & = & \left\{ \text{composable morphisms }c_{0}\xrightarrow{f_{0}}c_{1}\xrightarrow{f_{1}}c_{2}\text{ in }\mathscr{C}\right\} \\
 & \cdots\\
\mathcal{N}_{n}\mathscr{C} & = & \left\{ n\text{-composable morphisms }c_{0}\xrightarrow{f_{0}}c_{1}\xrightarrow{f_{1}}\cdots\xrightarrow{f_{n-1}}c_{n}\text{ in }\mathscr{C}\right\} 
\end{array}
\]
\[
\xymatrix{\mathcal{N}_{2}\mathscr{C}: & c_{1}\ar[dr]^{f_{1}}\\
\bullet\ar[ur]^{f_{0}}\ar@{-->}[rr]^{f_{1}f_{1}} &  & c_{2}
}
\]
\[
\xymatrix{\mathcal{N}_{3}\mathscr{C}: & c_{1}\ar[ddr]^{f_{1}}\ar[drr]^{f_{2}f_{1}}\\
\bullet\ar[ur]^{f_{0}}\ar[drr]_{f_{1}f_{0}}\ar@{-->}[rrr]^{f_{2}f_{1}f_{0}} &  &  & c_{3}\\
 &  & c_{2}\ar[ur]_{f_{2}}
}
\]
This suggests that 
\[
\begin{array}{cccc}
d_{i}: & \mathcal{N}_{n}\mathscr{C} & \longrightarrow & \mathcal{N}_{n-1}\mathscr{C}\\
 & \left(c_{0}\xrightarrow{f_{0}}c_{1}\xrightarrow{f_{1}}\cdots\xrightarrow{f_{n-1}}c_{n}\right) & \longmapsto & \left(c_{0}\xrightarrow{f_{0}}\cdots\xrightarrow{f_{i-2}}c_{i-1}\xrightarrow{f_{i-1}}\hat{c}_{i}\xrightarrow{f_{i}}c_{i+1}\xrightarrow{f_{i+1}}\cdots\xrightarrow{f_{n-1}}c_{n}\right)\\
 & \left(f_{n-1},\cdots,f_{0}\right) & \longmapsto & \begin{cases}
\left(f_{n-1},\cdots,f_{1}\right) & i=0\\
\left(f_{n-1},\cdots,f_{i}f_{i-1},\cdots,f_{0}\right) & 1\leq i\leq n-1\\
\left(f_{n-2},\cdots,f_{0}\right) & i=n
\end{cases}
\end{array}
\]
\[
\begin{array}{cccc}
s_{j}: & \mathcal{N}_{n}\mathscr{C} & \longrightarrow & \mathcal{N}_{n+1}\mathscr{C}\\
 & \left(c_{0}\xrightarrow{f_{0}}c_{1}\xrightarrow{f_{1}}\cdots\xrightarrow{f_{n-1}}c_{n}\right) & \longmapsto & \left(c_{0}\xrightarrow{f_{0}}\cdots\xrightarrow{f_{i-1}}c_{i}\xrightarrow{\text{Id}}c_{i}\xrightarrow{f_{i}}c_{i+1}\xrightarrow{f_{i+1}}\cdots\xrightarrow{f_{n-1}}c_{n}\right)\\
 & \left(f_{n-1},\cdots,f_{0}\right) & \longmapsto & \left(f_{n-1},\cdots,f_{i},f_{i},\cdots,f_{0}\right)
\end{array}
\]

Another way to view this construction is the following. Let 
\begin{itemize}
\item $s,t:Mor\left(\mathscr{C}\right)\to Ob\left(\mathscr{C}\right)$ the
source and target maps. 
\item $i:Ob\left(\mathscr{C}\right)\to Mor\left(\mathscr{C}\right)$ the
identity morphism map.
\item $\circ:Mor\left(\mathscr{C}\right)\times_{Ob\left(\mathscr{C}\right)}Mor\left(\mathscr{C}\right)\to Mor\left(\mathscr{C}\right)$
where $Mor\left(\mathscr{C}\right)\times_{Ob\left(\mathscr{C}\right)}Mor\left(\mathscr{C}\right)$
is the fibred product 
\[
\xymatrix{Mor\left(\mathscr{C}\right)\times_{Ob\left(\mathscr{C}\right)}Mor\left(\mathscr{C}\right)\ar[r]^{\text{pr}_{1}}\ar[d]_{\text{pr}_{2}} & Mor\left(\mathscr{C}\right)\ar[d]^{t}\\
Mor\left(\mathscr{C}\right)\ar[r]^{s} & Ob\left(\mathscr{C}\right)
}
\]
\end{itemize}
Notice that the structure of $\mathscr{C}$ gives us 
\[
\xymatrix{Ob\left(\mathscr{C}\right)\ar@{-->}[r]^{i} & Mor\left(\mathscr{C}\right)\ar@<1ex>[l]^{t}\ar@<-1ex>[l]_{s}\ar@{-->}@<1ex>[r]\ar@{-->}@<-1ex>[r] & Mor\left(\mathscr{C}\right)\times_{Ob\left(\mathscr{C}\right)}Mor\left(\mathscr{C}\right)\ar[l]\ar@<2ex>[l]\ar@<-2ex>[l]}
.
\]
The new construction can be viewed as an ``extension'' of a category
to a ``full'' simplicial set 
\[
\xymatrix{\mathcal{N}_{0}\left(\mathscr{C}\right)\ar@{-->}[r]^{i} & \mathcal{N}_{1}\left(\mathscr{C}\right)\ar@<1ex>[l]^{t}\ar@<-1ex>[l]_{s}\ar@{-->}@<1ex>[r]\ar@{-->}@<-1ex>[r] & \mathcal{N}_{2}\left(\mathscr{C}\right)\ar[l]\ar@<2ex>[l]\ar@<-2ex>[l]\cdots}
\]
where 
\[
\mathcal{N}_{n}\left(\mathscr{C}\right)=\underset{n}{\underbrace{Mor\left(\mathscr{C}\right)\times_{Ob\left(\mathscr{C}\right)}Mor\left(\mathscr{C}\right)\times_{Ob\left(\mathscr{C}\right)}\cdots\times_{Ob\left(\mathscr{C}\right)}Mor\left(\mathscr{C}\right)}}\eqqcolon Mor_{n}\left(\mathscr{C}\right)
\]
The maps $d_{i}'s$ and $s_{j}$'s are the structure maps of iterated
fibred products.

$\mathcal{N}_{*}\mathscr{C}$ (as a simplicial set up to isomorphism)
determines $\mathscr{C}$ uniquely up to isomorphism. 
\begin{defn}
The \emph{classifying space} $B\mathcal{C}$ of a category $\mathcal{C}$
is the geometric realization of $\mathcal{NC}$. 
\end{defn}
\begin{prop}
If $\mathcal{C},\mathcal{C}'$ are topological categories and $F_{0},F_{1}:\mathcal{C}\to\mathcal{C}'$
are continuous functors, and $\alpha:F_{0}\to F_{1}$ is a morphism
of functors, then the induced map $BF_{0},BF_{1}:B\mathcal{C}\to B\mathcal{C}'$
are homotopic. 
\end{prop}
\begin{proof}
$\alpha$ can be regarded as a functor $\mathcal{C}\times\mathbf{1}\to\mathcal{C}'$
where $\mathbf{1}=\left\{ 0<1\right\} $ is a category, then 
\[
B\alpha:B\left(\mathcal{C}\times\mathbf{1}\right)\cong B\mathcal{C}\times B\mathbf{1}=B\mathcal{C}\times I\to B\mathcal{C}'
\]
is a homotopy between $BF_{0}$ and $BF_{1}$. 
\end{proof}
%

\subsection{The Classifying Space of a Topological Group}

Let $G$ be a topological group. $G$ can be viewed as a topological
category with a single object $*$ and $Mor\left(G\right)=G$, then
$\mathcal{N}_{*}G=\left\{ \mathcal{N}_{k}G=G^{k}\right\} _{k\geq0}$. 

The space $BG=\left|\mathcal{N}_{*}G\right|$ is the classifying space
of $G$ in the usual sense as one can see in the follows. 

Consider the category $\overline{G}$ with $Ob\left(\overline{G}\right)=G$
and $Mor\left(\overline{G}\right)=G\times G$ where there is a unique
isomorphism $\left(g_{1},g_{2}\right)$ between any two objects $g_{1}\to g_{2}$,
then $\overline{G}\simeq*$ as category via 
\[
\xymatrix{p:\overline{G}\ar@<1ex>[r] & \bullet:i\ar@<1ex>[l]}
\]
where $i\left(\bullet\right)=e\in G$ and $i\left(\text{Id}_{\bullet}\right)=\left(e,e\right)\in G$
and $p\left(g\right)=\bullet$ and $p\left(g_{1},g_{2}\right)=\text{Id}_{\bullet}$,
then we observe that 
\[
pi=\text{Id}_{\bullet}
\]
 and 
\[
ip\cong\text{Id}_{\overline{G}}
\]
via the natural isomorphism $\eta:\text{Id}_{\overline{G}}\Rightarrow ip$
where $\eta_{\left(g\right)}=\left(g,g^{-1}\right)$, and we have
\[
\xymatrix{g_{1}\ar[r]^{\left(g_{1},g_{1}^{-1}\right)}\ar[d]_{\left(g_{1},g_{1}^{-1}g_{2}\right)} & ip\left(g_{1}\right)=e\ar[d]^{\left(e,e\right)}\\
g_{2}\ar[r]_{\left(g_{2},g_{2}^{-1}\right)} & e
}
\]
so $EG\coloneqq B\overline{G}$ is contractible. 

. 

Note $G$ acts on $\mathcal{N}_{*}\overline{G}$ freely, $G$ acts
on $EG$ freely, 

\paragraph{Relation between $\mathcal{N}_{*}\bar{G}$ and $\mathcal{N}_{*}G$}
\begin{enumerate}
\item There is a functor $\overline{G}\to G,\left(g_{1},g_{2}\right)\mapsto g_{2}g_{1}^{-1}$
and it induces a map $\mathcal{N}_{*}\overline{G}\to\mathcal{N}_{*}G$
given by 
\[
\begin{array}{ccc}
\mathcal{N}_{n}\overline{G} & \longrightarrow & \mathcal{N}_{n}G\\
\left(g_{0},g_{1},\cdots,g_{n}\right) & \mapsto & \left(g_{1}g_{0}^{-1},g_{2}g_{1}^{-1},\cdots,g_{n}g_{n-1}^{-1}\right)
\end{array}
\]
\item There is a right $G$-action of $\mathcal{N}_{*}\bar{G}$ 
\[
\begin{array}{ccc}
\mathcal{N}_{n}\overline{G}\times G & \longrightarrow & \mathcal{N}_{n}G\\
\left(g_{0},\cdots,g_{n}\right)\times g & \longmapsto & \left(g_{0}g,\cdots,g_{n}g\right)
\end{array}
\]
such that 
\[
\xymatrix{\mathcal{N}_{*}\overline{G}\ar[rr]^{p}\ar@{->>}[dr] &  & \mathcal{N}_{*}G\\
 & \mathcal{N}_{*}\overline{G}/G\ar[ur]_{\cong}
}
\]
This gives an example of simplicial principal $G$-bundle. In particular,
this gives us 
\[
EG/G\cong BG.
\]
\item Note that $\mathcal{N}_{*}\bar{G}$ is a simplicial group which acts
on the left on $\mathcal{N}_{*}G$
\[
\begin{array}{ccc}
\mathcal{N}_{*}\bar{G}\times\mathcal{N}_{*}G & \longrightarrow & \mathcal{N}_{*}G\\
\left(g_{0},\cdots,g_{n}\right)\times\left(h_{1},\cdots,h_{n}\right) & \longmapsto & \left(g_{0}h_{1}g_{1}^{-1},g_{1}h_{2}g_{2}^{-1},\cdots,g_{n-1}h_{n-1}g_{n}^{-1}\right)
\end{array}
\]
\end{enumerate}

\subsubsection{Relation between $B\underline{G}$ and $BG$}

Recall Milnor's construction of a classifying space $X^{\infty}G$
for a topological group $G$ in last chapter, and we will compare
this with our construction here when view $G$ as a topological category
$G$ with single object and morphisms are elements in $G$. Since
$X^{\infty}G$ and and $BG$ can both be identified with the quotient
of a contractible space, it suffices compare $E^{\infty}G$ and $EG$. 

If $G$ is a countable CW group, then $E^{\infty}G\cong EG$ since
we can take the normal join topology on $E^{n}G$. Otherwise the topology
on $E^{\infty}G$ is stronger than $EG$. 

\subsection{Categorical Bar Construction }

\subsubsection{(Co)Monad}
\begin{defn}
A \emph{monad} (\emph{triple}) on a category $\mathcal{C}$ is given
by an endofunctor $T:\mathcal{C}\Rightarrow\mathcal{C}$ with two
morphisms $\eta:\text{Id}_{\mathcal{C}}\Rightarrow T$ and $\mu:T\circ T\Rightarrow T$
satisfying 
\begin{enumerate}
\item associativity 
\[
\xymatrix{T\circ T\circ T\ar@{=>}[r]^{T\mu}\ar@{=>}[d]_{\mu T} & T\circ T\ar@{=>}[d]^{\mu}\\
T\circ T\ar@{=>}[r]^{\mu} & T
}
\]
\item unitality 
\[
\xymatrix{T\ar@{=>}[r]^{\eta T}\ar@{=>}[d]_{T\eta} & T\circ T\ar@{=>}[d]^{\mu}\\
T\circ T\ar@{=>}[r]^{\mu} & T
}
\]
\end{enumerate}
\end{defn}
\begin{rem}
This can be regarded as a ``generalized'' associative, unital algebras. 
\end{rem}
\begin{defn}
A \emph{comonad} (\emph{cotriple}) on a category $\mathcal{C}$ is
given by an endofunctor $\bot:\mathcal{C}\Rightarrow\mathcal{C}$
with two morphisms $\varepsilon:\bot\Rightarrow\text{Id}_{\mathcal{C}}$
and $\delta:\bot\Rightarrow\bot\circ\bot$ satisfying coassociative
and counital diagrams. 
\end{defn}

\subsubsection{Main application}

Given a pair of adjoint functors $\xymatrix{F:\mathcal{C}\ar@<1ex>[r] & \mathcal{D}:U\ar@<1ex>[l]}
$ with unit $\eta:\text{Id}_{\mathcal{C}}\Rightarrow UF$ and counit
$\varepsilon:FU\Rightarrow\text{Id}_{\mathcal{D}}$, we can define
\[
\begin{array}{ll}
T=UF:\mathcal{C}\to\mathcal{C} & \mu:T\circ T\Rightarrow T,UFUF\overset{U\varepsilon F}{\Longrightarrow}UF\\
\bot=FU:\mathcal{D}\to\mathcal{D} & \delta:\bot\to\bot\circ\bot,FU\overset{F\eta U}{\Longrightarrow}FUFU
\end{array}
\]

\begin{claim}
$\left(T=UF,\eta,\mu\right)$ is a monad on $\mathcal{C}$ and $\left(\bot=FU,\varepsilon,\delta\right)$
is a comonad in $\mathcal{D}$.
\end{claim}
\begin{proof}
Use identities \ref{eq:comp} for adjunction morphisms. Then we have
\[
\begin{array}{c}
\left(FU\overset{F\eta U}{\Rightarrow}FUFU\overset{\varepsilon FU}{\Rightarrow}FU\right)=\text{Id}_{FU}\\
\left(UF\overset{\eta UF}{\Rightarrow}UFUF\overset{U\varepsilon F}{\Rightarrow}UF\right)=\text{Id}_{UF}
\end{array}
\]
which gives the unitality diagrams. The associativity diagram follows
from naturality of the unit and counit functors, i.e. we have 
\[
\xymatrix{UFUFUFx\ar[r]^{UFU\varepsilon_{Fx}}\ar[d]_{U\varepsilon_{FUFx}} & UFUFx\ar[d]^{U\varepsilon_{Fx}}\\
UFUFx\ar[r]^{U\varepsilon_{Fx}} & UFx
}
\]
and 
\[
\xymatrix{FUy\ar[r]^{F\eta_{Uy}}\ar[d]_{F\eta_{Uy}} & FUFUy\ar[d]^{FUF\eta_{Uy}}\\
FUFUy\ar[r]^{F\eta_{UFUy}} & FUFUFUy
}
\]
\end{proof}
\begin{claim}
Every monad in $\mathcal{C}$ gives a functor $\mathcal{C}\to\mathbf{c}\mathcal{C}$
and every comonad gives a functor $\mathcal{D}\to\mathbf{s}\mathcal{D}$. 
\end{claim}
\begin{proof}
Given $\left(\bot,\varepsilon,\delta\right)$ on $\mathcal{D}$, and
$A\in Ob\left(\mathcal{D}\right)$, we define 
\[
\begin{array}{cccl}
\bot_{*}: & \mathcal{D} & \to & \mathbf{s}\mathcal{D}\\
 & A & \mapsto & \bot_{*}A=\left\{ \bot_{n}A\right\} _{n\geq0}
\end{array}
\]
where $\bot_{n}A=\bot^{n+1}A$ and 
\[
\begin{array}{cccc}
d_{i}=\bot^{i}\cdot\varepsilon\cdot\bot^{n-i}: & \bot^{n+1}A & \to & \bot^{n}A\\
s_{j}=\bot^{j}\cdot\delta\cdot\bot^{n-j}: & \bot^{n+1}A & \to & \bot^{n+2}A
\end{array}
\]
which is similar to the bar construction. Explicitly, $\bot_{*}A$
can be expressed as follows
\[
\xymatrix{\bot A & \bot^{2}A\ar@<1ex>[l]\ar@<-1ex>[l] & \bot^{3}A\cdots\ar[l]\ar@<2ex>[l]\ar@<-2ex>[l]}
\]
The simplicial identities are satisfied because of the functoriality
of units and counits and the identity for adjunction morphisms. In
particular, $d_{i}s_{j}=\text{Id},i=j,j+1$ follows from 
\[
\begin{array}{c}
\left(FU\overset{F\eta U}{\Rightarrow}FUFU\overset{\varepsilon FU}{\Rightarrow}FU\right)=\text{Id}_{FU}\\
\left(FU\overset{F\eta U}{\Rightarrow}FUFU\overset{FU\varepsilon}{\Rightarrow}FU\right)=\text{Id}_{FU}
\end{array}
\]
\end{proof}
%
\begin{defn}
Let $\mathcal{C}$ be a category and $\left(T,\eta,\mu\right)$ be
a monad on $\mathcal{C}$. A \emph{$T$-algebra} is a pair $\left(A,\rho\right)$
where $A$ is an object in $\mathcal{C}$ and $\rho:TA\to A$ called
the structure map of the algebra such that the diagrams 
\[
\xymatrix{T^{2}A\ar[r]^{T\rho}\ar[d]_{\mu_{A}} & TA\ar[d]^{\rho} &  & A\ar[r]^{\eta_{A}}\ar@{=}[dr] & TA\ar[d]^{\rho}\\
TA\ar[r]^{\rho} & A &  &  & A
}
\]
commute. A morphism $f:\left(A,\rho_{A}\right)\to\left(B,\rho_{B}\right)$
of $T$-algebras is an morphism $f:A\to B$ in $\mathcal{C}$ such
that the diagram 
\[
\xymatrix{TA\ar[r]^{Tf}\ar[d]_{\rho_{A}} & TB\ar[d]^{\rho_{B}}\\
A\ar[r]^{f} & B
}
\]
commutes. 
\end{defn}
The category $\mathcal{C}^{T}$ of $T$-algebras and their morphisms
is called the \emph{Eilenberg\textendash Moore category} or category
of (Eilenberg\textendash Moore) algebras of the monad $T$. The forgetful
functor $U:\mathcal{C}^{T}\to\mathcal{C}$ has a left adjoint $F:\mathcal{C}\to\mathcal{C}^{T}$
which takes $A$ to $\left(TA,\mu_{A}\right)$. 

\subsubsection{Classifying Space of a Category}

Let $\mathcal{C}$ be a category and $\left(T,\eta,\mu)\right)$ be
a monad on $\mathcal{C}$. Let $\mathcal{C}^{T}$ be the category
of $T$-algebras in $\mathcal{C}$, there is a natural fogetful functor
$U:\mathcal{C}^{T}\to\mathcal{C}$ which has a left adjoint $F:\mathcal{C}\to\mathcal{C}^{T}$.
This pair of adjoint functors gives a comonad 
\[
\bot=FU:\mathcal{C}^{T}\to\mathcal{C}^{T},\eta:\bot\Rightarrow\text{Id}_{\mathcal{C}^{T}},\delta:\bot\Rightarrow\bot\circ\bot,FU\overset{F\eta U}{\Longrightarrow}FUFU
\]
 on $\mathcal{C}^{T}$, and it induces a functor 
\[
\begin{array}{cccl}
\bot_{*}: & \mathcal{C}^{T} & \to & \mathbf{s}\mathcal{C}^{T}\\
 & A & \mapsto & \bot_{*}A=\left\{ \bot_{n}A=\bot^{n+1}A\right\} _{n\geq0}
\end{array}
\]
where $\bot_{n}A=\bot^{n+1}A$ . 

Recall that the (augmented) simplex category $\Delta_{+}$ is defined
as follows. The category $\Delta$ has terminal object $\left[0\right]$
but no initial object. Define the augmented simplex categor $\Delta_{+}$
as 

$Ob\left(\Delta_{+}\right)=Ob\left(\Delta\right)\cup\left\{ \left[-1\right]\right\} $.

$\text{Hom}_{\Delta_{+}}\left(\left[n\right],\left[m\right]\right)=\begin{cases}
\text{Hom}_{\Delta}\left(\left[n\right],\left[m\right]\right) & n,m\geq0\\
\left[-1\right]\to\left[m\right] & n=-1\\
\emptyset & m=-1,n\geq0
\end{cases}$
\begin{defn}
For any category $\mathcal{C}$ we define the \emph{augmented simplicial
object} as a functor $X:\Delta_{+}\to\mathcal{C}$. Denote $\mathbf{\mathbf{s}_{+}\mathcal{C}}=\mathbf{Fun}\left(\Delta_{+},\mathcal{C}\right)$
. 

Explicitly, each $X\in Ob\left(\mathbf{s}_{+}\mathcal{C}\right)$
is given by $X\in Ob\left(\mathbf{s}\mathcal{C}\right)$ together
with $X_{-1}\in Ob\left(\mathcal{C}\right)$ and $\varepsilon:X_{0}\to X_{-1}$
in $Mor\left(\mathcal{C}\right)$ such that 
\[
\xymatrix{X_{-1} & X_{0}\ar[l]^{\varepsilon} & X_{1}\ar@<-1ex>[l]_{d_{1}}\ar@<1ex>[l]^{d_{0}}\cdots}
\]
agrees in the sense that $\varepsilon d_{1}=\varepsilon d_{0}$. We
can denote $\varepsilon=d_{0}$ and extend $d_{i}d_{j}=d_{j-1}d_{i},i<j$
for $n=0$. 
\end{defn}
%
Therefore the above functor $\bot_{*}$ is in fact a functor $\bot_{*}:\mathcal{C}^{T}\to\mathbf{s}_{+}\mathcal{C}^{T}$. 
\begin{defn}
The \emph{bar construction} $B\left(T,A\right)$ is the simplicial
$T$-algebra given by $\bot_{*}\left(A\right)$. If we forget the
$T$-algebra structure on $A$, $\bot_{*}\left(A\right)$ is an (aumented)
simplicial object in $\mathcal{C}$ which is called the bar resolution
of $A$. 
\end{defn}
%

\subsection{Special cases (Algebraic Bar Construction)}

\subsubsection{Modules over Commutative Algebras }

Let $A$ be a commutative associative algebras over some ring $k$.
Write $\text{Mod}\left(A\right)$ for the category of connective chain
complexes of modules over $A$.

For $N$ a right module, $A\otimes_{k}N$ is canonically a module.
This construction extends to a functor 
\[
\left(-\right)\otimes_{k}A:\text{Mod}\left(A\right)\to\text{Mod}\left(A\right).
\]
The monoid-structure on $A$ makes this a monad in $\mathbf{Cat}$:
the monad product and unit are given by the product and unit in $A$.

For $N$ a module its right action $\rho:N\otimes_{k}A\to N$ makes
the module an algebra over this monad. The bar construction $B\left(A,N\right)$
is then the simplicial module 
\[
\xymatrix{\cdots\ar[r] & N\otimes_{k}A\otimes_{k}A\ar@<1ex>[r]^{1_{N}\otimes\mu_{A}}\ar@<-1ex>[r]_{\rho\otimes1_{A}} & N\otimes_{k}A}
\]

Under the Moore complex functor of the Dold-Kan correspondence this
is identified with a chain complex whose differential is given by
the alternating sums of the face maps indicated above. 

This chain complex provides a resolution that computes the Tor. This
gives the Hochschild homology of $A$. 
\begin{rem}
This chain complex is what originally was called the bar complex in
homological algebra. Because the first authors denoted its elements
using a notation involving vertical bars (Ginzburg). 
\end{rem}

\subsubsection{Bar and cobar constructions of differential graded Hopf algebras}
\begin{defn}
A $\mathbb{Z}$-graded Hopf algebra is a $\mathbb{Z}$-graded vector
space, which, for that grading, is both a $\mathbb{Z}$-graded algebra,
$\left(A,\mu,\varepsilon\right)$, with unity $\varepsilon:k\to A$,
and a Z-graded coalgebra $\left(A,\Delta,\eta\right)$ such that:
\begin{itemize}
\item $\varepsilon:k\to A$ is a morphism of $\mathbb{Z}$-graded coalgebras;
\item $\eta:A\to k$ is a morphism of $\mathbb{Z}$-graded algebras; 
\item $\mu:A\otimes A\to A$ is a morphism of $\mathbb{Z}$-graded coalgebras. 
\end{itemize}
\end{defn}

\paragraph{Bar Construction}

Let $\left(A,d,\varepsilon\right)$ be a commutative, augmented differential
$\mathbb{Z}$-graded algebra, $d\left(A_{n}\right)\subseteq A_{n-1}$
and $\overline{A}=\text{Ker}\left(\varepsilon\right)$. 

The \emph{bar construction} $BA$ is given by 
\[
BA=\left(T\left(s\overline{A}\right),D\right)
\]
where 
\begin{itemize}
\item $T\left(s\overline{A}\right)$ is the commutative differential graded
Hopf algebra generated by $s\overline{A}$ where $s:A\to A$ is the
suspension operator, i.e. $\left(sA\right)_{n}=A_{n-1}$. 
\item $D=d_{I}+d_{E}$, where 
\[
\begin{array}{ccl}
d_{I}\left(sa_{1}\otimes\cdots\otimes sa_{n}\right) & = & -{\displaystyle \sum_{i=1}^{n}\eta\left(i-1\right)sa_{1}\otimes\cdots\otimes sa_{i-1}\otimes sda_{i}\otimes\cdots sa_{n}}\\
d_{E}\left(sa_{1}\otimes\cdots\otimes sa_{n}\right) & = & -{\displaystyle \sum_{i=1}^{n}\eta\left(i-1\right)sa_{1}\otimes\cdots\otimes sa_{i-1}a_{i}\otimes\cdots sa_{n}}
\end{array}
\]
with $\eta\left(i\right)=\left(-1\right)^{\sum_{k=1}^{i}\left|sa_{k}\right|}$. 
\end{itemize}
%
$BA$ has a tensor algebra construction. This from one point of view
handles the formal concatenation aspect, but has also a structure
of a coalgebraic structure with reduced diagonal, given by 
\[
\overline{\Delta}\left(v_{1}\otimes\cdots\otimes v_{n}\right)=\sum_{i=1}^{n-1}\left(v_{1}\otimes\cdots\otimes v_{p}\right)\otimes\left(v_{p+1}\otimes\cdots\otimes v_{n}\right).
\]


\paragraph{Cobar Construction}

Let $\left(C,\partial,\eta\right)$ be a cocommutative differential
Z-graded coaugmented coalgeba, $\partial\left(C_{n}\right)\subseteq C_{n-1}$,
$\overline{C}=C/\eta\left(k\right)$, $\overline{\Delta}:\overline{C}\to\overline{C}\otimes\overline{C}$. 

The cobar construction $\Omega C$ is the cocommutative pre-dgha defined
by 
\[
\Omega C=\left(T\left(s^{-1}\overline{C}\right),\delta\right)
\]
where 
\begin{itemize}
\item $T\left(s^{-1}\overline{C}\right)$ is the commutative differential
graded Hopf algebra generated by $s^{-1}\overline{C}$. 
\item $\delta=\partial_{I}+\partial_{E}$, where 
\[
\begin{array}{ccl}
\partial_{I}\left(s^{-1}c_{1}\otimes\cdots\otimes s^{-1}c_{n}\right) & = & -{\displaystyle \sum_{i=1}^{n}\eta\left(i-1\right)s^{-1}c_{1}\otimes\cdots\otimes s^{-1}c_{i-1}\otimes s^{-1}\partial c_{i}\otimes\cdots s^{-1}c_{n}}\\
\partial_{E}\left(s^{-1}c_{1}\otimes\cdots\otimes s^{-1}c_{n}\right) & = & -{\displaystyle \sum_{i=1}^{n}\eta\left(i-1\right)\sum_{\mu}\left(-1\right)^{\left|c_{i\mu}'\right|+1}s^{-1}c_{1}\otimes\cdots\otimes s^{-1}c_{i\mu}'\otimes s^{-1}c_{i\mu}''\otimes\cdots s^{-1}c_{n}}
\end{array}
\]
with $\eta\left(i\right)=\left(-1\right)^{\sum_{k=1}^{i}\left|s^{-1}c_{k}\right|}$
and $\overline{\Delta}\left(c_{i}\right)={\displaystyle \sum_{\mu}c_{i\mu}'\otimes c_{i\mu}''}$. 
\end{itemize}
%
\begin{rem}
If $A$ is not (graded) commutative, the differential $d_{E}$ of
$BA$ does not respect the shuffle product on $T\left(s\overline{A}\right)$;
$BA$ thus becomes merely a differential $\mathbb{Z}$-graded coalgebra.
Similarly if $C$ is not (graded) cocommutative $\Omega C$ is merely
a differential $\mathbb{Z}$-graded algebra. 
\end{rem}

\paragraph{Twisting cochains}
\begin{defn}
Let $\left(C,d_{C}\right)$ be a differential graded coalgebra with
comultiplication $\Delta$ and $\left(A,d_{A}\right)$ a dg-algebra
with multiplication $\mu$. A \emph{twisting cochain} is a morphism
$\tau:C\to A\left[1\right]$ such that the following Maurer-Cartan
equation holds:
\[
d_{A}\circ\tau+\tau\circ d_{C}+\mu\circ\left(\tau\otimes\tau\right)\circ\Delta=0.
\]
\end{defn}
%
Let $\mathbf{DGC}_{0}$ be the category of connected dg-coalgebras
and $\mathbf{DGA}_{a}$ the category of augmented dg-algebras. Then
the barconstruction functor $B:\mathbf{DGA}_{a}\to\mathbf{DGC}_{0}$
is a right adjoint to the cobar construction functor $\Omega:\mathbf{DGC}_{0}\to\mathbf{DGA}_{a}$. 

Starting from a dg-coalgebra map $f:C\to BA$, one constructs a twisting
cochain $\tau f$ by postcomposing $f$ by the natural projection
$BA\to A\left[1\right]$, the Maurer-Cartan equation for $\tau f$
translates to saying that $f$ is a chain map, $d_{BA}\circ f=f\circ d_{C}$.
One then replaces $\tau f$ by the composition of the evident canonical
map $\tau_{0}:\Omega C\to C\left[-1\right]$ (called the canonical
twisting cochain) and $\tau f\left[-1\right]:C\left[-1\right]\to A$
to obtain a morphism $f':\Omega C\to A$. The Maurer\textendash Cartan
equation for $\tau$ is equivalent also to saying that $f'$ is a
chain map, i.e. $d_{A}\circ f=f'\circ d_{\Omega C}$. 

\pagebreak{}

\section{Dold-Kan Correspondence}

\subsection{Pointwise Kan Extensions}
\begin{defn}
A right Kan extension is called \emph{pointwise} if it is preserved
by all (covariant) representable functors $h^{d}=\text{Hom}\left(d,-\right):\mathcal{D}\to\mathbf{Set},d\in Ob\left(\mathcal{D}\right)$.
\[
\xymatrix{\mathcal{C}\ar[r]^{F}\ar[d]_{G} & \mathcal{D}\ar[r]^{h^{d}} & \mathbf{Set}\\
\mathcal{D}\ar[ur]_{L_{G}\left(F\right)}^{\Uparrow}\ar@/_{1.5pc}/[urr]_{L_{G}\left(h^{d}\circ F\right)}
}
\]
\end{defn}
%
\begin{defn}
A left Kan extension is called \emph{pointwise} if it is mapped to
a right Kan extension by all (contravariant) representable functors
$h_{d}=\text{Hom}\left(-,d\right):\mathcal{D}^{op}\to\mathbf{Set},d\in Ob\left(\mathcal{D}\right)$.
\[
\xymatrix{\mathcal{C}\ar[r]^{F}\ar[d]_{G} & \mathcal{D}\ar[r]^{h_{d}} & \mathbf{Set}^{op}\\
\mathcal{D}\ar[ur]_{R_{G}\left(F\right)}^{\Downarrow}\ar@/_{1.5pc}/[urr]_{L_{G}\left(h_{d}\circ F\right)}
}
\]
\end{defn}
%
\begin{rem}
This is very similar to the property of limits and colimits.
\end{rem}
We have the following: 
\begin{quotation}
Absolute Kan extensions $\subsetneqq$ pointwise Kan extensions $\subsetneqq$
Kan extensions.
\end{quotation}
We will give a characterization (formula) for pointwise Kan extensions.

\subsubsection{Comma Category}
\begin{defn}
Given a functor $F:\mathcal{C}\to\mathcal{D}$ and an object $d\in Ob\left(\mathcal{D}\right)$,
we can define the \emph{comma category }$F/d$ (or $F\downarrow d$)
as follows. 
\begin{description}
\item [{Objects:}] $Ob\left(F/d\right)=\left\{ \left(c,f\right)|c\in Ob\left(\mathcal{C}\right),f\in\text{Hom}_{\mathcal{D}}\left(Fc,d\right)\right\} $.
\item [{Morphisms:}] $\text{Hom}_{F/d}\left(\left(c,f\right),\left(c',f'\right)\right)=\left\{ \varphi\in\text{Hom}_{\mathcal{C}}\left(c,c'\right)|f'\circ F\varphi=f\right\} $,
i.e. morphisms are those $\varphi:c\to c'$ such that the following
diagram 
\[
\xymatrix{Fc\ar[rr]^{F\varphi}\ar[dr]_{f} &  & Fc'\ar[dl]^{f'}\\
 & d
}
\]
commutes.
\end{description}
\end{defn}
Dually we can define cocomma category $d\backslash F$ (or $d\downarrow F$). 

Note that there is a forgetful functor $U:F/d\to\mathcal{C}$ which
can be thought as a fibre functor. 
\begin{example}
Let $F=\text{Id}_{\mathcal{D}}:\mathcal{D}\to\mathcal{D}$, then $F/d$
is the category over $d$ and $d\backslash F$ is the category under
$d$. 
\end{example}
%
\begin{example}
(Category of simplicial sets) Take 
\[
\begin{array}{cccc}
F=h_{*}: & \Delta & \to & \mathbf{sSet}\\
 & \left[n\right] & \mapsto & \Delta\left[n\right]_{*}
\end{array}
\]
For any $X\in Ob\left(\mathbf{sSet}\right)$ we call the category
$\Delta X\coloneqq h_{*}/X$ the category of simplices of $X$, which
is given by 
\begin{description}
\item [{Objects:}] $Ob\left(\Delta X\right)=\left\{ \left(\left[n\right],x\right):\left[n\right]\in\Delta,x\in\text{Hom}_{\mathbf{sSet}}\left(\Delta\left[n\right]_{*},X\right)=X_{n}\right\} ={\displaystyle \coprod_{n\geq0}X_{n}}$.
\item [{Morphisms:}] $\text{Hom}\left(\left(\left[n\right],x\right),\left(\left[m\right],y\right)\right)=\left\{ f:\left[n\right]\to\left[m\right]|X\left(f\right)y=y\circ h_{*}\left(f\right)=x\right\} $.
\end{description}
\end{example}
Another way to define $\Delta X$ is to consider $X:\Delta^{op}\to\mathbf{Set}$
and take the Grothendieck construction $\Delta_{X}^{op}=\Delta^{op}\int X$
where 
\begin{description}
\item [{Objects:}] $Ob\left(\Delta_{X}^{op}\right)=\left\{ \left(\left[m\right],y\right)|\left[m\right]\in Ob\left(\Delta^{op}\right)=Ob\left(\Delta\right),y\in X\left(\left[m\right]\right)=X_{m}\right\} $.
\item [{Morphisms:}] $\text{Hom}\left(\left(\left[n\right],x\right)\left(\left[m\right],y\right)\right)=\left\{ f\in\text{Hom}_{\Delta^{op}}\left(\left[n\right],\left[m\right]\right)|X\left(f\right)x=y\right\} $. 
\end{description}
Hence $\Delta_{X}^{op}\cong\left(\Delta X\right)^{op}$. 

\begin{example}
Let $\mathcal{C}$ be a small category and take 
\[
\begin{array}{cccc}
\Delta^{*}: & \Delta & \hookrightarrow & \mathbf{Cat}\\
 & \left[n\right] & \mapsto & \overrightarrow{n}=\left\{ 0\to1\to\cdots\to n\right\} 
\end{array}
\]
This is a fully faithful functor. Then $\Delta^{*}/\mathcal{C}\coloneqq$
the simplicial complex over $\mathcal{C}$ defined by 
\begin{description}
\item [{Objects:}] $Ob\left(\Delta^{*}/\mathcal{C}\right)=\left\{ \left(\left[n\right],f\right)|\left[n\right]\in\Delta,f:\overrightarrow{n}\to\mathcal{C}\right\} \cong{\displaystyle \coprod_{n\geq0}}\mathcal{N}_{n}\mathcal{C}$. 
\item [{Morphisms:}] $\text{Hom}\left(\left(\left[n\right],f\right),\left(\left[m\right],g\right)\right)=\left\{ \varphi:\overrightarrow{n}\to\overrightarrow{m}|g\circ\Delta^{*}\left(\varphi\right)=f\right\} $. 
\end{description}
\end{example}
Therefore $\Delta^{*}/\mathcal{C}\cong\Delta\mathcal{NC}$. 
\begin{rem}
$\Delta X$ and $\left(\Delta X\right)^{op}$ are examples of Reedy
categories (with fibrant or cofibrant, respectively). 
\end{rem}

\subsubsection{Computing Kan extension via (co)limits}
\begin{thm}
A left Kan extension is pointwise if and only if it can be computed
by the formula 
\[
L_{G}\left(F\right)\left(e\right)=\text{colim}_{G/e}\left(G/e\xrightarrow{U}\mathcal{C}\xrightarrow{F}\mathcal{D}\right).
\]
\end{thm}
%
\begin{thm}
(Dual version) A right Kan extension is pointwise if and only if it
can be computed by the formula 
\[
R_{G}\left(F\right)\left(e\right)=\text{lim}_{G/e}\left(e\backslash G\xrightarrow{U}\mathcal{C}\xrightarrow{F}\mathcal{D}\right).
\]
\end{thm}
\begin{proof}
It suffices (and more convenient) to prove the dual version. Indeed,
$L_{G}\left(F\right)$ is characterized by 
\[
\begin{array}{ccc}
\text{Hom}_{\mathbf{Fun}\left(\mathcal{E},\mathcal{D}\right)}\left(L_{G}\left(F\right),H\right) & \cong & \text{Hom}_{\mathbf{Fun}\left(\mathcal{C},\mathcal{D}\right)}\left(F,G_{*}H\right)\\
\cong\downarrow &  & \downarrow\cong\\
\text{Hom}_{\mathbf{Fun}\left(\mathcal{E},\mathcal{D}\right)^{op}}\left(H,L_{G}\left(F\right)\right) &  & \text{Hom}_{\mathbf{Fun}\left(\mathcal{C},\mathcal{D}\right)^{op}}\left(G_{*}H,F\right)
\end{array}
\]
Note that $\mathbf{Fun}\left(\mathcal{C},\mathcal{D}\right)^{op}=\mathbf{Fun}\left(\mathcal{C}^{op},\mathcal{D}\right)$,
so $R_{G^{0}}\left(F^{0}\right)\cong L_{G}\left(F\right)$.

Since limits commutes with representable functors, i.e. given $F:\mathcal{J\to\mathcal{D}}$,
\[
\text{Hom}_{\mathcal{D}}\left(d,\lim_{\mathcal{J}}F\right)\cong\lim_{\mathcal{J}}\left(\text{Hom}_{\mathcal{D}}\left(d,F\left(-\right)\right)\right)
\]
so if $R_{G}\left(F\right)$ is given by such a formula, then it automatically
commutes with representable functors $h^{d}=\text{Hom}_{\mathcal{D}}\left(d,-\right),\forall d\in\mathcal{D}$. 

Assume that $R_{G}\left(F\right)$ is pointwise, then for any $d\in Ob\left(\mathcal{D}\right)$
and any $e\in Ob\left(\mathcal{E}\right)$, 
\[
\begin{array}{ccl}
\text{Hom}_{\mathcal{D}}\left(d,R_{G}F\left(e\right)\right) & = & h^{d}\left(R_{G}F\left(e\right)\right)=\left(h^{d}\circ R_{G}F\right)\left(e\right)\\
 & = & \text{Hom}_{\mathbf{Fun}\left(\mathcal{E},\mathbf{Set}\right)}\left(h^{e},h^{d}\circ R_{G}F\right)\\
 & \cong & \text{Hom}_{\mathbf{Fun}\left(\mathcal{E},\mathbf{Set}\right)}\left(h^{e},R_{G}\left(h^{d}\circ F\right)\right)\\
 & \cong & \text{Hom}_{\mathbf{Fun}\left(\mathcal{C},\mathbf{Set}\right)}\left(h^{e}\circ G,h^{d}\circ F\right)\\
 & \cong & \text{Hom}_{\mathbf{Fun}\left(\mathcal{C},\mathbf{Set}\right)}\left(\text{Hom}_{\mathcal{E}}\left(e,G\left(-\right)\right),\text{Hom}_{\mathbf{Fun}\left(\mathcal{D},\mathbf{Set}\right)}\left(d,F\left(-\right)\right)\right)\\
 & \cong & \text{the set of cones under }d\text{ of the functor }FU\\
 & = & \text{Hom}_{\mathbf{Fun}\left(e\backslash G,\mathcal{D}\right)}\left(\text{const}_{d},FU\right)\\
 & \cong & \text{Hom}_{\mathcal{D}}\left(d,\lim_{e\backslash G}FU\right)
\end{array}
\]
By Yoneda lemma, $R_{G}F\left(e\right)\cong\lim_{e\backslash G}\left(FU\right)$. 
\end{proof}
\begin{cor}
If $\mathcal{D}$ is cocomplete then every left Kan extension of $F:\mathcal{C}\to\mathcal{D}$
exists and is pointwise. If $\mathcal{D}$ is complete then every
right Kan extension of $F:\mathcal{C}\to\mathcal{D}$ exists and is
pointwise. 
\end{cor}
%
\begin{cor}
If $\mathcal{D}$ is cocomplete and $G$ is fully faithful, then $\eta_{\text{un}}$
is an isomorphism of functors. 
\end{cor}
\begin{proof}
Take $c\in Ob\left(\mathcal{C}\right)$ and consider $G/G\left(c\right)$,
then $G$ is fully faithful implies that $G/G\left(c\right)$ has
terminal object. Indeed, $Ob\left(G/G\left(c\right)\right)=\left\{ \left(c',f'\right)|c'\in Ob\left(\mathcal{C}\right),f':Gc'\to Gc\right\} $.
Then $\left(c,\text{Id}_{G\left(c\right)}\right)$ is terminal in
$G/G\left(c\right)$ because

\[
\text{Hom}\left(\left(c',f'\right),\left(c,\text{Id}_{G\left(c\right)}\right)\right)=\left\{ h:c'\to c|G\left(h\right)\circ\text{Id}_{G\left(e\right)}=f'\right\} =G^{-1}\left(f'\right)
\]
contains only one element. 

Recall, by UMP of colimits, if $\mathcal{J}$ has terminal object
$*$, then for any $F:\mathcal{J}\to\mathcal{D}$, $\text{colim}_{\mathcal{J}}\left(F\right)=F\left(*\right)$.
Now for any $c\in Ob\left(\mathcal{C}\right)$, take $e=G\left(c\right)$
and apply formula 
\[
L_{G}\left(F\right)\left(e\right)=\text{colim}_{G/G\left(e\right)}\left(FU\right)\cong FU\left(c,\text{Id}_{G\left(c\right)}\right)=F\left(c\right)
\]
So $L_{G}F\circ G\cong F$. 
\end{proof}
\begin{example}
(Co-Yoneda lemma) Simplest version. Consider 
\[
\xymatrix{\mathcal{C}\ar[r]^{F}\ar[d]_{\text{Id}_{\mathcal{C}}} & \mathcal{D}\\
\mathcal{C}\ar[ur]_{L_{\text{Id}_{\mathcal{C}}}F\cong F}
}
\]
then we have $L_{\text{Id}_{\mathcal{C}}}F\cong F$ and $F\left(c\right)\cong\text{colim}_{\mathcal{C}/c}\left(\mathcal{C}/c\xrightarrow{U}\mathcal{C}\xrightarrow{F}\mathcal{D}\right)$. 
\end{example}
%
\begin{example}
Take $\hat{\mathcal{C}}=\mathbf{Fun}\left(\mathcal{C}^{op},\mathbf{Set}\right)$
and Yoneda functor $h_{*}:\mathcal{C}\hookrightarrow\hat{\mathcal{C}}$
\[
\xymatrix{\mathcal{C}\ar@{^{(}->}[r]^{h_{*}}\ar@{^{(}->}[d]_{h_{*}} & \hat{\mathcal{C}}\\
\hat{\mathcal{C}}\ar[ur]_{L_{h}\left(h\right)\cong\text{Id}_{\hat{\mathcal{C}}}}
}
\]
Every presheaf on a small category $\mathcal{C}$ is canonically a
colimit of representable presheaf. For any $X\in Ob\left(\hat{\mathcal{C}}\right)$,
\[
X\cong\text{colim}_{h_{*}/X}\left(h_{*}/X\xrightarrow{U}\mathcal{C}\xrightarrow{h_{*}}\hat{\mathcal{C}}\right).
\]

Take $\mathcal{C}=\Delta$, then $\hat{\mathcal{C}}=\mathbf{sSet}$
and $h/X=\Delta X$ is the category of simplices over $X$. 

$Ob\left(\Delta X\right)=\coprod_{n\geq0}X_{n}$. Note $X_{n}\cong\text{Hom}_{\mathbf{sSet}}\left(\Delta\left[n\right]_{*},X\right)$.

$\text{Hom}_{\Delta X}\left(\left(\left[n\right],x\right),\left(\left[m\right],y\right)\right)=\left\{ f:\left[n\right]\to\left[m\right]|X\left(f\right)y=x\right\} $.
\[
\xymatrix{\Delta\ar@{^{(}->}[r]^{h_{*}}\ar[d]_{h_{*}} & \hat{\Delta}=\mathbf{sSet}\\
\mathbf{sSet}\ar[ur]_{L_{h}\left(h\right)\cong\text{Id}_{\mathbf{sSet}}}
}
\]

So we have $X=\text{colim}_{\Delta X}\left(\Delta X\xrightarrow{U}\Delta\xrightarrow{h}\mathbf{sSet}\right)=\text{colim}_{\Delta\left[n\right]_{*}\to X}\Delta\left[n\right]_{*}$. 
\end{example}

\subsection{Equivalence of Categories}

Given two locally small categories $\mathcal{C}$ and $\mathcal{D}$,
define $\mathbf{Adj}\left(\mathcal{C},\mathcal{D}\right)$ to be the
category of adjunctions as 
\begin{description}
\item [{Objects:}] $\left\{ \left(L,R,\varphi\right)|\xymatrix{L:\mathcal{C}\ar@<1ex>[r] & \mathcal{D}:R\ar@<1ex>[l]}
,\varphi:\text{Hom}_{\mathcal{D}}\left(F\left(-\right),-\right)\xrightarrow{\cong}\text{Hom}_{\mathcal{C}}\left(-,G\left(-\right)\right)\right\} $ where $\varphi$ is an isomorphism of bifunctors $\mathcal{C}^{op}\times\mathcal{C}\to\mathbf{Set}$.
\item [{Morphisms:}] $\text{Hom}_{\mathbf{Adj}\left(\mathcal{C},\mathcal{D}\right)}\left(\left(L,R,\varphi\right),\left(L',R',\varphi'\right)\right)=\left\{ \left(\alpha,\beta\right)|\alpha:L\Rightarrow L',\beta:R'\Rightarrow R,\varphi'=\beta_{*}\circ\varphi\circ\alpha^{*}\right\} $.
\end{description}
Explicitly, for each $c\in Ob\left(\mathcal{C}\right),d\in Ob\left(\mathcal{D}\right)$,
we have a commutative (factorization) diagram
\[
\xymatrix{\text{Hom}_{\mathcal{D}}\left(L'\left(c\right),d\right)\ar[r]^{\left(\alpha_{c}\right)^{*}}\ar[d]_{\varphi'_{c,d}} & \text{Hom}_{\mathcal{D}}\left(L\left(c\right),d\right)\ar[d]^{\varphi_{c,d}}\\
\text{Hom}_{\mathcal{C}}\left(c,R'\left(d\right)\right) & \text{Hom}_{\mathcal{C}}\left(c,R\left(d\right)\right)\ar[l]_{\left(\beta_{d}\right)_{*}}
}
\]

\begin{prop}
Let $\mathcal{C}$ be a small category, and $\mathcal{D}$ a locally
small, cocomplete, then there exists a natural equivalence of categories
\[
\xymatrix{\Phi:\mathcal{D^{C}}\ar@<1ex>[r]^{\cong} & \mathbf{Adj}\left(\hat{\mathcal{C}},\mathcal{D}\right):\Psi\ar@<1ex>[l]}
\]
where $\Psi$ is defined by restriction 
\[
\begin{array}{l}
\Psi\left(\xymatrix{L:\hat{\mathcal{C}}\ar@<1ex>[r] & \mathcal{D}:R\ar@<1ex>[l]}
,\varphi\right)=\left(h^{*}\left(L\right)=L\circ h:\mathcal{C}\overset{h}{\hookrightarrow}\hat{\mathcal{C}}\xrightarrow{L}\mathcal{D}\right)\\
\Psi\left(\alpha,\beta\right)=h^{*}\left(\alpha\right)=\alpha\circ h
\end{array}
\]
\end{prop}
\begin{proof}
Construction of $\Phi$. Given $F\in Ob\left(\mathcal{D^{C}}\right)$,
define 
\[
\Phi\left(F\right):\left(\xymatrix{L\left(F\right):\hat{\mathcal{C}}\ar@<1ex>[r] & \mathcal{D}:R\left(F\right)\ar@<1ex>[l]}
,\varphi\right)
\]
where $L\left(F\right)=L_{h}\left(F\right)$ and 
\[
\begin{array}{cccl}
R\left(F\right): & \mathcal{D} & \to & \hat{\mathcal{C}}\\
 & d & \mapsto & \text{Hom}_{\mathcal{D}}\left(F\left(-\right),d\right)=h_{d}\circ F:\mathcal{C}^{op}\to\mathbf{Set}
\end{array}
\]
Take any $c\in Ob\left(\mathcal{C}\right),d\in Ob\left(\mathcal{D}\right)$
and consider $h_{c}\in Ob\left(\hat{\mathcal{C}}\right)$, 
\[
\text{Hom}_{\hat{\mathcal{C}}}\left(h_{c},RF\left(d\right)\right)\overset{\text{Yoneda}}{\cong}RF\left(d\right)\left(c\right)=\text{Hom}_{\mathcal{D}}\left(Fc,d\right)=\text{Hom}_{\mathcal{D}}\left(L\left(F\right))h_{c},d\right)
\]
where the last equality follows from lemma on left Kan extension along
fully faithful functors. 

Hence $R\left(F\right)$ is right adjoint to $L\left(F\right)$ on
the representable functors. 

To extend it to all presheaves $X:\mathcal{C}^{op}\to\mathbf{Set}$,
we need the following facts 
\begin{enumerate}
\item Co-Yonada lemma. 
\[
\xymatrix{\mathcal{C}\ar@{^{(}->}[r]^{h}\ar@{^{(}->}[d]^{h} & \hat{\mathcal{C}}\\
\hat{\mathcal{C}}\ar[ur]_{\text{Id}_{\mathcal{C}}}
}
\]
Every $X$ is canonically a colimit to $h_{c}$'s. 
\[
X\cong\text{colim}_{h/X}\left(h/X\xrightarrow{U}\mathcal{C}\overset{h}{\hookrightarrow}\hat{\mathcal{C}}\right)\cong\text{colim}_{h/X}\left(h\right).
\]
 
\item Left Kan extension is given by 
\[
L\left(F\right)X=L_{h}F\left(X\right)\coloneqq\text{colim}_{h/X}\left(h/X\xrightarrow{U}\mathcal{C}\xrightarrow{F}\mathcal{D}\right)=\text{colim}_{h/X}\left(F\right).
\]
\end{enumerate}
Since colimit commutes with hom, 
\[
\begin{array}{ccl}
\text{Hom}_{\hat{\mathcal{C}}}\left(X,RF\left(d\right)\right) & \cong & \text{Hom}_{\hat{\mathcal{C}}}\left(\text{colim}_{h/X}\left(h\right),R\left(F\right)\left(d\right)\right)\\
 & \cong & \lim_{h/X}\text{Hom}_{\hat{\mathcal{C}}}\left(hU\left(-\right),R\left(F\right)\left(d\right)\right)\\
 & \cong & \lim_{h/X}\left(\text{Hom}_{\mathcal{D}}\left(LF\left(hU\left(-\right)\right),d\right)\right)\\
 & \cong & \text{Hom}_{\mathcal{D}}\left(\text{colim}_{h/X}LF\left(hU\left(-\right)\right),d\right)\\
 & \cong & \text{Hom}_{\mathcal{D}}\left(LF\left(\text{colim}_{h/X}h\right),d\right)\\
 & \cong & \text{Hom}_{\mathcal{D}}\left(LFX,d\right)
\end{array}
\]

We have $\Phi\circ\Psi\left(\xymatrix{L:\hat{\mathcal{C}}\ar@<1ex>[r] & \mathcal{D}:R\ar@<1ex>[l]}
,\varphi\right)=\left(\xymatrix{L:\hat{\mathcal{C}}\ar@<1ex>[r] & \mathcal{D}:R\ar@<1ex>[l]}
,\varphi\right)$ and a natural transformation $F\Rightarrow\Psi\circ\Phi\left(F\right)=h\circ L_{h}F$
by the universal property of left Kan extension.

Check that $\Phi$ and $\Psi$ are inverse to each other.  
\end{proof}
\begin{example}
The adjoint pair $\xymatrix{\left|-\right|:\mathbf{sSet}\ar@<1ex>[r] & \mathbf{Top}:\mathcal{S}\ar@<1ex>[l]}
$ comes from a cosimplicial object 
\[
\begin{array}{ccccc}
\Delta & \xrightarrow{\Delta^{*}} & \mathbf{sSet} & \xrightarrow{\left|-\right|} & \mathbf{Top}\\
\left[n\right] & \mapsto & \Delta\left[n\right] & \mapsto & \Delta^{n}
\end{array}
\]
\end{example}
%
\begin{example}
{[}Dold-Kan{]} The adjoint pair 
\[
\xymatrix{\mathcal{N}:\mathbf{sAb}\ar@<1ex>[r] & \mathbf{Ch}_{\geq0}\left(\mathbf{Ab}\right):\Gamma\ar@<1ex>[l]}
\]
comes from a cosimplicial object 
\[
\begin{array}{ccccccc}
\Delta & \xrightarrow{\Delta^{*}} & \mathbf{sSet} & \xrightarrow{\mathbb{Z}\left[-\right]} & \mathbf{sAb} & \xrightarrow{\mathcal{N}} & \mathbf{Ch}_{\geq0}\left(\mathbf{Ab}\right)\\
\left[n\right] & \mapsto & \Delta\left[n\right] & \mapsto & \mathbb{Z}\left[\Delta\left[n\right]\right] & \mapsto & \mathcal{N}_{*}\left(\mathbb{Z}\left[\Delta\left[n\right]\right]\right)
\end{array}
\]
\end{example}

\subsection{Dold-Kan correspondence}

\subsubsection{Simplicial Abelian Groups}

The functor $\xymatrix{\mathcal{N}:\mathbf{sAb}\ar[r] & \mathbf{Ch}_{\geq0}\left(\mathbf{Ab}\right)}
$ is defined as follows. 
\begin{defn}
The \emph{Moore complex} $A_{*}$associated to a simplcial abelian
group $A_{\bullet}$ is given by $A_{n}$ in dimension $n$ with differential
$\partial={\displaystyle \sum_{i=0}^{n}}\left(-1\right)^{i}d_{i}$
given by the alternating sum of the face maps.
\end{defn}
The simplicial identities easily imply that this is in fact a chain
complex. Thus $A_{\bullet}\mapsto A_{*}$ defines a functor from simplicial
abelian groups to chain complexes.
\begin{prop}
Let $A_{\bullet}$ be a simplicial abelian group. There is a subcomplex
$DA_{*}\subset A_{*}$ of the Moore complex such that $DA_{n}$ consists
of the sums of degenerate simplices in degree $n$.
\end{prop}
Consequently, if $A_{\bullet}$ is a simplicial abelian group, we
can consider the chain complex $\left(A/DA\right)_{*}$. This is functorial
in $A_{\bullet}$, and there is a natural transformation $A_{*}\to\left(A/DA\right)_{*}$. 

Nonetheless, in defining the Dold-Kan correspondence, we shall use
a different construction (which we will prove is isomorphic to $\left(A/DA\right)_{*}$).
\begin{defn}
Given a simplicial abelian group $A_{\bullet}$, we define the \emph{normalized
complex} $NA_{*}$ as 
\[
NA_{n}={\displaystyle \bigcap_{i<n}\left(d_{i}:A_{n}\to A_{n-1}\right)}
\]
with differential 
\[
d=d_{n}|_{NA_{n}}:NA_{n}\to NA_{n-1}.
\]
\end{defn}
This is a well-defined chain complex since for any element $a$ in
$NA_{n}$, $d_{i}a=0,i<n$, so $d_{i}d_{n}a=d_{n-1}d_{i}a=0$, i.e.
$d_{0}a\in NA_{n-1}$. Furthermore, $d_{n-1}d_{n}a=d_{n-1}d_{n-1}a=0$,
so $d$ is a differential. 

We thus have three different ways of obtaining a complex from $A_{\bullet}$.
By the way we defined the normalized chain complex, we have natural
morphisms 
\[
\xymatrix{NA_{*}\ar[r] & A_{*}\ar[r] & \left(A/DA\right)_{*}}
.
\]


\subsubsection{Dold-Kan correspondence}

Our goal is to prove: 
\begin{thm}
The adjoint pair 
\[
\xymatrix{\mathcal{N}:\mathbf{sAb}\ar@<1ex>[r] & \mathbf{Ch}_{\geq0}\left(\mathbf{Ab}\right):\Gamma\ar@<1ex>[l]}
\]
axsis an equivalence of categories. Moreover, the three complexes
$NA_{*},A_{*},\left(A/DA\right)_{*}$ are all naturally homotopically
equivalent (and the first and the last are even isomorphic).
\end{thm}
%
The normalized chain complex of a simplicial abelian group $A_{\bullet}$
 looks a lot different from $A_{\bullet}$, which has much more structure.
Nonetheless, we are going to see that it is possible to recover $A_{\bullet}$
entirely from this chain complex.

A key step in the proof of the Dold-Kan correspondence will be the
establishment of the functorial decomposition for any simplicial abelian
group 
\begin{equation}
\bigoplus_{\left[n\right]\twoheadrightarrow\left[k\right]}NA_{k}\cong A_{n}.\label{eq:decomp}
\end{equation}
Here the map from a factor $NA_{k}$ corresponding to some $\varphi:\left[n\right]\twoheadrightarrow\left[k\right]$
to $A_{n}$ is given by pulling back by $\varphi$. 

Now, let us assume that \ref{eq:decomp} is true. Motivated by this,
we shall define a functor from chain complexes to simplicial abelian
groups. Let us now determine how the simplicial maps will play with
the decomposition.

Given a map $f:\left[m\right]\to\left[n\right]$, we can compose it
with $\varphi:\left[n\right]\twoheadrightarrow\left[k\right]$ to
get a map $\varphi\circ f:\left[m\right]\to\left[k\right]$ which
can be factored as a surjection followed by an injection
\[
\xymatrix{\left[m\right]\ar@{->>}[r]^{\varphi_{1}} & \left[l\right]\ar@{^{(}->}[r]^{\varphi_{2}} & \left[k\right]}
.
\]
This corresponds to a map $NA_{k}\hookrightarrow A_{n}\xrightarrow{f^{*}}A_{m}$,
and we want to know where $f^{*}$ takes $NA_{k}$ into $A_{m}$.
Note that simplicial maps induced by injections in $\Delta$ preserve
$NA$. Hence we have a commutative diagram 
\[
\xymatrix{NA_{k}\ar[r]^{\varphi_{2}^{*}}\ar@{^{(}->}[d]_{\varphi^{*}} & NA_{l}\ar@{^{(}->}[d]^{\varphi_{1}^{*}}\\
A_{n}\ar[r]^{f^{*}} & A_{m}
}
\]
where $\varphi_{1}$ is surjective and thus $\varphi_{1}^{*}$ corresponds
to one of the map of the canonical decomposition. It follows that
we have a recipe for determining where the $\varphi$-factor $NA_{k}$
of $A_{n}$ goes. 

Given an chain complex $C_{*}$, we define a simplicial abelian group
$\Gamma C_{\bullet}$ by 
\[
\Gamma C_{n}=\bigoplus_{\left[n\right]\twoheadrightarrow\left[k\right]}C_{k}
\]
where the sum is taken over all surjections $\left[n\right]\twoheadrightarrow\left[k\right]$.
We can make this into a simplicial abelian group using the above \textquotedblleft recipe\textquotedblright{}
describing how the canonical decomposition for a simplicial abelian
group behaves, but there is a bit of subtlety.

Since the $\varphi_{2}^{*}$ in the explanation above does not make
sense, let us note that if we restrict to the subcategory $\Delta'\subset\Delta$
consisting of injective maps, then the map $\left[n\right]\mapsto C_{n}$
becomes a contravariant functor in a natural way. Indeed, we let the
map $C_{n}\to C_{m}$ induced by an injection $\left[m\right]\hookrightarrow\left[n\right]$
be zero unless $m=n-1$, in which case we let the map $C_{n}\to C_{n-1}$
be the differential. Since $C_{*}$ is a chain complex, this is indeed
a functor. So a chain complex gives an abelian presheaf on the \textquotedblleft semi-simplicial\textquotedblright{}
category. Note that if we started with a simplicial abelian group
$A_{\bullet}$, then if the chain complex $NA_{*}$ is made into a
contravariant functor $\Delta'\to\mathbf{Ab}$, we have gotten nothing
new: we just recover the simplicial structure maps. Indeed, if $\varphi:\left[m\right]\hookrightarrow\left[n\right]$
is an injection, then the map $\varphi^{*}:NA_{n}\to NA_{m}$ is zero
unless $\varphi=d_{n}$ and $m=n-1$. Otherwise $\varphi$ will contain
a $d_{i}$ for some $i<n$, and the definition of $NA_{*}$ completes
the proof. We thus see:
\begin{lem}
Let $C_{\ast}$ be a chain complex. Then there is a functor $\Delta'\to\mathbf{Ab},\left[n\right]\mapsto C_{n}$
and an injection $\left[m\right]\hookrightarrow\left[n\right]$ to
zero unless $m=n-1$ and the injection is $d_{n}$, in which case
it is the differential. If $A_{\bullet}$ is a simplicial abelian
group, this construction agrees with the simplicial maps when restricted
to $NA_{*}$. 
\end{lem}
%
Now, let us show how to make $\Gamma C_{\bullet}$ into a simplicial
abelian group. 

Given some map $\left[m\right]\to\left[n\right]$ in $\Delta$, we
map the individual terms as follows. Let $\varphi:\left[n\right]\twoheadrightarrow\left[k\right]$
be an epimorphism in $\Delta$, giving a factor $C_{k}\subset\Gamma C_{n}$,
we then define 
\[
C_{k}\to\Gamma C_{m}=\bigoplus_{\left[m\right]\twoheadrightarrow\left[l\right]}C_{l}
\]
as follows. We have a map $\left[m\right]\to\left[n\right]\twoheadrightarrow\left[k\right]$,
which we can factor as a composite $\left[m\right]\twoheadrightarrow\left[l\right]\hookrightarrow\left[k\right]$,
of a surjection and an injection. So we send $C_{k}$ (via $\varphi^{*}$,
which is defined by the functoriality) to $C_{m'}$, imbedded in $\Gamma C_{m}$
as the factor corresponding to the surjection $\left[m\right]\twoheadrightarrow\left[l\right]$. 
\begin{lem}
The above construction gives a functor from chain complexes to simplicial
abelian groups. 
\end{lem}
%
In fact, the above construction will give a simplicial object from
any semi-simplicial object.

As a result, we have constructed our functor \textgreek{sv} from chain
complexes to simplicial abelian groups. Note that there is a natural
transformation 
\[
\left(\Gamma NA_{*}\right)_{\bullet}\to A_{\bullet}
\]
for any simplicial abelian group $A_{\bullet}$. On the $n$-simplices,
this is the map 
\[
\bigoplus_{\varphi:\left[n\right]\twoheadrightarrow\left[k\right]}NA_{k}\to A_{n}
\]
where the factor corresponding to $\varphi$ is mapped to $A_{n}$
by pulling back by $\varphi$. This is the map discussed above. It
is immediate from the definition that this is a simplicial map. The
crux of the proof of the Dold-Kan correspondence is that this is an
isomorphism. 
\begin{prop}
\label{sab-ch} For a simplicial abelian group $A_{\bullet}$, we
have for each $n$, an isomorphism of abelian groups 
\[
\bigoplus_{\varphi:\left[n\right]\twoheadrightarrow\left[k\right]}NA_{k}\cong A_{n}
\]
Here the map is given by sending a summand $NA_{k}$ to $A_{n}$ via
the pull-back by the term $\varphi$. Alternatively, the morphism
of simplicial abelian groups 
\[
\left(\Gamma NA_{*}\right)_{\bullet}\to A_{\bullet}
\]
is an isomorphism. 
\end{prop}
This is going to take some work, and we are going to need first a
simpler splitting that will, incidentally, show that $NA_{*}$ and
$\left(A/DA\right)_{*}$ are isomorphic. We are going to prove the
above result by induction, using: 
\begin{lem}
Let $A_{\bullet}$ be a simplicial abelian group, then the map 
\[
NA_{n}\oplus DA_{n}\to A_{n}
\]
is an isomorphism. 
\end{lem}
\begin{proof}
(Sketch) The proof is by induction. Namely, for each $k<n$, we define
$N_{k}A_{n}={\displaystyle \bigcap_{0}^{k}}\ker d_{k}$ and $D_{k}A_{n}$
to be the group generated by the images of $s_{j}\left(A_{n-1}\right),j\leq k$.
So these are partial versions of $NA_{n},DA_{n}$. The claim is that
there is a natural splitting 
\[
N_{k}A_{n}\oplus D_{k}A_{n}\cong A_{n}.
\]

When $k=0$, this is $\ker d_{0}\oplus\text{Im}s_{0}\cong A_{n}$
which follows from 
\[
\xymatrix{A_{n-1}\ar@<1ex>[r]^{s_{0}} & A_{n}\ar@<1ex>[l]^{d_{0}}}
\]
where $d_{0}s_{0}=\text{Id}_{A_{n-1}}$, $s_{0}$ is called the splitting
injection and $d_{0}$ is a section. Hence $A_{n}$ splits as $\ker d_{0}\oplus\text{Im}s_{0}$. 

Now assume the splitting holds for $k-1$, $N_{k-1}A_{n}\oplus D_{k-1}A_{n}\cong A_{n}$. 

We have a split exact sequence 
\[
\xymatrix{0\ar[r] & A_{n-1}/D_{k-1}A_{n-1}\ar[r]^{s_{k}} & A_{n}/D_{k-1}A_{n}\ar[r] & A_{n}/D_{k}A_{n}\ar[r] & 0}
\]
where the splitting is given by $d_{k}$. 

Similarly, we have a split exact sequence (where the simplicial identities
show that $s_{k}\left(N_{k-1}A_{n-1}\right)\subset N_{k-1}A_{n}$)
\[
\xymatrix{0\ar[r] & N_{k-1}A_{n-1}\ar[r]^{s_{k}} & N_{k-1}A_{n}\ar[r] & N_{k}A_{n}\ar[r] & 0}
.
\]

Thus we have a commutative diagram 
\[
\xymatrix{0\ar[r] & N_{k-1}A_{n-1}\ar[r]^{s_{k}}\ar[d]^{\cong} & N_{k-1}A_{n}\ar[r]\ar[d]^{\cong} & N_{k}A_{n}\ar[r]\ar[d] & 0\\
0\ar[r] & A_{n-1}/D_{k-1}A_{n-1}\ar[r]^{s_{k}} & A_{n}/D_{k-1}A_{n}\ar[r] & A_{n}/D_{k}A_{n}\ar[r] & 0
}
\]
so by five lemma, $N_{k}A_{n}\cong A_{n}/D_{k}A_{n}$. Furthermore,
we have a map 
\[
\begin{array}{cccl}
r: & A_{n} & \to & N_{k}A\\
 & a & \mapsto & a-{\displaystyle \sum_{i=0}^{k}s_{i}d_{i}a}
\end{array}
\]
which gives the splitting. 
\end{proof}
\begin{cor}
The map $NA_{\ast}\to\left(A/DA\right)_{*}$ is an isomorphism of
chain complexes. 
\end{cor}
%
\begin{cor}
The map $NA_{*}\to\left(A/DA\right)_{*}$ is an isomorphism of chain
complexes. 
\end{cor}
%
Now we can prove proposition \ref{sab-ch}. 
\begin{claim}
The map 
\[
\Phi_{n}:\bigoplus_{\varphi:\left[n\right]\twoheadrightarrow\left[k\right]}A_{k}\to A_{n}
\]
is an isomorphism. 
\end{claim}
\begin{proof}
First we need to show $\Phi_{n}$ is surjective. By induction on $n$,
we may assume that $\Phi_{m}:\left(\Gamma NA_{*}\right)_{m}\to A_{m}$
is surjective for smaller $m<n$. Now An splits as the sum of $NA_{n}$
and $DA_{n}$. Clearly $NA_{n}$ is in the image of $\Phi_{n}$ (from
the factor $NA_{n}$). But by the inductive hypothesis, everything
in $A_{n-1}$ is in the image of $\Phi_{n-1}$, and taking degeneracies
now shows that anything in $DA_{n}$ is in the image of $\Phi_{n}$.
Thus $\Phi_{n}$ is surjective. 

Let us now show that $\Phi_{n}$ is injective. Suppose a family $\left(a_{\varphi}\right)\in{\displaystyle \bigoplus_{\varphi:\left[n\right]\twoheadrightarrow\left[k\right]}A_{k}}$
maps to zero under $\Phi_{n}$. By assumption, we have 
\[
\sum_{\varphi:\left[n\right]\twoheadrightarrow\left[k\right]}\varphi^{*}\left(a_{\varphi}\right)=0,
\]
then $a_{1:\left[n\right]\to\left[n\right]}\in A_{n}$ is zero since
this is the only term not in $DA_{n}$. 

We shall now define an ordering on the surjections $\varphi:\left[n\right]\twoheadrightarrow\left[k\right]$.
Say that $\varphi_{1}\leq\varphi_{2}$ if $\varphi_{1}\left(i\right)\leq\varphi_{2}\left(i\right)$
for each $i\in\left[n\right]$. We can assume that $\varphi$ is chosen
minimal with respect to this (partial) ordering such that $a_{\varphi}\neq0$.
Now choose a section $\phi:$$\left[k\right]\hookrightarrow\left[n\right]$
which is maximal in that $\phi$ is not a section of any $\varphi'>\varphi$.
If we think of $\varphi$ as determining a partition of $\left[n\right]$
into $k$ subsets, then we have $\phi$ sending $i\in\left[k\right]$
to the last element of the ith subset of $\left[n\right]$. Then $\varphi$
is a section of $\varphi$, and of no other $\varphi'<\varphi$. If
we apply $\phi^{*}$ to the equation $\sum_{\varphi:\left[n\right]\twoheadrightarrow\left[k\right]}\varphi^{*}\left(a_{\varphi}\right)=0$,
we find that 
\[
\Phi_{k}\left(\phi^{*}\left(a_{\varphi}\right)\right)=0
\]
which implies by the inductive hypothesis (as $k<n$) that $\phi^{*}$
pulls back $\left(a_{\varphi}\right)\in{\displaystyle \bigoplus_{\varphi:\left[n\right]\twoheadrightarrow\left[k\right]}A_{k}}$
to zero. But the component of the identity $\left[k\right]\to\left[k\right]$
of this pull-back is just $a_{\varphi}$. This means that $a_{\varphi}=0$. 
\end{proof}
%
We thus have defined a functor $\mathcal{N}$ from simplicial abelian
groups to chain complexes. We have defined a functor $\Gamma$ in
the opposite direction. We have, moreover, seen that the simplicial
abelian group associated to $NA_{*}$ for $A_{\bullet}$ a simplicial
abelian group is just $A_{\bullet}$ itself, in view of the canonical
decomposition of a simplicial abelian group. It suffices now, at least,
to prove that the normalized chain complex associated to $\Gamma C_{\bullet}$
is just $C_{*}$, for any chain complex $C_{*}$. So we need to compute
$N\left(\Gamma C_{\bullet}\right)_{*}$. 

In degree $n$, this consists of elements of ${\displaystyle \bigoplus_{\left[n\right]\twoheadrightarrow\left[k\right]}}C_{k}$
that are killed by the $d_{i},i<n$. The claim is that this consists
precisely of $C_{n}$ under the identity $\left[n\right]\twoheadrightarrow\left[n\right]$.
We can see this because we can show that $C_{n}\subset N\left(\Gamma C\right)_{n}$
by direct computation; if $i<n$, then the map $d^{i}:\left[n-1\right]\to\left[n\right]\twoheadrightarrow\left[n\right]$
pulls $C_{n}$ down to $C_{n-1}$ via the functor $\Delta'\to\mathbf{Ab}$
induced; however, this functor induces zero on coface maps that are
not the highest index. Conversely, we must show that $N\left(\Gamma C\right)_{n}\subset C_{n}$.
To do this, we know that $N\left(\Gamma C\right)_{n}$ is a complement
to the degeneracies. However, the $C_{k},k<n$ occurring in the expression
for $\left(\Gamma C\right)_{n}$ are all degeneracies. Thus our assertion
is clear. 

\subsection{{*}Simplicial groups and spaces}

\subsubsection{Twisted Cartesian products and principal bundles }

Let $\mathbf{sGr}=\mathbf{Fun}\left(\Delta^{op},\mathbf{Gr}\right)$
be the category of simplicial groups. 

Let $G_{*}=\left\{ G_{n}\right\} _{n\geq0}\in Ob\left(\mathbf{sGr}\right)$
and $X_{*}\in Ob\left(\mathbf{sSet}\right)$. 
\begin{defn}
A \emph{twisting function} $\tau:X_{*}\to G_{*-1}$ is a family of
maps $\left\{ \tau_{n}:X_{n}\to G_{n-1}\right\} _{n\geq1}$ such that
\[
\begin{array}{cclc}
d_{0}\left(\tau\left(x\right)\right) & = & \tau\left(d_{0}x\right)^{-1}\tau\left(d_{1}x\right)\\
d_{i}\left(\tau\left(x\right)\right) & = & \tau\left(d_{i+1}x\right) & i\geq1\\
s_{j}\left(\tau\left(x\right)\right) & = & \tau\left(s_{j+1}x\right) & j\geq0\\
\tau\left(s_{0}\left(x\right)\right) & = & 1_{G_{n}} & \forall x\in G_{n}
\end{array}
\]
\end{defn}
%
\begin{defn}
A \emph{(principal) twisted Cartesian product} with fibre $G_{*}$
and base $X_{*}$, and twisting function $\tau:X_{*}\to G_{*-1}$
is a simplicial set $E_{*}=G_{*}\times_{\tau}X_{*}$ with 
\[
E_{n}\coloneqq G_{n}\times X_{n},n\geq0
\]
and 
\[
\begin{array}{ccl}
d_{i}\left(g,x\right) & = & \begin{cases}
\left(\tau\left(x\right)\cdot d_{0}g,d_{0}x\right) & i=0,\\
\left(d_{i}g,d_{i}x\right) & i>0.
\end{cases}\\
s_{j}\left(g,x\right) & = & \left(s_{j}g,s_{j}x\right)\ \ \ \ \ \ \ \ \ \ \ \ \ \ j\geq0.
\end{array}
\]
\end{defn}
\begin{prop}
\label{prop-fib} Any principal $G_{*}$-fibration $p_{*}:E_{*}\to X_{*}$
with right $G_{*}$ action on $E_{*}$ with local cross section $\sigma_{*}:X_{*}\to E_{*}$
(i.e. $\sigma_{n}:X_{n}\to E_{n}$ such that $p_{n}\sigma_{n}=\text{Id}_{X_{n}}$
and $d_{i}\sigma=\sigma d_{i},\forall i>0$, $s_{j}\sigma=\sigma s_{j},\forall j\geq0$)
can be identified with $G\times_{\tau}X\to X$ where $\tau:X_{*}\to G_{*-1}$
is determined by $d_{0}\sigma\left(x\right)=\sigma\left(d_{0}x\right)\cdot\tau\left(x\right)$. 
\end{prop}

\subsection{Classifying Space of Simplicial Groups }

Given $G_{*}\in Ob\left(\mathbf{sGr}\right)$, define a reduced simplicial
set $\overline{\mathcal{W}}\left(G_{*}\right)$ by 
\[
\overline{\mathcal{W}}_{0}\left(G\right)\coloneqq\left\{ *\right\} ,\overline{\mathcal{W}}_{n}\left(G\right)\coloneqq G_{n-1}\times G_{n-2}\times\cdots\times G_{n},n\geq0
\]
with 
\[
\begin{array}{rccc}
s_{0}: & \overline{\mathcal{W}}{}_{0}\left(G\right) & \to & \overline{\mathcal{W}}{}_{1}\left(G\right)\\
 & * & \mapsto & 1_{G_{0}}\\
d_{0}=d_{1}: & \overline{\mathcal{W}}_{1}\left(G\right) & \to & \overline{\mathcal{W}}_{0}\left(G\right)\\
 & g & \mapsto & *
\end{array}
\]
and for $n\geq1$,
\[
\begin{array}{rll}
d_{0}\left(g_{n-1},\cdots,g_{0}\right) & = & \left(g_{n-2},\cdots,g_{0}\right)\\
d_{i+1}\left(g_{n-1},\cdots,g_{0}\right) & = & \left(d_{i}g_{n-1},\cdots,d_{1}g_{n-i},g_{n-i-2}\cdot d_{0}g_{n-i-1},g_{n-i-3},\cdots,g_{0}\right)\\
s_{0}\left(g_{n-1},\cdots,g_{0}\right) & = & \left(1,g_{n-1},\cdots,g_{0}\right)\\
s_{j+1}\left(g_{n-1},\cdots,g_{0}\right) & = & \left(s_{j}g_{n-2},\cdots s_{0}g_{n-j-1},1,g_{n-i-2},\cdots,g_{0}\right)
\end{array}
\]
This is a simplicial set  with a twisting function 
\[
\begin{array}{cccc}
\tau_{n}\left(G\right): & \overline{\mathcal{W}}_{n}\left(G\right) & \to & G_{n-1}\\
 & \left(g_{n-1},\cdots,g_{0}\right) & \mapsto & g_{n-1}
\end{array}
\]

\begin{lem}
\label{lem-twist} $\tau\left(G\right)$ is a universal twisting function
in the sense that any principal twisted product $G\times_{\tau}X$
can be induced from $G_{*}\times_{\tau\left(G\right)}X_{*}$ by a
unique classifying map $X_{*}\mapsto\overline{\mathcal{W}}\left(G_{*}\right)$
given by 
\[
x\in X_{n}\mapsto\left(\tau\left(x\right),\tau\left(d_{0}x\right),\cdots,\tau\left(d_{0}^{n-1}x\right)\right)\in\overline{\mathcal{W}}_{n}\left(G_{*}\right).
\]
\end{lem}
%
\begin{example}
If $G_{*}=\left\{ G_{n}\right\} _{n\geq0}$ is a discrete simplicial
group, $\overline{\mathcal{W}}\left(G\right)=B_{*}G$ the simplicial
nerve of $G$, then 
\[
\begin{array}{ccc}
G\times_{\tau\left(G\right)}\overline{\mathcal{W}}\left(G\right) & \cong & E_{*}G\\
G^{n+1} & \longleftrightarrow & E_{n}\left(G\right)\\
\left(g_{0},g_{0}g_{1},\cdots,g_{0}\cdots g_{n}\right) & \longleftrightarrow & \left(g_{0},\cdots,g_{n}\right)
\end{array}
\]
\end{example}
%

\subsubsection{The Kan loop group of simplicial sets}

Conversely, given $X_{*}\in Ob\left(\mathbf{sSet}_{0}\right)$ a reduced
simplicial set, define the Kan loop group of $X$, $\mathbb{G}\left(X\right)_{*}\in Ob\left(\mathbf{sGr}\right)$
by 
\[
\mathbb{G}_{n}\left(X\right)\coloneqq\mathbb{F}\left\langle X_{n+1}\right\rangle /\left(s_{0}\left(x\right)=1,\forall x\in X_{n}\right)
\]
induced by 
\[
B_{n}=X_{n+1}\backslash s_{0}\left(X_{n}\right)\hookrightarrow X_{n+1}
\]
(but $\left\{ B_{n}\right\} $ do not form a simplicial set), with
\[
\begin{array}{ccl}
d_{i}^{\mathbb{G}}\left(x\right) & = & \begin{cases}
d_{1}\left(x\right)d_{0}\left(x\right)^{-1} & i=0,\\
d_{i+1}x & i>0.
\end{cases}\\
s_{j}^{\mathbb{G}}\left(x\right) & = & s_{j+1}x\ \ \ \ \ \ \ \ \ \ \ \qquad j\geq0.
\end{array}
\]

Define 
\[
\tau\left(X\right):X_{*}\to\mathbb{G}\left(X\right)_{*-1}
\]
by 
\[
\tau_{n}\left(X\right):X_{n}\hookrightarrow\mathbb{F}\left\langle X_{n}\right\rangle \twoheadrightarrow\mathbb{G}X_{n-1}.
\]
 

Given $X_{*}\in Ob\left(\mathbf{sSet}\right)$ and $G_{*}\in Ob\left(\mathbf{sGr}\right)$
define 
\[
\text{Tw}\left(X_{*},G_{*}\right)\coloneqq\left\{ \text{twisting functions }\tau:X_{*}\to G_{*}\right\} .
\]

\begin{thm}
\label{Thm-bij} There are natural bijections 
\[
\begin{array}{ccccc}
\text{Hom}_{\mathbf{sGr}}\left(\mathbb{G}X_{*},G_{*}\right) & \xrightarrow{\sim} & \text{Tw}\left(X_{*},G_{*}\right) & \xleftarrow{\sim} & \text{Hom}_{\mathbf{sSet}}\left(X_{*},\overline{\mathcal{W}}G_{*}\right)\\
f & \mapsto & f\circ\tau\left(X\right)\\
 &  & \tau\left(G)\circ g\right) & \mapsfrom & g
\end{array}
\]
Hence we have adjunction 
\[
\xymatrix{\mathbb{G}:\mathbf{sSet}_{0}\ar@<1ex>[r] & \mathbf{sGr}:\overline{\mathcal{W}}\ar@<1ex>[l]}
.
\]
\end{thm}
Theorem \ref{Thm-bij}, proposition \ref{prop-fib} and lemma \ref{lem-twist}
implies
\begin{cor}
For fixed $G_{*}\in Ob\left(\mathbf{sGr}\right),X\in Ob\left(\mathbf{sSet}\right)$,
there is a natural bijection between the set of twisting function
$\text{Tw}\left(X_{*},G_{*}\right)$ and the isomorphism classes of
pairs $\left(E_{*},G_{*}\right)$ where $E_{*}$ is a principal $G_{*}$-bundle
over $X_{*}$ with local section $\sigma:X_{*}\to E_{*}$. The bijection
is given by 
\[
\tau\mapsto\left(G\times_{\tau}X,\sigma\right)
\]
where $d_{0}\sigma_{n}\left(x\right)=\sigma_{n-1}\left(d_{0}x\right)\tau\left(x\right)$. 
\end{cor}
%
Our main theorem is the following. 
\begin{thm}
(Kan) 
\begin{enumerate}
\item For $X_{*}\in Ob\left(\mathbf{sSet}_{0}\right)$ a reduced simplicial
set and $G_{*}\in Ob\left(\mathbf{sGr}\right)$, there are weak homotopy
equivalences of spaces 
\[
\begin{array}{ccc}
\left|\mathbb{G}\left(X_{*}\right)\right| & \simeq & \Omega\left|X_{*}\right|\\
\left|\overline{\mathcal{W}}\left(G_{*}\right)\right| & \simeq & B\left|G_{*}\right|
\end{array}
\]
This shows that the homotopy type of $\left|\mathbb{G}\left(X_{*}\right)\right|$
is the loop space of $\left|X\right|$, and the homotopy type of $\left|\overline{\mathcal{W}}\left(G_{*}\right)\right|$
is the classifying space of $\left|G_{*}\right|$, which is the reason
for the name of the two functors. 
\item The adjoint functors $\left(\mathbb{G},\overline{\mathcal{W}}\right)$
give Quillen pair of model categories 
\[
\xymatrix{\mathbb{G}:\mathbf{sSet}_{0}\ar@<1ex>[r] & \mathbf{sGr}:\overline{\mathcal{W}}\ar@<1ex>[l]}
\]
and 
\[
\text{Ho}\left(\mathbf{sSet}_{0}\right)\cong\text{Ho}\left(\mathbf{sGr}\right).
\]
\end{enumerate}
\end{thm}
%

\subsection{Geometric Bar Construction }

Let $\mathcal{U}$ be the category of compactly generated weak Hausdorff
spaces. Let $\mathcal{W}$ be the full subcategory of $\mathcal{U}$
of spaces having the same homotopy type of CW complexes. 

\subsubsection{Neighborhood Deformation Retract}

The object is to define a category of pairs $\left(X,A\right)$ where
$X\in Ob\left(\mathcal{U}\right)$ and $A$ is a closed subset in
$X$ having certain useful properties. Most important is that each
$\left(X,A\right)$ should have the homotopy extension property (i.e.
the inclusion $A\subset X$ is a cofibration). Also we require that
the category should be closed under the operations of forming products
and adjunction spaces. 
\begin{defn}
\cite{key-3-1} A closed subset $A$ of a space $X\in Ob\left(\mathcal{U}\right)$
is called a \emph{neighborhood deformation retract }(NDR) in $X$
if there exists a map $u:X\to I$ such that $A=u^{-1}\left(0\right)$
and a homotopy $h:I\times X\to X$ such that $h\left(0,x\right)=x$
for all $x\in X$, $h\left(t,a\right)=a$ for all $\left(t,a\right)\in I\times A$,
and $h\left(1,x\right)\in A$ for all $x\in u^{-1}[0,1)$. The pair
$\left(X,A\right)$ is called an \emph{NDR-pair}. The pair $\left(h,u\right)$
is said to be a \emph{representation} of $\left(X,A\right)$ as an
NDR-pair. If in addition, $h\left(1,x\right)\in A$ for all $x\in X$,
then $A$ is a \emph{deformation retract} of $X$ and $\left(X,A\right)$
is a \emph{DR-pair}. An NDR-pair $\left(X,A\right)$ is a \emph{strong
NDR-pair} if $u\circ h\left(t,x\right)<1$ whenever $u\left(x\right)<1$;
thus if $B=u^{-1}[0,1)$, it is required that $\left(h,u\right)$
restrict to a representation of $\left(B,A\right)$ as a DR-pair.
\end{defn}
%
The pair $\left(X,\emptyset\right)$ is always an NDR pair. $\left(X,X\right)$
and $\left(I,\left\{ 0\right\} \right)$ is always a DR pair. 
\begin{thm}
\label{Thm-NDR-prod} If $\left(X,A\right)$ and $\left(Y,B\right)$
are NDR pairs, then so is their product 
\[
\left(X,A\right)\times\left(Y,B\right)=\left(X\times Y,X\times B\cup A\times Y\right).
\]
If in addition, one of them is a DR pair, then their product is a
DR pair. 
\end{thm}
\begin{proof}
Let $\left(h,u\right)$ and $\left(j,v\right)$ be the representations
of $\left(X,A\right)$ and $\left(Y,B\right)$ respectively. Define
\[
\begin{array}{cccc}
w: & X\times Y & \to & I\\
 & \left(x,y\right) & \mapsto & u\left(x\right)v\left(y\right)
\end{array}
\]
then $X\times B\cup A\times Y=w^{-1}\left(0\right)$. Define the homotopy
\[
\begin{array}{cccc}
q: & I\times X\times Y & \to & X\times Y\\
 & \left(t,x,y\right) & \mapsto & \begin{cases}
\left(x,y\right) & x\in A,y\in B\\
\left(h\left(t,x\right),j\left(\frac{u\left(x\right)}{v\left(y\right)}t,y\right)\right) & u\left(x\right)\leq v\left(y\right),v\left(y\right)>0\\
\left(h\left(\frac{v\left(y\right)}{u\left(x\right)}t,x\right),j\left(t,y\right)\right) & v\left(y\right)\leq u\left(x\right),u\left(x\right)>0
\end{cases}
\end{array}
\]
The domain of the last two lanes intersect in the relatively closed
set where $u\left(x\right)=v\left(y\right)>0$ which both reduce to
$\left(h\left(t,x\right),j\left(t,y\right)\right)$, thus $q$ is
continuous on $I\times\left(X\times Y-A\times B\right)$. Thus we
just need to verify that $q$ is continuous at a point $\left(t,x,y\right)$
in $I\times A\times B$. Let $U\ni x,V\ni y$ be open sets in $X$
and $Y$ respectively, then we have $I\times\left\{ x\right\} \in h^{-1}\left(U\right)$
and $I\times\left\{ y\right\} \in j^{-1}\left(V\right)$. Since $I$
is compact and $h^{-1}\left(U\right)$ is open, there is an open neighborhood
$S\ni x$ such that $I\times S\subset h^{-1}\left(U\right)$. Similarly,
there is an open neighborhood $T\ni y$ such that $I\times T\subset j^{-1}\left(V\right)$,
so $q\left(I\times S\times T\right)\subset U\times V$. Thus $q$
is continuous. 

We also see that $q\left(t,x,y\right)=\left(x,y\right)$ whenever
$x\in A$ (equivalently, $u\left(x\right)=0$)or $y\in B$ (equivalently,
$v\left(y\right)=0$). 

When $t=1$, suppose that $0<w\left(x,y\right)<1$. 
\begin{enumerate}
\item If $u\left(x\right)<1$, we have 
\begin{enumerate}
\item $u\left(x\right)\leq v\left(y\right)$, then $q\left(1,x,y\right)=\left(h\left(1,x\right),j\left(\frac{u\left(x\right)}{v\left(y\right)},y\right)\right)\in A\times Y$.
\item $v\left(y\right)\leq u\left(x\right)$, then $q\left(1,x,y\right)=\left(h\left(\frac{v\left(y\right)}{u\left(x\right)},x\right),j\left(1,y\right)\right)\in X\times B$.
\end{enumerate}
\item If $v\left(y\right)<1$, the proof is similar. 
\end{enumerate}
Thus $\left(X,A\right)\times\left(Y,B\right)=\left(X\times Y,X\times B\cup A\times Y\right)$
is a NDR pair. 

In addition, if $\left(h,u\right)$ represents $\left(X,A\right)$
as a DR pair, replace $u$ by $u'=\frac{1}{2}u$, then $\left(h,u'\right)$
also represents $\left(X,A\right)$ as a DR pair, then the product
is a DR pair. 
\end{proof}
\begin{thm}
If $X$ is CGWH and $A$ is closed in $X$, then TFAE: 
\begin{enumerate}
\item $\left(X,A\right)$ is an NDR, 
\item $\left\{ 0\right\} \times X\cup I\times A$ is a DR of $I\times X$, 
\item $\left\{ 0\right\} \times X\cup I\times A$ is a retract of $I\times X$, 
\item $\left(X,A\right)$ has the homotopy extension property, i.e. the
inclusion $A\subset X$ is a cofibration. 
\end{enumerate}
\end{thm}
\begin{proof}
$1\implies2$: By previous theorem \ref{Thm-NDR-prod}. 

$2\implies3$: trivial. 

$3\implies4$: trivial. 

$4\implies1$: 
\end{proof}

\subsubsection{Geometric Realization}

Let $\Delta^{n}$ denote the standar topological $n$-simplex 
\[
\Delta^{n}=\left\{ \left(x_{0},\cdots,x_{n}\right)\in\mathbf{R}^{n+1}|\sum x_{i}=1,x_{i}\geq0\right\} .
\]
Define $\delta_{i}:\Delta^{n-1}\to\Delta^{n}$ and $\sigma_{j}:\Delta^{n+1}\to\Delta^{n}$
for $0\leq i,j\leq n$ by 
\[
\begin{array}{ccl}
\delta_{i}\left(x_{0},\cdots,x_{n-1}\right) & = & \left(x_{0},\cdots,x_{i-1},0,x_{i},\cdots,x_{n-1}\right)\\
\sigma_{j}\left(x_{0},\cdots,x_{n+1}\right) & = & \left(x_{0},\cdots,x_{i-1},x_{i}+x_{i+1},\cdots,x_{n+1}\right)
\end{array}
\]

\begin{defn}
Given a simplicial object $X$ in $\mathcal{U}$, $X\in Ob\left(s\mathcal{U}\right)$,
the \emph{geometric realization} $\left|X\right|$ of $X$ is a topological
space defined as 
\[
\left|X\right|=\coprod_{n\geq0}X_{n}\times\Delta^{n}/\sim
\]
where $\left(\partial_{i}x,u\right)\sim\left(x,\delta_{i}u\right)$
for $x\in X_{n},u\in\Delta^{n-1}$ and $\left(s_{j}x,u\right)\sim\left(x,\sigma_{j}\right)$
for $x\in X_{n},u\in\Delta^{n+1}$. The topology on $X$ is given
by the coproduct topology of the quotient topology of the product
topology. 
\end{defn}
%
If $X$ is a simplicial set, then the classical geometric realization
of $X$ coincides with the geometric realization of $X$ as a discrete
simplicial space. 
\begin{defn}
Let $X$ be a simplicial space, $X\in Ob\left(s\mathcal{U}\right)$,
define $sX_{n}={\displaystyle \bigcup_{j=0}^{n}s_{j}X_{n}\subset X_{n+1}}$.
We say $X$ is \emph{proper} if each $\left(X_{n+1},sX_{n}\right)$
is a strong NDR pair and $X$ is \emph{strictly proper} if in addition,
each $\left(X_{n+1},s_{k}X_{n}\right),0\leq k\leq n$ is a strong
NDR pair via a homotopy $h:I\times X_{n+1}\to X_{n+1}$ such that
\[
h\left(I\times\bigcup_{j=0}^{k-1}s_{j}X_{n}\right)\subset\bigcup_{j=1}^{k-1}s_{j}X_{n}
\]
\end{defn}
\begin{thm}
\label{Thm-connect} Fix $n\geq0$. If $X$ is a (strictly) proper
simplicial space scuh that each $X_{i}$ is $\left(n-i\right)$-connected
for all $i\leq n$, then $\left|X\right|$ is $n$-connected. 
\end{thm}
\begin{proof}
 
\end{proof}
\begin{defn}
An object $X\in Ob\left(s\mathcal{U}\right)$ is \emph{cellular} if
each $X_{n}$ is a CW complex and each $\partial_{i},s_{j}$ is a
cellular map. 
\end{defn}
\begin{thm}
For $X,Y\in Ob\left(s\mathcal{U}\right)$, the map $\left|\pi_{1}\right|\times\left|\pi_{2}\right|:\left|X\times Y\right|\to\left|X\right|\times\left|Y\right|$
is a natural homeomorphism whose inverse $\zeta$ is commutative and
associative and is cellular if $X$ and $Y$ are cellular. 
\end{thm}

\subsubsection{Classifying Spaces of Topological Monoid}

Let $G$ be a topological monoid such that its identity element $e$
is a strongly nondegenerated basepoint (in the sense that $\left(G,e\right)$
is a strong NDR-pair). 

Let $X$ and $Y$ be left and right $G$-spaces. Define a simplicical
topological space $B_{*}\left(Y,G,X\right)$ by letting the space
of $j$-simplices be $Y\times G^{j}\times X$ with elements $y\left[g_{1},\cdots,g_{j}\right]x$
and the face and degeneracy maps are 
\[
\partial_{i}\left(y\left[g_{1},\cdots,g_{j}\right]x\right)=\begin{cases}
yg_{1}\left[g_{2},\cdots,g_{j}\right]x & i=0\\
y\left[g_{1},\cdots,g_{i-1},g_{i}g_{i+1},g_{i+2},\cdots,g_{j}\right]x & 1\leq i<j\\
y\left[g_{1},\cdots,g_{j-1}\right]g_{j}x & i=j
\end{cases}
\]
and 
\[
s_{j}\left(y\left[g_{1},\cdots,g_{j}\right]x\right)=y\left[g_{1},\cdots,g_{i},e,g_{i+1},\cdots,g_{j}\right]x.
\]

Let $B\left(Y,G,X\right)$ be the geometric realization of $B_{*}\left(Y,G,X\right)$.
Then $B$ is a functor 
\[
\begin{array}{cccc}
B: & \mathcal{A}\left(\mathcal{U}\right) & \to & \mathcal{U}\\
 & \left(Y,G,X\right) & \mapsto & \left|B_{*}\left(Y,G,X\right)\right|
\end{array}
\]
where $\mathcal{A}\left(\mathcal{U}\right)$ is the category where
objects are triples $\left(Y,G,X\right)$ and morphisms are $\left(k,f,j\right):\left(Y,G,X\right)\to\left(Y',G',X'\right)$
where $f:G\to G'$ is a map of topological monoids, and $k:Y\to Y'$
and $j:X\to X'$ are $f$-equivariant maps, i.e. $j\left(gx\right)=f\left(g\right)j\left(x\right)$
and $k\left(yg\right)=k\left(y\right)f\left(g\right)$. 

Let $*$ be the one-point $G$-space and define 
\[
\begin{array}{ccc}
BG & = & B\left(*,G,*\right)\\
EG & = & B\left(*,G,G\right)
\end{array}
\]
then $BG$ is the standard classifying space of $G$. 

Now we state basic facts about the topological behavior of the functor
$B$. 
\begin{prop}
$B_{*}\left(Y,G,X\right)$ is a proper simplicial space. $B\left(Y,G,X\right)$
is $n$-connected if $G$ is $\left(n-1\right)$-connected and $X$
and $Y$ are $n$-connected. 
\end{prop}
\begin{proof}
The first statement is equivalent to say that 
\[
\left(Y,\emptyset\right)\times\left(G,e\right)^{j}\times\left(X,\emptyset\right)
\]
 is a strong NDR pair, which follows from Theorem \ref{Thm-NDR-prod}. 

The second statement follows from Theorem \ref{Thm-connect}. 
\end{proof}
\begin{prop}
If $Y,G,X$ are in $\mathcal{W}$ then so is $B\left(Y,G,X\right)$. 
\end{prop}
%
\begin{prop}
Let $\left(k,f,j\right):\left(Y,G,X\right)\to\left(Y',G',X'\right)$
be a morphism in $\mathcal{A}\left(\mathcal{U}\right)$. 
\begin{enumerate}
\item If $k,f,j$ induces isomorphsims on integral homology, then so does
$B\left(k,f,j\right)$. 
\item If $k,f,j$ are homotopy equivalences, then so is $B\left(k,f,j\right)$. 
\end{enumerate}
\end{prop}
%
\begin{prop}
$B$ preserves products. 
\end{prop}
\begin{defn}
A map $p:E\to B$ is a \emph{quasifibration} if $p$ is onto and 
\[
p_{*}:\pi_{i}\left(E,p^{-1}\left(x\right),y\right)\to\pi_{i}\left(B,x\right)
\]
is an isomorphism for all $x\in B$, $y\in p^{-1}\left(x\right)$
and $i\geq0$. A subset $U\subseteq B$ is \emph{distinguished} if
$p:p^{-1}\left(U\right)\to U$ is a quasi-fibration. 
\end{defn}
%
\begin{lem}
Let $p:E\to B$ be a map onto a filtered space $B$, then each $F_{j}B$
is distinguished and $p$ is a quasifibration provided that 
\begin{enumerate}
\item $F_{0}B$ and each open subset of $F_{j}B-F_{j-1}B$ for $j>0$ is
distinguished. 
\item For each $j>0$, there is an open subset $U$ of $F_{j}B$ which contains
$F_{j-1}B$ and there are homotopies $h_{t}:U\to U$ and $H_{t}:p^{-1}\left(U\right)\to p^{-1}\left(U\right)$
such that 
\begin{enumerate}
\item $h_{0}=1,h_{t}\left(F_{j-1}B\right)\subset F_{j-1}B$ and $h_{1}\left(U\right)\subset F_{i-1}B$. 
\item $H_{0}=1$ and $H$ covers $h$, $pH_{t}=h_{t}p$; and
\item $H_{1}:p^{-1}\left(x\right)\to p^{-1}\left(h_{1}\left(x\right)\right)$
is a weak homotopy equivalence for all $x\in U$. 
\end{enumerate}
\end{enumerate}
\end{lem}
%
Let $p:B\left(Y,G,X\right)\to B\left(Y,G,*\right)$ and $q:\left(Y,G,X\right)\to B\left(*,G,X\right)$
be the maps induced by the trivial $G$-maps $X\to*$ and $Y\to*$. 
\begin{thm}
If $G$ is grouplike, then $p$ and $q$ are quasi-fibrations. 
\end{thm}
\begin{defn}
A cover $\mathscr{C}$ of a space $B$ is \emph{numerable} if it is
locally finite and if for each $U\in\mathscr{C}$, there is a map
$\lambda_{U}:B\to I$ such that $U=\lambda_{U}^{-1}(0,1]$.
\end{defn}
\pagebreak{}
\begin{thebibliography}{M1}
\bibitem[M1]{key-1-1} J. Milnor, Construction of universal bundles.
I, Ann. of Math. (2) 63 (1956), 272-284. 

\bibitem[M2]{key-1-2} J. Milnor, Construction of universal bundles.
II, Ann. of Math. (2) 63 (1956), 430\textendash 436. 

\bibitem[Se]{key-2} G. B. Segal, Classifying spaces and spectral
sequences, Publ. Math. IHES 34 (1968), 105-112. 

\bibitem[M1]{key-3-1} {]} J. P. May. The Geometry of Iterated Loop
Spaces. Springer Lecture Notes in Mathematics Volume 271 (1972). 

\bibitem[M2]{key-3-2} J. P. May. Classifying spaces and fibrations.
Memoirs Amer. Math. Soc. 155 (1975). 

\bibitem[M3]{key-3-3} J. P. May. $E_{\infty}$-spaces, group completions,
and permutative categories. London Math. Soc. Lecture Notes No.11
(1974), 61-93. 

\bibitem[St]{key-4} N. E. Steenrod, A convenient category of topological
spaces. Mich. Math J 14 (1967), 133-152. 

\bibitem{key-13} AA
\end{thebibliography}

\end{document}
